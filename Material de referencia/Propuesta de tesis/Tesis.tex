\documentclass[]{article}
\usepackage{lmodern}
\usepackage{amssymb,amsmath}
\usepackage{ifxetex,ifluatex}
\usepackage{fixltx2e} % provides \textsubscript
\ifnum 0\ifxetex 1\fi\ifluatex 1\fi=0 % if pdftex
  \usepackage[T1]{fontenc}
  \usepackage[utf8]{inputenc}
\else % if luatex or xelatex
  \ifxetex
    \usepackage{mathspec}
  \else
    \usepackage{fontspec}
  \fi
  \defaultfontfeatures{Ligatures=TeX,Scale=MatchLowercase}
\fi
% use upquote if available, for straight quotes in verbatim environments
\IfFileExists{upquote.sty}{\usepackage{upquote}}{}
% use microtype if available
\IfFileExists{microtype.sty}{%
\usepackage{microtype}
\UseMicrotypeSet[protrusion]{basicmath} % disable protrusion for tt fonts
}{}
\usepackage[margin=1in]{geometry}
\usepackage{hyperref}
\PassOptionsToPackage{usenames,dvipsnames}{color} % color is loaded by hyperref
\hypersetup{unicode=true,
            pdftitle={Determinación de modelos ARIMA vía sobre parametrización según la temporalidad de la serie cronológica con aplicaciones en datos costarricenses},
            colorlinks=true,
            linkcolor=blue,
            citecolor=Blue,
            urlcolor=blue,
            breaklinks=true}
\urlstyle{same}  % don't use monospace font for urls
\usepackage{graphicx,grffile}
\makeatletter
\def\maxwidth{\ifdim\Gin@nat@width>\linewidth\linewidth\else\Gin@nat@width\fi}
\def\maxheight{\ifdim\Gin@nat@height>\textheight\textheight\else\Gin@nat@height\fi}
\makeatother
% Scale images if necessary, so that they will not overflow the page
% margins by default, and it is still possible to overwrite the defaults
% using explicit options in \includegraphics[width, height, ...]{}
\setkeys{Gin}{width=\maxwidth,height=\maxheight,keepaspectratio}
\IfFileExists{parskip.sty}{%
\usepackage{parskip}
}{% else
\setlength{\parindent}{0pt}
\setlength{\parskip}{6pt plus 2pt minus 1pt}
}
\setlength{\emergencystretch}{3em}  % prevent overfull lines
\providecommand{\tightlist}{%
  \setlength{\itemsep}{0pt}\setlength{\parskip}{0pt}}
\setcounter{secnumdepth}{0}
% Redefines (sub)paragraphs to behave more like sections
\ifx\paragraph\undefined\else
\let\oldparagraph\paragraph
\renewcommand{\paragraph}[1]{\oldparagraph{#1}\mbox{}}
\fi
\ifx\subparagraph\undefined\else
\let\oldsubparagraph\subparagraph
\renewcommand{\subparagraph}[1]{\oldsubparagraph{#1}\mbox{}}
\fi

%%% Use protect on footnotes to avoid problems with footnotes in titles
\let\rmarkdownfootnote\footnote%
\def\footnote{\protect\rmarkdownfootnote}

%%% Change title format to be more compact
\usepackage{titling}

% Create subtitle command for use in maketitle
\providecommand{\subtitle}[1]{
  \posttitle{
    \begin{center}\large#1\end{center}
    }
}

\setlength{\droptitle}{-2em}

  \title{Determinación de modelos ARIMA vía sobre parametrización según la
temporalidad de la serie cronológica con aplicaciones en datos
costarricenses}
    \pretitle{\vspace{\droptitle}\centering\huge}
  \posttitle{\par}
    \author{Universidad de Costa Rica\\
César Gamboa Sanabria\\
www.cesargamboasanabria.com\\
\href{mailto:info@cesargamboasanabria.com}{\nolinkurl{info@cesargamboasanabria.com}}}
    \preauthor{\centering\large\emph}
  \postauthor{\par}
      \predate{\centering\large\emph}
  \postdate{\par}
    \date{01 enero, 2020}

\usepackage{geometry}
\geometry{a4paper, left=20mm, right=30mm}
\usepackage{setspace}
\doublespacing
\usepackage[spanish]{babel}
\usepackage{color}
\usepackage{xcolor}
\usepackage{framed}
\colorlet{shadecolor}{gray!20}
\setcounter{secnumdepth}{0}
\usepackage{sectsty}
\chapternumberfont{\Large}
\chaptertitlefont{\Large}
\setcounter{tocdepth}{5}
\setcounter{secnumdepth}{5}
\usepackage{graphics}
\usepackage{setspace} %paquete para el doble espaciado
\doublespacing %inicia el doble espaciado
%Esto quita el punto final en la numeracion de cada seccion
\usepackage{tocloft}
\usepackage{titlesec}
\titleformat{\section}
{\Large\bfseries}{\thesection}{0.5em}{}
\titleformat{\subsection}
{\large\bfseries}{\thesubsection}{0.5em}{}
\titleformat{\subsubsection}
{\normalsize\bfseries}{\thesubsubsection}{0.5em}{}
\titleformat{\paragraph}
{\normalsize\bfseries}{\theparagraph}{0.5em}{}
\renewcommand\cftsecaftersnum{}
\renewcommand\thesection{\arabic{section}}
\renewcommand\thesubsection{\thesection.\arabic{subsection}}
\usepackage{caption}
\usepackage{fancyhdr}
\pagestyle{fancy}
\fancyhf{}
\fancyhead[R]{\rightmark}
\fancyfoot[C]{\thepage}
\setlength{\headheight}{21.9pt}
\renewcommand\sectionmark[1]{
\markright{\thesection\ #1}}

\begin{document}
\maketitle

\newpage

\section*{RESUMEN}

La metodología de Box-Jenkins busca encontrar el mejor proceso
autorregresivo integrado de medias móviles (ARIMA) que explique una
serie temporal \(y_t\) de \(T\) periodos, para pronosticar hasta
\(T+h\). El paquete \texttt{forecast} de R permite hacer uso de la
función \texttt{auto.arima()} para estimar un modelo ARIMA basado en
pruebas de raíz unitaria, minimización del AICc y de la MLE. De esta
forma se obtiene un modelo temporal definiendo las diferenciaciones
requeridas en la parte estacional d mediante las pruebas KPSS o ADF, y
la no estacional D utilizando las pruebas OCSB o la Canova-Hansen,
seleccionado el orden óptimo para los términos \(ARIMA(p,d,q)(P,D,Q)\)
para una serie \(y_t\). Se propone un método de selección fundamentada
en las permutaciones de los parámetros de un modelo ARIMA, seleccionando
la mejor especificación con base en medidas de rendimiento MAE, RMSE,
MAPE y MASE: se comparan todos los posibles términos definiendo una
diferenciación adecuada para la serie y permutando hasta un máximo
determinado para los términos de especificación de un
\(ARIMA(p,d,q)(P,D,Q)\). El método propuesto se probó en 6 series
cronológicas de distinta temporalidad: mortalidad infantil, mortalidad
por causa externa, nacimientos, demanda eléctrica, intereses y
comisiones del sector público, e incentivos salariales del sector
público.

\textbf{\emph{Palabras clave}}: ARIMA, R, automatización, selección,
estadística

\newpage

\tableofcontents

\newpage

\section{INTRODUCCIÓN}

El manejo de información obtenida de manera secuencial a lo largo del
tiempo hace referencia al uso de series cronológicas. Este tipo de datos
se encuentras en diferentes áreas de investigación. En el campo
financiero, por ejemplo, es común hablar de la devaluación del colón con
respecto al dólar, cantidad de exportaciones mensuales de un determinado
producto o las ventas de este (Hernández,
\protect\hyperlink{ref-oscarh-1}{2011}\protect\hyperlink{ref-oscarh-1}{a}).

En demografía, por ejemplo, el tema de las proyecciones de población
tiene un alto impacto a nivel social, pues conocer con anticipación en
posible comportamiento de la población en el futuro es clave para una
adecuada planificación en diversos proyectos sobre los cuales se debe
distribuir un presupuesto que es finito. Durante una emergencia, que
difícilmente se sabe cuándo ocurrirá, conocer la posible población que
se tiene en una zona es clave para la rápida reacción de las autoridades
para el envío de ayuda o para ejecutar planes de evacuación.

El campo actuarial también se ve beneficiado al mejorar sus métodos de
pronóstico, pues uno de sus campos de estudio es la mortalidad pues
representan un insumo de vital importancia para la planificación y
sostenibilidad de los sistemas de pensiones, servicios de salud tanto
pública como privada, seguros de vida y asuntos hipotecarios
(Rosero-Bixby, \protect\hyperlink{ref-supenprodc}{2018}).

Sin embargo, las series cronológicas por sí solas representan solo un
insumo para abordar, como mínimo, tres objetivos básicos: 1) realizar
análisis exploratorios de estos datos mediante métodos de visualización
y medidas de posición y variabilidad, como ver su crecimiento o
decrecimiento a lo largo del tiempo, detectar valores atípicos o cambios
drásticos en el nivel o valor medio de la serie, 2) generar modelos
estadísticos que sirvan como un simplificación de la realidad, y 3)
generar pronósticos para los posibles valores futuros que tomará el
problema en cuestión (Hernández,
\protect\hyperlink{ref-oscarh-2}{2011}\protect\hyperlink{ref-oscarh-2}{b}).

Los tres objetivos anteriores se trabajan de manera secuencial, pues es
necesario realizar primero el análisis exploratorio de los datos para
tener una noción global del panorama y así conocer la serie cronológica
con la que se está trabajando. Una vez hecho esto, existen múltiples
formas de generar modelos para estos datos, como por ejemplo los métodos
de suavizamiento exponencial desarrollados en la década de 1950 (Brown,
\protect\hyperlink{ref-brown}{1956}), modelos de regresión para series
temporales (Kedem \& Fokianos, \protect\hyperlink{ref-kedem}{2005}) o
los procesos autorregresivos integrados de medias móviles (ARIMA) (Box,
Jenkins, \& Reinsel, \protect\hyperlink{ref-box-jenkins}{1994}). Cuando
se ha establecido el modelo, los pronósticos son utilizados en
instituciones públicas, gobiernos municipales, instituciones del sector
privado, centros académicos, población civil, centros nacionales o
regionales de investigación y ONG dedicadas al desarrollo social. Si las
entidades previamente mencionadas cuentan con proyecciones de calidad,
la puesta en marcha de sus respectivos planes tendrá un impacto mayor y
más efectivo.

De lo anterior, generar un modelo adecuado es fundamental para obtener
un pronóstico de calidad, y es aquí donde resulta importante mencionar
una diferencia clave entre los dos modelos clásicos más comúnmente
utilizados: los modelos de suavizamiento y los modelos ARIMA. Ambos
representan enfoques complementarios a un problema, pues según Hyndman
(R. J. Hyndman \& Athanasopoulos,
\protect\hyperlink{ref-hyndman2018forecasting}{2018}\protect\hyperlink{ref-hyndman2018forecasting}{a}),
los modelos de suavizamiento exponencial se fundamentan en un enfoque
más descriptivo de los componentes de la serie cronológica en estudio;
mientras que los modelos ARIMA tienen como objetivo explicar las
relaciones pasadas de ésta. La importancia de la metodología de
Box-Jenkins radica en que no supone ningún patrón en particular en la
serie histórica que se busca pronosticar, sino que contempla un proceso
iterativo para identificar un posible modelo a partir de una clase
general de modelos y luego someter dicho modelo a diferentes pruebas y
medidas de rendimiento para evaluar su ajuste. Al trabajar la
metodología de Box-Jenkins, uno de los pasos es identificar el los
parámetros del proceso ARIMA(p,d,q)(P,D,Q) que gobiernan la serie,
siendo la manera clásica de trabajar este paso, el análisis visual de
las funciones de autocorrelación parcial y total.

El gran obstáculo que presenta esta identificación visual es que en la
actualidad contar con una gran cantidad de series cronológicas para
analizar es algo muy común. Incluso con cantidades moderadas de series
cronológicas a analizar, es difícil contar con personal capacitado para
realizar este análisis visual y poder identificar los modelos, por lo
que la generación de algoritmos que ayuden a dicha identificación se
vuelven cada vez más necesarios (Hyndman \& Khandakar,
\protect\hyperlink{ref-auto.arima}{2008}).

Han sido varias las aproximaciones a un método que genere de manera
automática un modelo ARIMA, como por ejemplo los propuestos por Hannan y
Rissanen (Hannan \& Rissanen, \protect\hyperlink{ref-hannan}{1982}), la
extensión de dicha propuesta realizada por Gómez (Gómez,
\protect\hyperlink{ref-gomez}{1998}) y posteriormente aplicada (Gómez \&
Maraval, \protect\hyperlink{ref-tramo}{1998}) en los software
\textbf{TRAMO} y \textbf{SEATS}; de manera similar se planteó una
aplicación en los software \textbf{SCA-Expert} (Liu,
\protect\hyperlink{ref-liu}{1989}) y \textbf{TSE-AX} (Mélard \&
Pasteels, \protect\hyperlink{ref-melard}{2000}). Otros algoritmos
implementados en programas de cómputo de paga son \textbf{Forecast Pro}
(Goodrich, \protect\hyperlink{ref-forecastpro}{2000}) y \textbf{Autobox}
(Reilly, \protect\hyperlink{ref-autobox}{2000}). Uno de los métodos
automatizados de estimación es el que ofrece el paquete
\texttt{forecast} (Hyndman \& Khandakar,
\protect\hyperlink{ref-auto.arima}{2008}) del lenguaje de programación
R\footnote{\url{https://cran.r-project.org/}} permite hacer uso de la
función \texttt{auto.arima()} para estimar un modelo ARIMA basado en
pruebas de raíz unitaria, minimización del AICc y de la MLE. De esta
forma se obtiene un modelo temporal definiendo las diferenciaciones
requeridas en la parte estacional d mediante las pruebas KPSS o ADF, y
la no estacional D utilizando las pruebas OCSB o la Canova-Hansen,
seleccionado el orden óptimo para los términos \(ARIMA(p,d,q)(P,D,Q)_s\)
para una serie cronológica determinada.

Es a partir de esta necesidad que se propone una metodología para la
estimación del mejor modelo ARIMA para una serie cronológica determinada
cuya temporalidad sea mensual, bimensual, trimestral o cuatrimestral
mediante un proceso de selección fundamentada en las permutaciones de
todos los parámetros de un modelo ARIMA hasta un cierto límite,
considerando además la inclusión semi-automática de intervenciones en
periodos específicos y la validación cruzada para evaluar la calidad de
las particiones de la base de datos en conjuntos para entrenar y probar
el rendimiento del modelo; dichas pruebas involucran, entre otras
medidas de rendimiento, el MAE, RMSE, MAPE y MASE, las cuales sirven de
insumo para utilizar un método de consenso entre ellas para seleccionar
el modelo más adecuado: se comparan todos los posibles términos
definiendo una diferenciación adecuada para la serie y permutando hasta
un máximo definido para los términos de especificación de un
\(ARIMA(p,d,q)(P,D,Q)_s\) para así seleccionar la especificación que
ofrezca mejores resultados al momento de pronosticar valores futuros de
la serie cronológica. El método propuesto se probará comparándose con
los resultados de 6 series con distintas temporalidades: mortalidad
infantil, mortalidad por causa externa, nacimientos, demanda eléctrica,
intereses y comisiones del sector público, e incentivos salariales del
sector público.

\subsection{Contribución de la tesis a la Estadística como disciplina}

El principal aporte de este estudio es, por medio de un estudio de
simulación, aportar evidencia sobre cómo la sobreparametrización puede
representar una herramienta para definir la especificación de un modelo
ARIMA que genere pronósticos adecuados, contrastando la calidad de estos
con respecto a otros métodos similares, como lo es la función
auto.arima().

\subsection{Objetivos}

\subsubsection{Objetivos general}

\begin{itemize}
\tightlist
\item
  Evaluar la calidad de los pronósticos realizados con modelos ARIMA
  especificados vía sobre parametrización.
\end{itemize}

\subsubsection{Objetivos específicos}

\begin{itemize}
\tightlist
\item
  Diseñar un algoritmo para la selección del mejor modelo ARIMA según la
  temporalidad de la serie.
\item
  Aplicar validación cruzada en distintos horizontes de pronóstico para
  identificar la mejor especificación de un modelo ARIMA.
\item
  Comparar la precisión de los pronósticos con el método propuesto por
  Rob Hyndman.
\item
  Integrar la metodología de análisis de series temporales en una
  librería del lenguaje estadístico R.
\end{itemize}

\section{REFERENTES O ELEMENTOS TEÓRICOS QUE VA A UTILIZAR}

\subsection{Modelos Arima}

Los modelos ARIMA, junto con los de suavizamiento exponencial, son los
de uso más extendido en el análisis de series cronológicas. El nombre
ARIMA es la abreviatura inglesa para \emph{AutoRegresive Integrated
Moving Average}, y son aplicados mediante la metodología de Box-Jenkins.
Como menciona Rob. Hyndman (R. J. Hyndman \& Athanasopoulos,
\protect\hyperlink{ref-hyndman_box-jenkins}{2018}\protect\hyperlink{ref-hyndman_box-jenkins}{b}),
la metodología de Box-Jenkins difiere a los demás métodos porque no
supone un determinado patrón en la serie cronológica, si no que parte de
un proceso iterativo para identificar el modelo de un gran grupo de
estos para luego ponerlo a prueba según varias medidas de rendimiento.
Un proceso ARIMA es caracterizado por dos funciones: la autocorrelación
y la autocorrelación parcial; el enfoque Box-Jenkins compara estas
funciones con el objetivo de identificar el proceso que describa de
manera adecuada el comportamiento de una serie cronológica (Hernández,
\protect\hyperlink{ref-oscarh-3}{2011}\protect\hyperlink{ref-oscarh-3}{c}).

El componente \textbf{AR} de los modelos ARIMA hace referencia al uso de
modelos autorregresivos, en los cuales los pronósticos para la variable
de interés utilizan una combinación lineal de las observaciones previas,
llamándose así \emph{autorregresivos} porque se aplica una regresión de
dicha variable de interés con respecto a sí misma; caso contrario a la
regresión múltiple, en donde los pronósticos se realizan con respecto a
una combinación lineal de distintos predictores. Un modelo
autorregresivo de orden \(p\) para una serie cronológica \(y_t\) puede
expresarse de la siguiente manera

\begin{equation}
y_t=c+\phi_1y_{t-1}+\phi_2y_{t-2}+\cdots+\phi_py_{t-p}+\epsilon_t
\end{equation}

Donde el término \(\epsilon_t\) representa ruido blanco. El modelo
anterior es muy similar a una regresión lineal múltiple, donde cada
coeficiente \(\phi\) va acompañado por su correspondiente rezago
\(y_{t-p}\). De manera muy similar, el término \textbf{MA} en los
modelos ARIMA se refieren a los modelos de medias móviles, los cuales
para pronosticar hacen uso de los errores; el modelo de medias móviles
puede representarse de la siguiente manera:

\begin{equation}
y_t=c+\epsilon_t+\theta_1\epsilon_{t-1}+\theta_2\epsilon_{t-2}+\cdots+\theta_q\epsilon_{t-q}
\end{equation}

Donde el término \(\epsilon_t\) representa nuevamente el ruido blanco.
La ecuación anterior representa un modelo de medias móviles de orden
\(q\), en la cual cada término \(\epsilon_{t-q}\) se entiende como una
media móvil de los \(t\) previos errores de predicción.

El componente \textbf{I} de los modelos ARIMA se refiere a
``Itegrated'', es decir, a la estacionariedad de la serie cronológica.
Tradicionalmente, la metodología de Box-Jenkins consiste en visualizar
la serie cronológica con el objetivo de, en caso de ser necesario,
transformar los datos para estabilizar la variancia y generar así un
proceso estacionario. Se dice que una serie posee un comportamiento
estacionario si el comportamiento de esta no depende del tiempo, por lo
que en principio no presentaría ningún patrón particular con respecto al
tiempo; en otras palabras, la serie posee un movimiento bastante
horizontal.

Cuando la serie cronológica muestre indicios de tendencia o patrones
estacionales que resulten en un conjunto de datos que no es estacionario
por naturaleza, es necesario realizar transformaciones sobre los datos
para hacer que la serie se vuelva estacionaria (Adhikari et~al.,
\protect\hyperlink{ref-diferenciacion}{2013}\protect\hyperlink{ref-diferenciacion}{a}).
Estas transformaciones hacen referencia al uso de logaritmos o alguna
potencia que logre estabilizar la variabilidad de la serie. Los métodos
más clásicos para identificar la no estacionariedad en una serie
cronológica son las previamente mencionadas funciones de autocorrelación
y autocorrelación parcial, las cuales sirven de indicador acerca de qué
tan relacionadas están las observaciones unas de otras. Estas funciones
ofrecen indicios sobre el orden de los términos previamente mencionados
\textbf{AR} y \textbf{MA}.

\subsection{Función de autocorrelación}

Para medir la relación lineal entre dos variables cuantitativas, es
común utilizar el coeficiente de correlación \(r\) de Pearson (Benesty
\& Chen, \protect\hyperlink{ref-pearson}{2009}), el cual se define para
dos variables \(X\) e \(Y\) como sigue:

\begin{equation}
r_{X,Y}=\frac{E(XY)}{\sigma_X \sigma_Y} = \frac{\sum_{i=1}^n \left(X_i- \bar X\right) \left(Y_i- \bar Y\right)}{\sqrt{\sum_{i=1}^n \left(X_i- \bar X\right)^2 \sum_{i=1}^n \left(Y_i- \bar Y\right)^2}}
\end{equation}

Este mismo concepto puede aplicarse a las series cronológicas para
comparar el valor de la misma en el tiempo \(t\), con su valor en el
tiempo \(t-1\), es decir, se comparan las observaciones consecutivas
\(Y_t\) con \(Y_{t-1}\). Esto también es aplicable a no solo una
observación rezagada \((Y_{t-1})\), sino también con múltiples rezagos
\((Y_{t-2}), (Y_{t-3}), \cdots,(Y_{t-n})\). Para esto se hace uso del
coeficiente de autocorrelación.

El coeficiente de autocorrelación recibe su nombre debido a que se
utiliza el coeficiente de correlación para pares de observaciones
\(r_{Y_t, Y_{t-1}}\) de la serie cronológica. Al conjunto de todas las
autocorrelaciones se le llama función de autocorrelación.

\subsection{Función de autocorrelación parcial}

La función de autocorrelación parcial busca medir la asociación lineal
entre las observaciones \(Y_t\) y \(Y_{t-k}\), descartando los efectos
de los rezagos \(1,2, \cdots ,k-1\)(Hernández,
\protect\hyperlink{ref-oscarh-4}{2011}\protect\hyperlink{ref-oscarh-4}{d})

\subsection{Modelos ARIMA no estacionales}

Como se mencionó anteriormente, los modelos ARIMA aplicados a una serie
cronológica \(Y_t\) son una combinación de un modelo autorregresivo, uno
de medias móviles, y alguna clase de diferenciación (logarítmica,
exponenciación) para así obtener una serie diferenciada \(Y’_t\). Si se
juntan ambas se obtiene un modelo ARIMA(p,d,q) que no cubre los efectos
estacionales, donde \(p\) es el orden del modelo autorregresivo, \(d\) e
el grado de la diferenciación y \(q\) es el orden del modelo de medias
móviles; y cuya estructura se muestra en la ecuación (4):

\begin{equation}
y'_t=c+\phi_1y'_{t-1}+\phi_2y'_{t-2}+\cdots+\phi_py'_{t-p}+\theta_1\epsilon_{t-1}+\theta_2\epsilon_{t-2}+\cdots+\theta_q\epsilon_{t-q} +\epsilon_t
\end{equation}

\subsection{Modelos ARIMA estacionales}

Los modelos ARIMA son también capaces de cubrir los efectos
estacionales, es decir, particularidades de la serie cronológica que se
repiten periódicamente con una cierta temporalidad (mensual, bimensual,
etc.). Para ello se incorporan términos adicionales al modelo
relacionados con la parte estacional de una manera similar a como se
incorporan en el modelo ARIMA no estacional, pero ahora considerando
retrocesos según sea la temporalidad estacional, pasando así de un
\(ARIMA(p,d,q)\) a un \(ARIMA(p,d,q)(P,D,Q)_S\), donde \(P\), \(D\) y
\(Q\) se refieren a la parte estacional y \(S\) a la temporalidad
presente en la serie.

\subsection{Medidas de rendimiento}

Cuando se tiene el modelo ARIMA estimado, es importante realizar los
pronósticos. Sin embargo, estos pronósticos no son imperativos, sino que
se debe evaluar su calidad con las llamadas medidas de rendimiento.
Estas mediciones son hechas comparando el pronóstico y su diferencia con
el valor real. Existen múltiples medidas de rendimiento, Adhikari
(Adhikari et~al.,
\protect\hyperlink{ref-medidas}{2013}\protect\hyperlink{ref-medidas}{b})
menciona las siguientes:

\subsubsection{MAE}

El error absoluto medio se define como
\(\frac{1}{n}\sum_{t=1}^n |e_t|\).

\subsubsection{MAPE}

El porcentaje promedio de error absoluto se define como
\(\frac{1}{n}\sum_{t=1}^n \left|\frac{e_t}{y_t}\right|\cdot 100\).

\subsubsection{RMSE}

Es la raíz del error cuadrático medio
\(\sqrt{\frac{1}{n}\sum_{t=1}^n |e_t^2|}\).

\subsubsection{MASE}

Para series no estacionales:
\(\frac{\frac{1}{J}\sum_j|e_j|}{\frac{1}{T-1}\sum_{t=2}^T|Y_t-Y_{t-1}|}\)

Para series estacionales:
\(\frac{\frac{1}{J}\sum_j|e_j|}{\frac{1}{T-m}\sum_{t=m+1}^T|Y_t-Y_{t-m}|}\)

Donde \(m\) es la temporalidad de la serie.

\subsubsection{AIC}

Se calcula de la siguiente manera:
\(AIC=-2logL\left(\hat\theta\right)+2k\). Donde \(k\) es el número de
parámetros y \(n\) el número de datos.

\subsubsection{AICc}

Se calcula de la siguiente manera:
\(AICc=-2logL\left(\hat\theta\right)+2k+\frac{2k+1}{n-k-1}\). Donde
\(k\) es el número de parámetros y \(n\) el número de datos.

\subsubsection{BIC}

Se calcula de la siguiente manera:
\(BIC=-2logL\left(\hat\theta\right)+k\cdot log(n)\).

Donde \(k\) es el número de parámetros y \(n\) el número de datos.

\section{FUENTE(S) DE DATOS (SI YA ESTÁN RECOLECTADOS) O DISEÑO DEL ESTUDIO (SI NO ESTÁN RECOLECTADOS)}

Los datos reales que se van a utilizar pertenecen a la Unidad de
Estadísticas demográficas del Instituto Nacional de Estadística y
Censos. Corresponden a las series históricas de la tasa de mortalidad
infantil desde el año 1970 hasta la actualidad; así como la serie
histórica de nacimientos ocurridos en Costa Rica desde 1950 hasta la
actualidad.

\section{DEFINICIÓN DE VARIABLE(S) DE ESTUDIO}

Mortalidad infantil: Se define como la muerte de seres humanos que
nacieron vivos y cuya defunción se dió antes de cumplir el primer año de
edad.

Nacimientos: Se define el acto de terminar el periodo de gestación en la
madre y llegar con vida al mundo.

\section{EVIDENCIAS DE CALIDAD DE LA MEDICIÓN PARA LA(S) VARIABLE(S) DEL ESTUDIO}

Las estadísticas vitales son sistematizadas y divulgadas año tras año,
por tanto, revelan los cambios acontecidos durante este periodo. Esta
información junto con la proveniente de los censos de población,
constituye la base para construir los diferentes índices, tasas y otros
indicadores que revelan la situación demográfica del país, información
de gran relevancia para la planificación nacional, regional y local en
diversos campos. Uno de estos principales campos o áreas de acción es la
salud pública, para la cual la tasa de mortalidad infantil se considera
uno de los indicadores prioritarios dado que refleja no solo las
condiciones de salud de la población infante, sino también los niveles
de desarrollo del país, esto porque depende de la calidad de la atención
de la salud, principalmente de la prenatal y perinatal, así como de las
condiciones de saneamiento. Por tanto, su continuo monitoreo es
fundamental para diseñar, implementar y evaluar políticas de salud
pública orientadas a disminuir y erradicar aquellas que son prevenibles
(INEC, \protect\hyperlink{ref-calidad_vitales}{2017}).

\section{MODELO(S) ESTADÍSTICOS O TÉCNICA(S) ESTADÍSTICA(S) DE ANÁLISIS A EMPLEAR}

\subsection{Selección de ARIMA vía sobreparametrización}

El método consiste en una selección fundamentada en las permutaciones de
los parámetros de un modelo ARIMA, seleccionando la mejor especificación
con base en medidas de rendimiento MAE, RMSE, MAPE y MASE: se comparan
todos los posibles términos definiendo una diferenciación adecuada para
la serie y permutando hasta un máximo determinado para los términos de
especificación de un \(ARIMA(p,d,q)(P,D,Q)\)

\subsection{Estudio de simulación}

A partir de datos reales o bien, valores aleatorios de una cierta
distribución de probabilidad, se generarán \(n\) valores aleatorios que
sigan un determinado proceso de series cronológicas.

Para generar un determinado proceso se deben fijar los valores de
\(p,d,q\) en la parte no estacional y \(P,D,Q\) en la parte estacional
de un modelo ARIMA, así como la temporalidad que se desea para la misma.
Además, se ofrece la posibilidad de definir el valor de los coeficientes
del modelo para cada orden del procese; por ejemplo, si se define un
\(ARIMA(2,1,1)(1,1,3)\), el \(2\) indica que se pueden fijar los valores
de los coeficientes \(AR(1)\) y \(AR(2)\) en, digamos, \(.2\) y \(.46\)
respectivamente; de forma análoga, pueden definirse los coeficientes
\(SMA(1)\), \(SMA(2)\) y \(SMA(3)\) en \(.4\), \(.1\) y \(.3\)
respectivamente.

\section{Software estadístico a utilizar}

Se utilizará el lenguaje de programación R (R Core Team,
\protect\hyperlink{ref-Rbase}{2019}\protect\hyperlink{ref-Rbase}{a})
mediante su interfaz RStudio para todos los procesos relacionados con la
estimación y la simulación. Se hará uso de funciones contenidas en los
paquetes \texttt{tidyr} (Wickham \& Henry,
\protect\hyperlink{ref-tidyr}{2019}), \texttt{dplyr} (Wickham et~al.,
\protect\hyperlink{ref-dplyr}{2019}) y \texttt{parallel} (R Core Team,
\protect\hyperlink{ref-parallel}{2019}\protect\hyperlink{ref-parallel}{b}).

\section{REFERENCIAS}

\hypertarget{refs}{}
\leavevmode\hypertarget{ref-diferenciacion}{}%
Adhikari, R., K, A. R., \& Agrawal, R. K. (2013a). \emph{An Introductory
Study on Time Series Modeling and Forecasting} (p. 16). Recuperado de
\url{https://arxiv.org/ftp/arxiv/papers/1302/1302.6613.pdf}

\leavevmode\hypertarget{ref-medidas}{}%
Adhikari, R., K, A. R., \& Agrawal, R. K. (2013b). \emph{An Introductory
Study on Time Series Modeling and Forecasting} (pp. 42-45). Recuperado
de \url{https://arxiv.org/ftp/arxiv/papers/1302/1302.6613.pdf}

\leavevmode\hypertarget{ref-pearson}{}%
Benesty, J., \& Chen, Y. C., J.and Huang. (2009). Pearson Correlation
Coefficient. En \emph{Noise Reduction in Speech Processing} (pp. 37-38).
\url{https://doi.org/10.1007/978-3-642-00296-0_5}

\leavevmode\hypertarget{ref-box-jenkins}{}%
Box, G. E., Jenkins, G. M., \& Reinsel, G. C. (1994). \emph{Time Series
Analysis: Forecasting and Control}. Recuperado de
\url{https://books.google.co.cr/books?id=sRzvAAAAMAAJ}

\leavevmode\hypertarget{ref-brown}{}%
Brown, R. G. (1956). \emph{Exponential Smoothing for Predicting Demand}.
Recuperado de \url{https://www.industrydocuments.ucsf.edu/docs/jzlc0130}

\leavevmode\hypertarget{ref-forecastpro}{}%
Goodrich, R. (2000). The Forecast Pro Methodology. \emph{International
Journal of Forecasting}, \emph{16}(4), 533-535. Recuperado de
\url{http://www.forecasting-competition.com/downloads/NN3/methods/Goodrich\%20(2000)\%20The\%20Forecast\%20Pro\%20methodology\%20science.pdf}

\leavevmode\hypertarget{ref-gomez}{}%
Gómez, V. (1998). \emph{Automatic Model Identification in the Presence
of Missing Observations and Outliers.} (D. G. de A. y P. P. Ministerio
de Economía y Hacienda, Ed.). Working paper D-98009.

\leavevmode\hypertarget{ref-tramo}{}%
Gómez, V., \& Maraval, A. (1998). \emph{Programs TRAMO and SEATS,
Instructions for the Users.} (D. G. de A. y P. P. Ministerio de Economía
y Hacienda, Ed.). Working paper 97001.

\leavevmode\hypertarget{ref-hannan}{}%
Hannan, E. J., \& Rissanen, J. (1982). Recursive Estimation of Mixed
Autoregressive-Moving Average Order. \emph{Biometrika}, \emph{69}(1),
81-94. Recuperado de \url{http://www.jstor.org/stable/2335856}

\leavevmode\hypertarget{ref-oscarh-1}{}%
Hernández, O. (2011a). \emph{Introducción a las Series Cronológicas}.
Recuperado de
\url{http://www.editorial.ucr.ac.cr/ciencias-naturales-y-exactas/item/1985-introduccion-a-las-series-cronologicas.html}

\leavevmode\hypertarget{ref-oscarh-2}{}%
Hernández, O. (2011b). \emph{Introducción a las Series Cronológicas}.
Recuperado de
\url{http://www.editorial.ucr.ac.cr/ciencias-naturales-y-exactas/item/1985-introduccion-a-las-series-cronologicas.html}

\leavevmode\hypertarget{ref-oscarh-3}{}%
Hernández, O. (2011c). \emph{Introducción a las Series Cronológicas}.
Recuperado de
\url{http://www.editorial.ucr.ac.cr/ciencias-naturales-y-exactas/item/1985-introduccion-a-las-series-cronologicas.html}

\leavevmode\hypertarget{ref-oscarh-4}{}%
Hernández, O. (2011d). \emph{Introducción a las Series Cronológicas}.
Recuperado de
\url{http://www.editorial.ucr.ac.cr/ciencias-naturales-y-exactas/item/1985-introduccion-a-las-series-cronologicas.html}

\leavevmode\hypertarget{ref-hyndman2018forecasting}{}%
Hyndman, R. J., \& Athanasopoulos, G. (2018a). \emph{Forecasting:
principles and practice}. Recuperado de
\url{https://books.google.co.cr/books?id=/_bBhDwAAQBAJ}

\leavevmode\hypertarget{ref-hyndman_box-jenkins}{}%
Hyndman, R. J., \& Athanasopoulos, G. (2018b). \emph{Forecasting:
principles and practice}. Recuperado de
\url{https://books.google.co.cr/books?id=/_bBhDwAAQBAJ}

\leavevmode\hypertarget{ref-auto.arima}{}%
Hyndman, R., \& Khandakar, Y. (2008). Automatic Time Series Forecasting:
The forecast Package for R. \emph{Journal of Statistical Software,
Articles}, \emph{27}(3), 1-22.
\url{https://doi.org/10.18637/jss.v027.i03}

\leavevmode\hypertarget{ref-calidad_vitales}{}%
INEC. (2017). \emph{Población, nacimientos, defunciones y matrimonios}.
Recuperado de
\url{http://inec.cr/sites/default/files/documetos-biblioteca-virtual/repoblacev2017_0.pdf}

\leavevmode\hypertarget{ref-kedem}{}%
Kedem, B., \& Fokianos, K. (2005). \emph{Regression Models for Time
Series Analysis}. Recuperado de
\url{https://books.google.co.cr/books?id=8r0qE35wt44C}

\leavevmode\hypertarget{ref-liu}{}%
Liu, L.-M. (1989). Identification of seasonal arima models using a
filtering method. \emph{Communications in Statistics - Theory and
Methods}, \emph{18}(6), 2279-2288.
\url{https://doi.org/10.1080/03610928908830035}

\leavevmode\hypertarget{ref-melard}{}%
Mélard, G., \& Pasteels, J.-M. (2000). Automatic ARIMA modeling
including interventions, using time series expert software.
\emph{International Journal of Forecasting}, \emph{16}(4), 497-508.
\url{https://doi.org/https://doi.org/10.1016/S0169-2070(00)00067-4}

\leavevmode\hypertarget{ref-Rbase}{}%
R Core Team. (2019a). \emph{R: A Language and Environment for
Statistical Computing}. Recuperado de \url{https://www.R-project.org/}

\leavevmode\hypertarget{ref-parallel}{}%
R Core Team. (2019b). \emph{R: A Language and Environment for
Statistical Computing}. Recuperado de \url{https://www.R-project.org/}

\leavevmode\hypertarget{ref-autobox}{}%
Reilly, D. (2000). The Autobox System. \emph{International Journal of
Forecasting}, \emph{16}(4), 531-533. Recuperado de
\url{https://ideas.repec.org/a/eee/intfor/v16y2000i4p531-533.html}

\leavevmode\hypertarget{ref-supenprodc}{}%
Rosero-Bixby, L. (2018). \emph{Producto C para SUPEN. Proyección de la
mortalidad de Costa Rica 2015-2150}. Recuperado de CCP-UCR website:
\url{http://srv-website.cloudapp.net/documents/10179/999061/Nota+t\%C3\%A9cnica+tablas+de+vida+segunda+parte}

\leavevmode\hypertarget{ref-dplyr}{}%
Wickham, H., François, R., Henry, L., \& Müller, K. (2019). \emph{dplyr:
A Grammar of Data Manipulation}. Recuperado de
\url{https://CRAN.R-project.org/package=dplyr}

\leavevmode\hypertarget{ref-tidyr}{}%
Wickham, H., \& Henry, L. (2019). \emph{tidyr: Tidy Messy Data}.
Recuperado de \url{https://CRAN.R-project.org/package=tidyr}


\end{document}
