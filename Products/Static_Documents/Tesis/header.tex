

%\usepackage{fancyhdr}
%\pagestyle{fancy}
%\rhead{\includegraphics[width = 1\textwidth]{marca.jpg}}


\usepackage{geometry}
\geometry{a4paper, left=35mm, right=25mm, bottom=15mm}
\usepackage{setspace}
\doublespacing
\usepackage[spanish]{babel}
\usepackage{color}
\usepackage{xcolor}
\usepackage{framed}
\colorlet{shadecolor}{gray!20}
\setcounter{secnumdepth}{0}
\usepackage{sectsty}



\chapternumberfont{\Large}
\chaptertitlefont{\Large}
\setcounter{tocdepth}{5}
\setcounter{secnumdepth}{5}
\setlength{\footskip}{20pt}%Esto sube el número de página
\usepackage{graphics}
\usepackage{setspace} %paquete para el doble espaciado
\doublespacing %inicia el doble espaciado
 %Esto quita el punto final en la numeracion de cada seccion
\usepackage{tocloft}

\usepackage{titlesec}
\titleformat{\section}
{\Large\bfseries}{\thesection}{0.5em}{}
\titleformat{\subsection}
{\large\bfseries}{\thesubsection}{0.5em}{}
\titleformat{\subsubsection}
{\normalsize\bfseries}{\thesubsubsection}{0.5em}{}
\titleformat{\paragraph}
{\normalsize\bfseries}{\theparagraph}{0.5em}{}
\renewcommand\cftsecaftersnum{}
\renewcommand\thesection{\arabic{section}}
\renewcommand\thesubsection{\thesection.\arabic{subsection}}
\usepackage{caption}
\usepackage{fancyhdr}
\pagestyle{fancy}
\fancyhf{}
\fancyhead[R]{\thepage}
%\fancyfoot[R]{\rightmark}
%\fancyfoot[C]{Teléfono  2511-1400    /    posgrado@sep.ucr.ac.cr  /   www.sep.ucr.ac.cr}
\setlength{\headheight}{21.9pt}
\renewcommand\sectionmark[1]{%
\markright{\thesection\ #1}}
%\renewcommand{\footrulewidth}{0.4pt}


%\renewcommand{\footnoterule}{%
%  \kern -1pt
%  \hrule width \textwidth height 1pt
%  \kern 4pt
%}


%MARCA DE AGUA
%\usepackage{graphicx}
% \usepackage{fancyhdr}
%  \pagestyle{fancy}
%  \setlength\headheight{28pt}
%   \fancyhead[L]{\includegraphics[width=16cm]{marca.jpg}}
%   \fancyfoot[LE,RO]{}

\usepackage{booktabs}
\usepackage{longtable}
\usepackage{array}
\usepackage{multirow}
\usepackage{wrapfig}
%\usepackage{float}
\usepackage{colortbl}
\usepackage{pdflscape}
\usepackage{tabu}
\usepackage{threeparttable}
\usepackage{threeparttablex}
\usepackage[normalem]{ulem}
\usepackage{makecell}
\usepackage{xcolor}

\usepackage{tocloft}
\renewcommand{\cftsecleader}{\cftdotfill{\cftdotsep}}

%\renewcommand{\familydefault}{\sfdefault} %Para cambiar la fuente


%Para referenciar chunks
\usepackage{caption}
\usepackage{floatrow}
\floatsetup[figure]{capposition=top}
\floatsetup[table]{capposition=top}
\floatplacement{figure}{H}
\floatplacement{table}{H}

\DeclareNewFloatType{chunk}{placement=H, fileext=chk, name=}
\captionsetup{options=chunk}
\renewcommand{\thechunk}{Código~\arabic{chunk}}
\makeatletter
\@addtoreset{chunk}{section}
\makeatother


% Para etiquetar matrices
% Load TikZ
\usepackage{tikz}
\usetikzlibrary{matrix,decorations.pathreplacing,calc}

% Set various styles for the matrices and braces. It might pay off to fiddle around with the values a little bit
\pgfkeys{tikz/mymatrixenv/.style={decoration=brace,every left delimiter/.style={xshift=3pt},every right delimiter/.style={xshift=-3pt}}}
\pgfkeys{tikz/mymatrix/.style={matrix of math nodes,left delimiter=[,right delimiter={]},inner sep=2pt,column sep=1em,row sep=0.5em,nodes={inner sep=0pt}}}
\pgfkeys{tikz/mymatrixbrace/.style={decorate,thick}}
\newcommand\mymatrixbraceoffseth{0.5em}
\newcommand\mymatrixbraceoffsetv{0.2em}

% Now the commands to produce the braces. (I'll explain below how to use them.)
\newcommand*\mymatrixbraceright[4][m]{
    \draw[mymatrixbrace] ($(#1.north west)!(#1-#3-1.south west)!(#1.south west)-(\mymatrixbraceoffseth,0)$)
        -- node[left=2pt] {#4} 
        ($(#1.north west)!(#1-#2-1.north west)!(#1.south west)-(\mymatrixbraceoffseth,0)$);
}
\newcommand*\mymatrixbraceleft[4][m]{
    \draw[mymatrixbrace] ($(#1.north east)!(#1-#2-1.north east)!(#1.south east)+(\mymatrixbraceoffseth,0)$)
        -- node[right=2pt] {#4} 
        ($(#1.north east)!(#1-#3-1.south east)!(#1.south east)+(\mymatrixbraceoffseth,0)$);
}
\newcommand*\mymatrixbracetop[4][m]{
    \draw[mymatrixbrace] ($(#1.north west)!(#1-1-#2.north west)!(#1.north east)+(0,\mymatrixbraceoffsetv)$)
        -- node[above=2pt] {#4} 
        ($(#1.north west)!(#1-1-#3.north east)!(#1.north east)+(0,\mymatrixbraceoffsetv)$);
}
\newcommand*\mymatrixbracebottom[4][m]{
    \draw[mymatrixbrace] ($(#1.south west)!(#1-1-#3.south east)!(#1.south east)-(0,\mymatrixbraceoffsetv)$)
        -- node[below=2pt] {#4} 
        ($(#1.south west)!(#1-1-#2.south west)!(#1.south east)-(0,\mymatrixbraceoffsetv)$);
}
%%%

%%% Para hacer la subsubsubsection

\setcounter{secnumdepth}{4}

\titleformat{\paragraph}
{\normalfont\normalsize\bfseries}{\theparagraph}{1em}{}
\titlespacing*{\paragraph}
{0pt}{3.25ex plus 1ex minus .2ex}{1.5ex plus .2ex}
