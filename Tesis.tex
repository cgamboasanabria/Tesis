% Options for packages loaded elsewhere
\PassOptionsToPackage{unicode}{hyperref}
\PassOptionsToPackage{hyphens}{url}
\PassOptionsToPackage{dvipsnames,svgnames*,x11names*}{xcolor}
%
\documentclass[
]{article}
\usepackage{lmodern}
\usepackage{amssymb,amsmath}
\usepackage{ifxetex,ifluatex}
\ifnum 0\ifxetex 1\fi\ifluatex 1\fi=0 % if pdftex
  \usepackage[T1]{fontenc}
  \usepackage[utf8]{inputenc}
  \usepackage{textcomp} % provide euro and other symbols
\else % if luatex or xetex
  \usepackage{unicode-math}
  \defaultfontfeatures{Scale=MatchLowercase}
  \defaultfontfeatures[\rmfamily]{Ligatures=TeX,Scale=1}
\fi
% Use upquote if available, for straight quotes in verbatim environments
\IfFileExists{upquote.sty}{\usepackage{upquote}}{}
\IfFileExists{microtype.sty}{% use microtype if available
  \usepackage[]{microtype}
  \UseMicrotypeSet[protrusion]{basicmath} % disable protrusion for tt fonts
}{}
\makeatletter
\@ifundefined{KOMAClassName}{% if non-KOMA class
  \IfFileExists{parskip.sty}{%
    \usepackage{parskip}
  }{% else
    \setlength{\parindent}{0pt}
    \setlength{\parskip}{6pt plus 2pt minus 1pt}}
}{% if KOMA class
  \KOMAoptions{parskip=half}}
\makeatother
\usepackage{xcolor}
\IfFileExists{xurl.sty}{\usepackage{xurl}}{} % add URL line breaks if available
\IfFileExists{bookmark.sty}{\usepackage{bookmark}}{\usepackage{hyperref}}
\hypersetup{
  colorlinks=true,
  linkcolor=blue,
  filecolor=Maroon,
  citecolor=Blue,
  urlcolor=blue,
  pdfcreator={LaTeX via pandoc}}
\urlstyle{same} % disable monospaced font for URLs
\usepackage[margin=1in]{geometry}
\usepackage{color}
\usepackage{fancyvrb}
\newcommand{\VerbBar}{|}
\newcommand{\VERB}{\Verb[commandchars=\\\{\}]}
\DefineVerbatimEnvironment{Highlighting}{Verbatim}{commandchars=\\\{\}}
% Add ',fontsize=\small' for more characters per line
\usepackage{framed}
\definecolor{shadecolor}{RGB}{248,248,248}
\newenvironment{Shaded}{\begin{snugshade}}{\end{snugshade}}
\newcommand{\AlertTok}[1]{\textcolor[rgb]{0.94,0.16,0.16}{#1}}
\newcommand{\AnnotationTok}[1]{\textcolor[rgb]{0.56,0.35,0.01}{\textbf{\textit{#1}}}}
\newcommand{\AttributeTok}[1]{\textcolor[rgb]{0.77,0.63,0.00}{#1}}
\newcommand{\BaseNTok}[1]{\textcolor[rgb]{0.00,0.00,0.81}{#1}}
\newcommand{\BuiltInTok}[1]{#1}
\newcommand{\CharTok}[1]{\textcolor[rgb]{0.31,0.60,0.02}{#1}}
\newcommand{\CommentTok}[1]{\textcolor[rgb]{0.56,0.35,0.01}{\textit{#1}}}
\newcommand{\CommentVarTok}[1]{\textcolor[rgb]{0.56,0.35,0.01}{\textbf{\textit{#1}}}}
\newcommand{\ConstantTok}[1]{\textcolor[rgb]{0.00,0.00,0.00}{#1}}
\newcommand{\ControlFlowTok}[1]{\textcolor[rgb]{0.13,0.29,0.53}{\textbf{#1}}}
\newcommand{\DataTypeTok}[1]{\textcolor[rgb]{0.13,0.29,0.53}{#1}}
\newcommand{\DecValTok}[1]{\textcolor[rgb]{0.00,0.00,0.81}{#1}}
\newcommand{\DocumentationTok}[1]{\textcolor[rgb]{0.56,0.35,0.01}{\textbf{\textit{#1}}}}
\newcommand{\ErrorTok}[1]{\textcolor[rgb]{0.64,0.00,0.00}{\textbf{#1}}}
\newcommand{\ExtensionTok}[1]{#1}
\newcommand{\FloatTok}[1]{\textcolor[rgb]{0.00,0.00,0.81}{#1}}
\newcommand{\FunctionTok}[1]{\textcolor[rgb]{0.00,0.00,0.00}{#1}}
\newcommand{\ImportTok}[1]{#1}
\newcommand{\InformationTok}[1]{\textcolor[rgb]{0.56,0.35,0.01}{\textbf{\textit{#1}}}}
\newcommand{\KeywordTok}[1]{\textcolor[rgb]{0.13,0.29,0.53}{\textbf{#1}}}
\newcommand{\NormalTok}[1]{#1}
\newcommand{\OperatorTok}[1]{\textcolor[rgb]{0.81,0.36,0.00}{\textbf{#1}}}
\newcommand{\OtherTok}[1]{\textcolor[rgb]{0.56,0.35,0.01}{#1}}
\newcommand{\PreprocessorTok}[1]{\textcolor[rgb]{0.56,0.35,0.01}{\textit{#1}}}
\newcommand{\RegionMarkerTok}[1]{#1}
\newcommand{\SpecialCharTok}[1]{\textcolor[rgb]{0.00,0.00,0.00}{#1}}
\newcommand{\SpecialStringTok}[1]{\textcolor[rgb]{0.31,0.60,0.02}{#1}}
\newcommand{\StringTok}[1]{\textcolor[rgb]{0.31,0.60,0.02}{#1}}
\newcommand{\VariableTok}[1]{\textcolor[rgb]{0.00,0.00,0.00}{#1}}
\newcommand{\VerbatimStringTok}[1]{\textcolor[rgb]{0.31,0.60,0.02}{#1}}
\newcommand{\WarningTok}[1]{\textcolor[rgb]{0.56,0.35,0.01}{\textbf{\textit{#1}}}}
\usepackage{graphicx,grffile}
\makeatletter
\def\maxwidth{\ifdim\Gin@nat@width>\linewidth\linewidth\else\Gin@nat@width\fi}
\def\maxheight{\ifdim\Gin@nat@height>\textheight\textheight\else\Gin@nat@height\fi}
\makeatother
% Scale images if necessary, so that they will not overflow the page
% margins by default, and it is still possible to overwrite the defaults
% using explicit options in \includegraphics[width, height, ...]{}
\setkeys{Gin}{width=\maxwidth,height=\maxheight,keepaspectratio}
% Set default figure placement to htbp
\makeatletter
\def\fps@figure{htbp}
\makeatother
\setlength{\emergencystretch}{3em} % prevent overfull lines
\providecommand{\tightlist}{%
  \setlength{\itemsep}{0pt}\setlength{\parskip}{0pt}}
\setcounter{secnumdepth}{-\maxdimen} % remove section numbering


%\usepackage{fancyhdr}
%\pagestyle{fancy}
%\rhead{\includegraphics[width = 1\textwidth]{marca.jpg}}


\usepackage{geometry}
\geometry{a4paper, left=35mm, right=25mm, bottom=15mm}
\usepackage{setspace}
\doublespacing
\usepackage[spanish]{babel}
\usepackage{color}
\usepackage{xcolor}
\usepackage{framed}
\colorlet{shadecolor}{gray!20}
\setcounter{secnumdepth}{0}
\usepackage{sectsty}


\chapternumberfont{\Large}
\chaptertitlefont{\Large}
\setcounter{tocdepth}{5}
\setcounter{secnumdepth}{5}
\setlength{\footskip}{20pt}%Esto sube el número de página
\usepackage{graphics}
\usepackage{setspace} %paquete para el doble espaciado
\doublespacing %inicia el doble espaciado
 %Esto quita el punto final en la numeracion de cada seccion
\usepackage{tocloft}

\usepackage{titlesec}
\titleformat{\section}
{\Large\bfseries}{\thesection}{0.5em}{}
\titleformat{\subsection}
{\large\bfseries}{\thesubsection}{0.5em}{}
\titleformat{\subsubsection}
{\normalsize\bfseries}{\thesubsubsection}{0.5em}{}
\titleformat{\paragraph}
{\normalsize\bfseries}{\theparagraph}{0.5em}{}
\renewcommand\cftsecaftersnum{}
\renewcommand\thesection{\arabic{section}}
\renewcommand\thesubsection{\thesection.\arabic{subsection}}
\usepackage{caption}
\usepackage{fancyhdr}
\pagestyle{fancy}
\fancyhf{}
\fancyhead[R]{\thepage}
%\fancyfoot[R]{\rightmark}
%\fancyfoot[C]{Teléfono  2511-1400    /    posgrado@sep.ucr.ac.cr  /   www.sep.ucr.ac.cr}
\setlength{\headheight}{21.9pt}
\renewcommand\sectionmark[1]{%
\markright{\thesection\ #1}}
%\renewcommand{\footrulewidth}{0.4pt}


%\renewcommand{\footnoterule}{%
%  \kern -1pt
%  \hrule width \textwidth height 1pt
%  \kern 4pt
%}


%MARCA DE AGUA
%\usepackage{graphicx}
% \usepackage{fancyhdr}
%  \pagestyle{fancy}
%  \setlength\headheight{28pt}
%   \fancyhead[L]{\includegraphics[width=16cm]{marca.jpg}}
%   \fancyfoot[LE,RO]{}

\usepackage{booktabs}
\usepackage{longtable}
\usepackage{array}
\usepackage{multirow}
\usepackage{wrapfig}
%\usepackage{float}
\usepackage{colortbl}
\usepackage{pdflscape}
\usepackage{tabu}
\usepackage{threeparttable}
\usepackage{threeparttablex}
\usepackage[normalem]{ulem}
\usepackage{makecell}
\usepackage{xcolor}

\usepackage{tocloft}
\renewcommand{\cftsecleader}{\cftdotfill{\cftdotsep}}

%\renewcommand{\familydefault}{\sfdefault} %Para cambiar la fuente


%Para referenciar chunks
\usepackage{caption}
\usepackage{floatrow}
\floatsetup[figure]{capposition=top}
\floatsetup[table]{capposition=top}
\floatplacement{figure}{H}
\floatplacement{table}{H}

\DeclareNewFloatType{chunk}{placement=H, fileext=chk, name=}
\captionsetup{options=chunk}
\renewcommand{\thechunk}{Código~\arabic{chunk}}
\makeatletter
\@addtoreset{chunk}{section}
\makeatother
\usepackage{booktabs}
\usepackage{longtable}
\usepackage{array}
\usepackage{multirow}
\usepackage{wrapfig}
\usepackage{float}
\usepackage{colortbl}
\usepackage{pdflscape}
\usepackage{tabu}
\usepackage{threeparttable}
\usepackage{threeparttablex}
\usepackage[normalem]{ulem}
\usepackage{makecell}
\usepackage{xcolor}

\title{UNIVERSIDAD DE COSTA RICA\\
SISTEMA DE ESTUDIOS DE POSGRADO\\
~\\
~\\
~\\}
\usepackage{etoolbox}
\makeatletter
\providecommand{\subtitle}[1]{% add subtitle to \maketitle
  \apptocmd{\@title}{\par {\large #1 \par}}{}{}
}
\makeatother
\subtitle{LA SOBREPARAMETRIZACIÓN EN EL ARIMA: UNA APLICACIÓN A DATOS
COSTARRICENCES\\
~\\
~\\
~\\
~\\
Tesis sometida a la consideración de la Comisión del Programa de
Estudios de Posgrado en Estadística para optar por el grado y título de
Maestría Académica en Estadística}
\author{CÉSAR ANDRÉS GAMBOA SANABRIA B12672\\
~\\
~\\
~\\
~\\
~\\
Ciudad Universitaria Rodrigo Facio, Costa Rica\\
~\\
~\\}
\date{2020}

\begin{document}
\maketitle

\pagenumbering{gobble}
\cleardoublepage

\newpage

\addcontentsline{toc}{section}{DEDICATORIA}
\section*{DEDICATORIA}

\pagenumbering{roman}

Pendiente

\cleardoublepage

\addcontentsline{toc}{section}{AGRADECIMIENTOS}
\section*{AGRADECIMIENTOS}

También pendiente

\cleardoublepage

\begin{center}

``Esta tesis fue aceptada por la Comisión del Programa de Estudios de Posgrado en Estadística de la Universidad de Costa Rica, como requisito parcial para optar al grado y título de Maestría Académica en Estadística''

\text{}

\noindent\rule{7cm}{0.4pt}\\
Ph.D. Álvaro Morales Ramírez\\
\textbf{Decano Sistema de Estudios de Posgrado}

\text{}

\noindent\rule{7cm}{0.4pt}\\
MSc. Óscar Centeno Mora\\
\textbf{Director de Tesis}

\text{}

\noindent\rule{7cm}{0.4pt}\\
Ph.D. Gilbert Brenes Camacho\\
\textbf{Lector}

\text{}

\noindent\rule{7cm}{0.4pt}\\
Ph.D. ShuWei Chou.\\
\textbf{Lector}

\text{}

\noindent\rule{7cm}{0.4pt}\\
MSc. Johnny Madrigal Pana\\
\textbf{Director Programa de Posgrado en Estadística}

\text{}

\noindent\rule{7cm}{0.4pt}\\
César Andrés Gamboa Sanabria\\
\textbf{Candidato}

\end{center}

\cleardoublepage

\tableofcontents
\listoftables
\listoffigures

\cleardoublepage
\pagenumbering{arabic}

\newpage

\addcontentsline{toc}{section}{RESUMEN}
\section*{RESUMEN}

\cleardoublepage

\addcontentsline{toc}{section}{ABSTRACT}
\section*{ABSTRACT}

\cleardoublepage

\section{INTRODUCCIÓN}

\subsection{Antecedentes}

\subsection{El problema}

Una prueba de referencia (Rosero-Bixby,
\protect\hyperlink{ref-supenprodc}{2018}) USANDO R\footnote{Una nota al
  pie}

El cuadro \ref{tab:cuadro1} se referencia así, y la Figura
\ref{fig:figura1} así. Si se quiere referenciar un código en un chunk
(digamos en la sección de anexos), puede escribirse algo como
\ref{funcion1}

\begin{table}[!h]

\caption{\label{tab:unnamed-chunk-7}\label{tab:cuadro1}Un cuadro}
\centering
\resizebox{\linewidth}{!}{
\begin{tabular}[t]{rrrrl}
\toprule
Sepal.Length & Sepal.Width & Petal.Length & Petal.Width & Species\\
\midrule
\rowcolor{gray!6}  5.1 & 3.5 & 1.4 & 0.2 & setosa\\
4.9 & 3.0 & 1.4 & 0.2 & setosa\\
\rowcolor{gray!6}  4.7 & 3.2 & 1.3 & 0.2 & setosa\\
4.6 & 3.1 & 1.5 & 0.2 & setosa\\
\rowcolor{gray!6}  5.0 & 3.6 & 1.4 & 0.2 & setosa\\
\addlinespace
5.4 & 3.9 & 1.7 & 0.4 & setosa\\
\rowcolor{gray!6}  4.6 & 3.4 & 1.4 & 0.3 & setosa\\
5.0 & 3.4 & 1.5 & 0.2 & setosa\\
\rowcolor{gray!6}  4.4 & 2.9 & 1.4 & 0.2 & setosa\\
4.9 & 3.1 & 1.5 & 0.1 & setosa\\
\bottomrule
\multicolumn{5}{l}{\textit{Fuente:} Elaboración propia}\\
\end{tabular}}
\end{table}

\begin{figure}[!h]
\includegraphics[width=1\linewidth,height=1\textheight]{Tesis_files/figure-latex/figura1-1} \caption{Un gráfico}\label{fig:figura1}
\end{figure}

\subsection{Objetivos del estudio}

\subsection{Metodología del estudio}

\subsection{Justificación del estudio}

\subsection{Organización del estudio}

\newpage

\section{MARCO TEÓRICO}

\subsection{Introducción}

\subsection{Investigaciones relacionadas}

\subsection{Observaciones finales sobre la revisión bibliográfica}

\newpage

\section{METODOLOGÍA}

\subsection{Introducción}

\subsection{Conceptos y definiciones en el análisis de series cronológicas}

\subsubsection{Definición de una serie cronológica}

\subsubsection{Procedimiento al analizar series cronológicas}

\subsubsection{Estacionaridad}

\subsubsection{La parsimonia}

\subsection{Componentes de una serie cronológica}

\subsubsection{La tendencia}

\subsubsection{Componentes estacionales}

\subsubsection{Componente cíclico}

\subsubsection{Componente irregular}

\subsection{Supuestos en el análisis de series cronológicas}

\subsection{Modelos de series cronológicas}

\subsection{Modelos Autorregresivos Integrados de Medias Móviles}

\subsubsection{Modelos Autorregresivos}

\subsubsection{Modelos de Medias Móviles}

\subsubsection{Metodología Box-Jenkins}

\subsubsection{Etapa 1 - Identificación}

\subsubsection{Etapa 2 - Estimación y diagnóstico}

\subsubsection{Etapa 3 - Pronóstico}

\subsubsection{Notación de los modelos ARIMA}

\subsubsection{Diferenciación}

\subsection{Análisis de intervención}

\subsection{Validación cruzada}

\subsection{Medidas de rendimiento}

\subsubsection{MFE}

\subsubsection{MAE}

\subsubsection{MAPE}

\subsubsection{MPE}

\subsubsection{MSE}

\subsubsection{SSE}

\subsubsection{SMSE}

\subsubsection{RMSE}

\subsubsection{NMSE}

\subsubsection{AIC}

\subsubsection{AICc}

\subsubsection{BIC}

\subsection{La sobreparametrización}

\subsection{Simulación de series cronológicas}

\subsection{El método propuesto}

\newpage

\section{RESULTADOS}

\subsection{Introducción}

\subsection{Datos simulados}

\subsubsection{Comparación en datos simulados - Sobreparametrización vs auto.arima}

\subsection{Estimaciones en datos costarricenses}

\subsubsection{Tasa de mortalidad infantil interanual}

\subsubsection{Tasa global de fecundidad}

\subsubsection{Mortalidad por causa externa}

\subsubsection{Incentivos salariales del sector público}

\subsubsection{Intereses y comisiones del sector público}

\subsubsection{Demanda eléctrica}

\subsubsection{Comparación en datos reales - Sobreparametrización vs auto.arima}

\subsection{Discusión de los resultados}

\newpage

\section{CONCLUSIONES Y RECOMENDACIONES}

\subsection{Introducción}

\subsection{Conclusiones}

\subsection{Recomendaciones}

\newpage

\section{ANEXOS}

\subsection{La función funcion\_1}

\captionof{chunk}{Una función}\label{funcion1}

\begin{Shaded}
\begin{Highlighting}[]
\NormalTok{funcion_}\DecValTok{1}\NormalTok{ <-}\StringTok{ }\ControlFlowTok{function}\NormalTok{(x,y)\{}
\NormalTok{    x}\OperatorTok{+}\NormalTok{y}
\NormalTok{\}}
\end{Highlighting}
\end{Shaded}

\newpage

\section{REFERENCIAS}

\hypertarget{refs}{}
\leavevmode\hypertarget{ref-supenprodc}{}%
Rosero-Bixby, L. (2018). \emph{Producto C para SUPEN. Proyección de la
mortalidad de Costa Rica 2015-2150}. Recuperado de CCP-UCR website:
\url{http://srv-website.cloudapp.net/documents/10179/999061/Nota+t\%C3\%A9cnica+tablas+de+vida+segunda+parte}

\end{document}
