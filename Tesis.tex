% Options for packages loaded elsewhere
\PassOptionsToPackage{unicode}{hyperref}
\PassOptionsToPackage{hyphens}{url}
\PassOptionsToPackage{dvipsnames,svgnames*,x11names*}{xcolor}
%
\documentclass[
]{article}
\usepackage{lmodern}
\usepackage{amsmath}
\usepackage{ifxetex,ifluatex}
\ifnum 0\ifxetex 1\fi\ifluatex 1\fi=0 % if pdftex
  \usepackage[T1]{fontenc}
  \usepackage[utf8]{inputenc}
  \usepackage{textcomp} % provide euro and other symbols
  \usepackage{amssymb}
\else % if luatex or xetex
  \usepackage{unicode-math}
  \defaultfontfeatures{Scale=MatchLowercase}
  \defaultfontfeatures[\rmfamily]{Ligatures=TeX,Scale=1}
\fi
% Use upquote if available, for straight quotes in verbatim environments
\IfFileExists{upquote.sty}{\usepackage{upquote}}{}
\IfFileExists{microtype.sty}{% use microtype if available
  \usepackage[]{microtype}
  \UseMicrotypeSet[protrusion]{basicmath} % disable protrusion for tt fonts
}{}
\makeatletter
\@ifundefined{KOMAClassName}{% if non-KOMA class
  \IfFileExists{parskip.sty}{%
    \usepackage{parskip}
  }{% else
    \setlength{\parindent}{0pt}
    \setlength{\parskip}{6pt plus 2pt minus 1pt}}
}{% if KOMA class
  \KOMAoptions{parskip=half}}
\makeatother
\usepackage{xcolor}
\IfFileExists{xurl.sty}{\usepackage{xurl}}{} % add URL line breaks if available
\IfFileExists{bookmark.sty}{\usepackage{bookmark}}{\usepackage{hyperref}}
\hypersetup{
  colorlinks=true,
  linkcolor=blue,
  filecolor=Maroon,
  citecolor=Blue,
  urlcolor=blue,
  pdfcreator={LaTeX via pandoc}}
\urlstyle{same} % disable monospaced font for URLs
\usepackage[margin=1in]{geometry}
\usepackage{color}
\usepackage{fancyvrb}
\newcommand{\VerbBar}{|}
\newcommand{\VERB}{\Verb[commandchars=\\\{\}]}
\DefineVerbatimEnvironment{Highlighting}{Verbatim}{commandchars=\\\{\}}
% Add ',fontsize=\small' for more characters per line
\usepackage{framed}
\definecolor{shadecolor}{RGB}{248,248,248}
\newenvironment{Shaded}{\begin{snugshade}}{\end{snugshade}}
\newcommand{\AlertTok}[1]{\textcolor[rgb]{0.94,0.16,0.16}{#1}}
\newcommand{\AnnotationTok}[1]{\textcolor[rgb]{0.56,0.35,0.01}{\textbf{\textit{#1}}}}
\newcommand{\AttributeTok}[1]{\textcolor[rgb]{0.77,0.63,0.00}{#1}}
\newcommand{\BaseNTok}[1]{\textcolor[rgb]{0.00,0.00,0.81}{#1}}
\newcommand{\BuiltInTok}[1]{#1}
\newcommand{\CharTok}[1]{\textcolor[rgb]{0.31,0.60,0.02}{#1}}
\newcommand{\CommentTok}[1]{\textcolor[rgb]{0.56,0.35,0.01}{\textit{#1}}}
\newcommand{\CommentVarTok}[1]{\textcolor[rgb]{0.56,0.35,0.01}{\textbf{\textit{#1}}}}
\newcommand{\ConstantTok}[1]{\textcolor[rgb]{0.00,0.00,0.00}{#1}}
\newcommand{\ControlFlowTok}[1]{\textcolor[rgb]{0.13,0.29,0.53}{\textbf{#1}}}
\newcommand{\DataTypeTok}[1]{\textcolor[rgb]{0.13,0.29,0.53}{#1}}
\newcommand{\DecValTok}[1]{\textcolor[rgb]{0.00,0.00,0.81}{#1}}
\newcommand{\DocumentationTok}[1]{\textcolor[rgb]{0.56,0.35,0.01}{\textbf{\textit{#1}}}}
\newcommand{\ErrorTok}[1]{\textcolor[rgb]{0.64,0.00,0.00}{\textbf{#1}}}
\newcommand{\ExtensionTok}[1]{#1}
\newcommand{\FloatTok}[1]{\textcolor[rgb]{0.00,0.00,0.81}{#1}}
\newcommand{\FunctionTok}[1]{\textcolor[rgb]{0.00,0.00,0.00}{#1}}
\newcommand{\ImportTok}[1]{#1}
\newcommand{\InformationTok}[1]{\textcolor[rgb]{0.56,0.35,0.01}{\textbf{\textit{#1}}}}
\newcommand{\KeywordTok}[1]{\textcolor[rgb]{0.13,0.29,0.53}{\textbf{#1}}}
\newcommand{\NormalTok}[1]{#1}
\newcommand{\OperatorTok}[1]{\textcolor[rgb]{0.81,0.36,0.00}{\textbf{#1}}}
\newcommand{\OtherTok}[1]{\textcolor[rgb]{0.56,0.35,0.01}{#1}}
\newcommand{\PreprocessorTok}[1]{\textcolor[rgb]{0.56,0.35,0.01}{\textit{#1}}}
\newcommand{\RegionMarkerTok}[1]{#1}
\newcommand{\SpecialCharTok}[1]{\textcolor[rgb]{0.00,0.00,0.00}{#1}}
\newcommand{\SpecialStringTok}[1]{\textcolor[rgb]{0.31,0.60,0.02}{#1}}
\newcommand{\StringTok}[1]{\textcolor[rgb]{0.31,0.60,0.02}{#1}}
\newcommand{\VariableTok}[1]{\textcolor[rgb]{0.00,0.00,0.00}{#1}}
\newcommand{\VerbatimStringTok}[1]{\textcolor[rgb]{0.31,0.60,0.02}{#1}}
\newcommand{\WarningTok}[1]{\textcolor[rgb]{0.56,0.35,0.01}{\textbf{\textit{#1}}}}
\usepackage{graphicx}
\makeatletter
\def\maxwidth{\ifdim\Gin@nat@width>\linewidth\linewidth\else\Gin@nat@width\fi}
\def\maxheight{\ifdim\Gin@nat@height>\textheight\textheight\else\Gin@nat@height\fi}
\makeatother
% Scale images if necessary, so that they will not overflow the page
% margins by default, and it is still possible to overwrite the defaults
% using explicit options in \includegraphics[width, height, ...]{}
\setkeys{Gin}{width=\maxwidth,height=\maxheight,keepaspectratio}
% Set default figure placement to htbp
\makeatletter
\def\fps@figure{htbp}
\makeatother
\setlength{\emergencystretch}{3em} % prevent overfull lines
\providecommand{\tightlist}{%
  \setlength{\itemsep}{0pt}\setlength{\parskip}{0pt}}
\setcounter{secnumdepth}{-\maxdimen} % remove section numbering


%\usepackage{fancyhdr}
%\pagestyle{fancy}
%\rhead{\includegraphics[width = 1\textwidth]{marca.jpg}}


\usepackage{geometry}
\geometry{a4paper, left=35mm, right=25mm, bottom=15mm}
\usepackage{setspace}
\doublespacing
\usepackage[spanish]{babel}
\usepackage{color}
\usepackage{xcolor}
\usepackage{framed}
\colorlet{shadecolor}{gray!20}
\setcounter{secnumdepth}{0}
\usepackage{sectsty}



\chapternumberfont{\Large}
\chaptertitlefont{\Large}
\setcounter{tocdepth}{5}
\setcounter{secnumdepth}{5}
\setlength{\footskip}{20pt}%Esto sube el número de página
\usepackage{graphics}
\usepackage{setspace} %paquete para el doble espaciado
\doublespacing %inicia el doble espaciado
 %Esto quita el punto final en la numeracion de cada seccion
\usepackage{tocloft}

\usepackage{titlesec}
\titleformat{\section}
{\Large\bfseries}{\thesection}{0.5em}{}
\titleformat{\subsection}
{\large\bfseries}{\thesubsection}{0.5em}{}
\titleformat{\subsubsection}
{\normalsize\bfseries}{\thesubsubsection}{0.5em}{}
\titleformat{\paragraph}
{\normalsize\bfseries}{\theparagraph}{0.5em}{}
\renewcommand\cftsecaftersnum{}
\renewcommand\thesection{\arabic{section}}
\renewcommand\thesubsection{\thesection.\arabic{subsection}}
\usepackage{caption}
\usepackage{fancyhdr}
\pagestyle{fancy}
\fancyhf{}
\fancyhead[R]{\thepage}
%\fancyfoot[R]{\rightmark}
%\fancyfoot[C]{Teléfono  2511-1400    /    posgrado@sep.ucr.ac.cr  /   www.sep.ucr.ac.cr}
\setlength{\headheight}{21.9pt}
\renewcommand\sectionmark[1]{%
\markright{\thesection\ #1}}
%\renewcommand{\footrulewidth}{0.4pt}


%\renewcommand{\footnoterule}{%
%  \kern -1pt
%  \hrule width \textwidth height 1pt
%  \kern 4pt
%}


%MARCA DE AGUA
%\usepackage{graphicx}
% \usepackage{fancyhdr}
%  \pagestyle{fancy}
%  \setlength\headheight{28pt}
%   \fancyhead[L]{\includegraphics[width=16cm]{marca.jpg}}
%   \fancyfoot[LE,RO]{}

\usepackage{booktabs}
\usepackage{longtable}
\usepackage{array}
\usepackage{multirow}
\usepackage{wrapfig}
%\usepackage{float}
\usepackage{colortbl}
\usepackage{pdflscape}
\usepackage{tabu}
\usepackage{threeparttable}
\usepackage{threeparttablex}
\usepackage[normalem]{ulem}
\usepackage{makecell}
\usepackage{xcolor}

\usepackage{tocloft}
\renewcommand{\cftsecleader}{\cftdotfill{\cftdotsep}}

%\renewcommand{\familydefault}{\sfdefault} %Para cambiar la fuente


%Para referenciar chunks
\usepackage{caption}
\usepackage{floatrow}
\floatsetup[figure]{capposition=top}
\floatsetup[table]{capposition=top}
\floatplacement{figure}{H}
\floatplacement{table}{H}

\DeclareNewFloatType{chunk}{placement=H, fileext=chk, name=}
\captionsetup{options=chunk}
\renewcommand{\thechunk}{Código~\arabic{chunk}}
\makeatletter
\@addtoreset{chunk}{section}
\makeatother


% Para etiquetar matrices
% Load TikZ
\usepackage{tikz}
\usetikzlibrary{matrix,decorations.pathreplacing,calc}

% Set various styles for the matrices and braces. It might pay off to fiddle around with the values a little bit
\pgfkeys{tikz/mymatrixenv/.style={decoration=brace,every left delimiter/.style={xshift=3pt},every right delimiter/.style={xshift=-3pt}}}
\pgfkeys{tikz/mymatrix/.style={matrix of math nodes,left delimiter=[,right delimiter={]},inner sep=2pt,column sep=1em,row sep=0.5em,nodes={inner sep=0pt}}}
\pgfkeys{tikz/mymatrixbrace/.style={decorate,thick}}
\newcommand\mymatrixbraceoffseth{0.5em}
\newcommand\mymatrixbraceoffsetv{0.2em}

% Now the commands to produce the braces. (I'll explain below how to use them.)
\newcommand*\mymatrixbraceright[4][m]{
    \draw[mymatrixbrace] ($(#1.north west)!(#1-#3-1.south west)!(#1.south west)-(\mymatrixbraceoffseth,0)$)
        -- node[left=2pt] {#4} 
        ($(#1.north west)!(#1-#2-1.north west)!(#1.south west)-(\mymatrixbraceoffseth,0)$);
}
\newcommand*\mymatrixbraceleft[4][m]{
    \draw[mymatrixbrace] ($(#1.north east)!(#1-#2-1.north east)!(#1.south east)+(\mymatrixbraceoffseth,0)$)
        -- node[right=2pt] {#4} 
        ($(#1.north east)!(#1-#3-1.south east)!(#1.south east)+(\mymatrixbraceoffseth,0)$);
}
\newcommand*\mymatrixbracetop[4][m]{
    \draw[mymatrixbrace] ($(#1.north west)!(#1-1-#2.north west)!(#1.north east)+(0,\mymatrixbraceoffsetv)$)
        -- node[above=2pt] {#4} 
        ($(#1.north west)!(#1-1-#3.north east)!(#1.north east)+(0,\mymatrixbraceoffsetv)$);
}
\newcommand*\mymatrixbracebottom[4][m]{
    \draw[mymatrixbrace] ($(#1.south west)!(#1-1-#3.south east)!(#1.south east)-(0,\mymatrixbraceoffsetv)$)
        -- node[below=2pt] {#4} 
        ($(#1.south west)!(#1-1-#2.south west)!(#1.south east)-(0,\mymatrixbraceoffsetv)$);
}
%%%
\usepackage{booktabs}
\usepackage{longtable}
\usepackage{array}
\usepackage{multirow}
\usepackage{wrapfig}
\usepackage{float}
\usepackage{colortbl}
\usepackage{pdflscape}
\usepackage{tabu}
\usepackage{threeparttable}
\usepackage{threeparttablex}
\usepackage[normalem]{ulem}
\usepackage{makecell}
\usepackage{xcolor}
\ifluatex
  \usepackage{selnolig}  % disable illegal ligatures
\fi
\newlength{\cslhangindent}
\setlength{\cslhangindent}{1.5em}
\newlength{\csllabelwidth}
\setlength{\csllabelwidth}{3em}
\newenvironment{CSLReferences}[2] % #1 hanging-ident, #2 entry spacing
 {% don't indent paragraphs
  \setlength{\parindent}{0pt}
  % turn on hanging indent if param 1 is 1
  \ifodd #1 \everypar{\setlength{\hangindent}{\cslhangindent}}\ignorespaces\fi
  % set entry spacing
  \ifnum #2 > 0
  \setlength{\parskip}{#2\baselineskip}
  \fi
 }%
 {}
\usepackage{calc}
\newcommand{\CSLBlock}[1]{#1\hfill\break}
\newcommand{\CSLLeftMargin}[1]{\parbox[t]{\csllabelwidth}{#1}}
\newcommand{\CSLRightInline}[1]{\parbox[t]{\linewidth - \csllabelwidth}{#1}\break}
\newcommand{\CSLIndent}[1]{\hspace{\cslhangindent}#1}

\title{UNIVERSIDAD DE COSTA RICA\\
SISTEMA DE ESTUDIOS DE POSGRADO\\
~\\
~\\
~\\}
\usepackage{etoolbox}
\makeatletter
\providecommand{\subtitle}[1]{% add subtitle to \maketitle
  \apptocmd{\@title}{\par {\large #1 \par}}{}{}
}
\makeatother
\subtitle{LA SOBREPARAMETRIZACIÓN EN EL ARIMA: UNA APLICACIÓN A DATOS
COSTARRICENCES\\
~\\
~\\
~\\
~\\
Tesis sometida a la consideración de la Comisión del Programa de
Estudios de Posgrado en Estadística para optar por el grado y título de
Maestría Académica en Estadística}
\author{\hfill\break
\hfill\break
\hfill\break
\hfill\break
\hfill\break
CÉSAR ANDRÉS GAMBOA SANABRIA B12672\\
~\\
~\\
~\\
~\\
~\\
Ciudad Universitaria Rodrigo Facio, Costa Rica\\
~\\
~\\}
\date{2021}

\begin{document}
\maketitle

\pagenumbering{gobble}
\cleardoublepage

\newpage

\addcontentsline{toc}{section}{DEDICATORIA}
\section*{DEDICATORIA}

\pagenumbering{roman}

Pendiente

\cleardoublepage

\addcontentsline{toc}{section}{AGRADECIMIENTOS}
\section*{AGRADECIMIENTOS}

También pendiente

\cleardoublepage

\begin{center}

``Esta tesis fue aceptada por la Comisión del Programa de Estudios de Posgrado en Estadística de la Universidad de Costa Rica, como requisito parcial para optar al grado y título de Maestría Académica en Estadística''

\text{}

\noindent\rule{7cm}{0.4pt}\\
Ph.D. Álvaro Morales Ramírez\\
\textbf{Decano Sistema de Estudios de Posgrado}

\text{}

\noindent\rule{7cm}{0.4pt}\\
MSc. Óscar Centeno Mora\\
\textbf{Director de Tesis}

\text{}

\noindent\rule{7cm}{0.4pt}\\
Ph.D. Gilbert Brenes Camacho\\
\textbf{Lector}

\text{}

\noindent\rule{7cm}{0.4pt}\\
Ph.D. ShuWei Chou.\\
\textbf{Lector}

\text{}

\noindent\rule{7cm}{0.4pt}\\
MSc. Johnny Madrigal Pana\\
\textbf{Director Programa de Posgrado en Estadística}

\text{}

\noindent\rule{7cm}{0.4pt}\\
César Andrés Gamboa Sanabria\\
\textbf{Candidato}

\end{center}

\cleardoublepage

\tableofcontents
\listoftables
\listoffigures

\cleardoublepage
\pagenumbering{arabic}

\newpage

\addcontentsline{toc}{section}{RESUMEN}
\section*{RESUMEN}

\cleardoublepage

\addcontentsline{toc}{section}{ABSTRACT}
\section*{ABSTRACT}

\cleardoublepage

\section{INTRODUCCIÓN}

\subsection{Antecedentes}

Estimar los valores futuros en un determinado contexto ha producido un
aumento en el análisis de los datos referidos en el tiempo, conocido
también como series cronológicas. Este tipo de datos se encuentra en
diferentes áreas, tanto en investigación académica como en el análisis
de datos para la toma de decisiones. En el campo financiero es común
hablar de la devaluación del colón con respecto al dólar, cantidad de
exportaciones mensuales de un determinado producto o las ventas, entre
otros (\protect\hyperlink{ref-oscarh-1}{Hernández, 2011a}). Las series
cronológicas son particularmente importantes en la investigación de
mercados o en las proyecciones demográficas; de manera conjunta apoyan
la toma de decisiones para la aprobación presupuestaria en distintas
áreas.

En la actualidad, la información temporal es muy relevante: El Banco
Mundial\footnote{\url{https://databank.worldbank.org/home.aspx}} cuenta
en su sitio web con datos para el análisis de series cronológicas de
indicadores de desarrollo, capacidad estadística, indicadores
educativos, estadísticas de género, nutrición y población.
Kaggle\footnote{Se trata de una subsidiaria de la compañía Google que
  sirve de centro de reunión para todos aquellos interesados en la
  ciencia de datos.}, uno de los sitios más populares relacionados con
el análisis de información, ofrece una gran cantidad de datos temporales
para realizar competencias relacionadas con las series temporales y
determinar los modelos ganadores para una determinada
temática\footnote{Muchas de ellas incluyen recompensas económicas que
  van desde los \$500 hasta los \$100,000 para aquellos que logren
  obtener los mejor pronósticos.}.

Asimismo, los pronósticos (estimación futura de una partícula en una
serie temporal) son utilizados por instituciones públicas o del sector
privado, centros nacionales o regionales de investigación y
organizaciones no gubernamentales dedicadas al desarrollo social. Si las
entidades previamente mencionadas cuentan con proyecciones de calidad,
la puesta en marcha de sus respectivos planes tendrá un impacto más
efectivo.

Los métodos existentes para llevar a cabo un análisis de series
cronológicas son diversos, y responden al propio contexto y tipo de
datos. Obtener buenos pronósticos o explicar el comportamiento de un
fenómeno en el tiempo, siempre será un tema recurrente de investigación.
Generar una adecuada estimación es fundamental para obtener un
pronóstico de confianza, además resulta importante mencionar que los
modelos ARIMA tienen como objetivo explicar las relaciones pasadas de la
serie cronológica, para de esta manera conocer el posible comportamiento
futuro de la misma (\protect\hyperlink{ref-hyndman2018forecasting}{R. J.
Hyndman \& Athanasopoulos, 2018a}).

Al trabajar con la metodología de Box-Jenkins, uno de las etapas a
concretar es identificar los parámetros de estimación que gobiernan la
serie temporal. Para indagar los términos en el proceso de investigación
se ha utilizado la identificación de parámetros mediante
autocorrelogramas parciales y totales. Sin embargo, los
autocorrelogramas formados no analizan de forma exhaustiva y óptima los
posibles coeficientes que podrían contemplarse la ecuación de Wold.
Según su definición matemática, esta posee infinitos coeficientes, por
tanto, se debe buscar una alternativa distinta, que opte por aproximar
de una mejor manera la identificación de los parámetros estimados,
cubriendo un mayor número de posibilidades. Esto se podría obtener
mediante un método analítico de sobreparametrización.

\subsection{La problemática}

La dificultad visual a la hora de identificar un modelo ARIMA radica en
que los autocorrelogramas solo aportan una aproximación al proceso que
gobierna la serie. De forma complementaria, es común caer en el problema
de la subjetividad, pues a pesar de que alguien proponga un patrón que
gobierne la serie, otro analista podría tener una interpretación visual
diferente del mismo proceso, proponiendo así distintas identificaciones
para un mismo proceso. Además, se posee el inconveniente de que algunos
métodos de identificación automática del proceso que gobierna la serie
subestiman el número de parámetros que se debería de contemplar.

Alternativas como la función \texttt{auto.arima()}, que ofrece el
paquete \texttt{forecast} del lenguaje de programación R\footnote{Descarga
  gratuita en \url{https://cran.r-project.org/}}
(\protect\hyperlink{ref-auto.arima}{R. Hyndman \& Khandakar, 2008}),
permite estimar un modelo ARIMA basado en pruebas de raíz unitaria y
minimización del AICc (\protect\hyperlink{ref-burnham2007model}{Burnham
\& Anderson, 2007}). Así se obtiene un modelo temporal definiendo las
diferenciaciones requeridas en la parte estacional \(d\) mediante las
pruebas KPSS
(\protect\hyperlink{ref-doi:10.1111ux2f1467-9892.00213}{Xiao, 2001}) o
ADF (\protect\hyperlink{ref-fuller1995introduction}{Fuller, 1995}), y la
no estacional \(D\) utilizando las pruebas OCSB
(\protect\hyperlink{ref-Osborn2009SEASONALITYAT}{Osborn, Chui, Smith, \&
Birchenhall, 2009}) o la Canova-Hansen
(\protect\hyperlink{ref-10.2307ux2f1392184}{Canova \& Hansen, 1995}),
seleccionado el orden óptimo para los términos
\(ARIMA(p, d, q)(P, D, Q)_s\) para una serie cronológica determinada.

Sin embargo, estas pruebas suelen ignorar diversos términos que bien
podrían ofrecer mejores pronósticos; no someten a prueba las posibles
especificaciones de un modelo en un rango determinado, sino que realizan
aproximaciones analíticas para definir el proceso que gobierna la serie
cronológica, dejando así un vacío en el cual se corre el riesgo de no
seleccionar un modelo que ofrezca mejores pronósticos. Poner a prueba un
mayor número de posibilidades para la especificación de los modelos
tiene la ventaja descartar ciertos modelos, y mantener otros con un
criterio más científico y una evidencia numérica que despalde esa
decisión.

\subsection{Objetivos del estudio}

El objetivo general de la presente investigación es proponer un
algoritmo alternativo más exhaustivo para la selección de modelos ARIMA
mediante la sobreparametrización de los términos de la ecuación del
ARIMA.

Para lograr esto, se pretende:

\textbf{1.} Generar los escenarios de estimación de los distintos
modelos ARIMA mediante permutaciones de los términos \((p,d,q)\) y
\((P,D,Q)\) para la estimación de los posibles procesos que gobiernan
una determinada serie temporal.

\textbf{2.} Aplicar diversos métodos de validación en la estimación de
procesos que gobiernan la serie cronológica.

\textbf{3.} Contrastar la precisión de la estimación así como la
generación de pronósticos con otros métodos similares, aplicados en
datos costarricenses.

\textbf{4.} Integrar el desarrollo de la metodología de análisis de
series temporales en una librería del lenguaje estadístico R.

\subsection{Justificación del estudio}

El accionar de políticas gubernamentales, así como de otro tipo de
sectores, se apoyan cada vez más en un acertado análisis de la
información temporal. En demografía, uno de los principales temas de
investigación son las proyecciones de población; durante una emergencia,
conocer la posible cantidad de población que habita una zona es clave
para la rápida reacción de las autoridades en el envío de ayuda o en la
ejecución de planes de evacuación. Asimismo, los análisis actuariales se
ven beneficiados al mejorar sus métodos de pronóstico. Una de sus
principales áreas de estudio es la mortalidad, ya que representa un
insumo de vital importancia para la planificación y sostenibilidad de
los sistemas de pensiones, servicios de salud tanto pública como
privada, seguros de vida y asuntos hipotecarios
(\protect\hyperlink{ref-supenprodc}{Rosero-Bixby, 2018}).

La estimación de series de tiempo es una labor común es distintos campos
de investigación: el objetivo es poder pronosticar de forma correcta lo
que sucederá dentro de los próximos periodos. Métodos actuales como el
\texttt{auto.arima()} solamente realizan aproximaciones analíticas no
óptimas, por lo que suelen omitir procesos que describirían de una mejor
manera el comportamiento futuro de una serie cronológica.

Estimar modelos ARIMA considerando diversas permutaciones en sus
estimadores, permite mitigar las falencias de otras aproximaciones
analíticas que no analizan de forma exhaustiva todos los posibles
parámetros a estimar, o escenarios de selección de la mejor serie que
gobierne el proceso de interés. El desarrollo y evaluación del método
propuesto, la sobreparametrización, mostrará el potencial de esta
metodología en la calidad de los pronósticos. El principal aporte de
este estudio es brindar evidencia sobre cómo la sobreparametrización
puede contribuir a definir la especificación de un modelo ARIMA que
genere pronósticos más precisos.

\subsection{Organización del estudio}

El presente trabajo de investigación consta de cinco capítulos. El
primer ofreció una contextualización del uso de las series de tiempo,
así como la importancia de poder contar con pronósticos de calidad. Se
presentó el objetivo del estudio, así como una breve descripción de la
metodología empleada en la aplicación de series temporales, y cómo se
planea modificar el método de estimación en los modelos ARIMA. Se
concluye esta sección con hechos que justifican la importancia de esta
investigación.

El siguiente capítulo consiste en el marco teórico, abarcando aspectos
fundamentales de la ecuación de Wold, la metodología Box-Jenkins, la
selección de los procesos que gobiernan la serie, la descripción del
proceso iterativo, el análisis combinatorio que aborda los escenarios de
estimación, entre otros.

El tercer capítulo describe la metodología relacionada al estudio. Se
inicia con una descripción global de los conceptos más fundamentales del
análisis de series cronológicas, pasando por los componentes
fundamentales de las mismas. Se discuten también los supuestos clásicos
del análisis de series cronológicas, los distintos tipos de modelos, el
análisis de intervención, los métodos de validación y las medidas de
rendimiento; aspectos cruciales para obtener un modelo ARIMA vía
sobreparametrización. La sección metodológica culmina con la descripción
del proceso de simulación que se utilizará, así como la discusión del
método propuesto.

El capítulo cuatro consiste en la presentación de los resultados, tanto
en los datos simulados como en la aplicación a datos costarricenses y se
contrastarán contra los obtenidos por otros métodos como el de la
función \texttt{auto.arima()}, entre otros.

El último capítulo busca discutir los principales resultados, así como
señalar las conclusiones más importantes y ofrecer algunas
recomendaciones que orienten futuros estudios relacionados.

\newpage

\section{MARCO TEÓRICO}

Los modelos de series cronológicas han sido un importante tema de
investigación durante décadas (\protect\hyperlink{ref-tsa_decades}{De
Gooijer \& Hyndman, 2006}). Su objetivo principal consiste en obtener
simplificaciones de la realidad mediante el ajuste de diversos modelos,
los cuales se ajustan a datos recolectados a lo largo del tiempo de
forma regular. Estos modelos son luego utilizados para generar
pronósticos sobre el comportamiento futuro del fenómeno de interés.

Sin embargo, encontrar un modelo que presente un buen comportamiento con
respecto a los datos no es tarea fácil, pues deben considerarse diversos
aspectos teóricos para obtener un modelo adecuado que logre generar
pronósticos realistas y pertinentes para la toma de decisiones
(\protect\hyperlink{ref-tsa_decision_making}{Rezaee, Aliabadi,
Dorestani, \& Rezaee, 2020}).

Una serie temporal se define como una secuencia de datos observados,
cuyas mediciones ocurren de manera sucesiva durante un periodo de
tiempo. Los registros de estos datos pueden referirse a una única
variable en cuyo caso de dice que es una serie univariada; o bien,
pueden registrarse distintas variables para el mismo periodo de tiempo,
conocida como serie temporal multivariada. Según
\protect\hyperlink{ref-Hipel}{Hipel \& McLeod}
(\protect\hyperlink{ref-Hipel}{1994}), cada observación puede ser
continua o discreta, como la temperatura de una ciudad durante el día o
las variaciones diarias del precio de un activo financiero,
respectivamente; las observaciones continuas, además, pueden ser
convertidas a su vez en observaciones discretas.

El presente capítulo consta de seis apartados: El primer apartado abarca
los cuatro componentes de una serie cronológica, siendo estos la
tendencia y los componentes estacionales, cíclicos e irregulares.
Posteriormente, la segunda sección repasa los supuestos fundamentales en
el análisis de series cronológicas. Con los elementos más básicos
introducidos, el tercer apartado cubre el eje central de esta
investigación: Los modelos Autorregresivos Integrados de Medias Móviles
y sus componentes, los modelos autorregresivos y los modelos de medias
móviles, así como la metodología Box-Jenkins y el proceso para la
identificación de los modelos. En el cuarto apartado se introducen los
métodos para la identificación de los modelos. El quinto apartado abarca
los componentes relacionados a los autocorrelogramas, la forma más
difundida para la selección de modelos y, finalmente, el sexto apartado
introduce el principal aporte de este estudio, la sobreparametrización
como método para la selección de modelos.

\subsection{Componentes de una serie cronológica}

En el análisis de series cronológicas existen dos grandes corrientes de
estudio: Los componentes inherentes a la serie cronológica y el estudio
de las autocorrelaciones. De acuerdo con
\protect\hyperlink{ref-oscarh-1}{Hernández}
(\protect\hyperlink{ref-oscarh-1}{2011a}), las series cronológicas
poseen cuatro componentes principales: Tendencia, Ciclos, Estacionalidad
e Irregularidad. Considerando estos cuatro elementos, las series
cronológicas pueden ser \emph{aditivas}, como se muestra en la ecuación
\ref{eqn:serie_aditiva}, en cuyo caso se asume que los cuatro
componentes son independientes entre sí; o \emph{multiplicativa}, donde,
por el contrario, los cuatro componentes son independientes, como
muestra la ecuación \ref{eqn:serie_multiplicativa}.

\begin{equation}
\label{eqn:serie_aditiva}
Y(t)=T(t)+S(t)+C(t)+I(t)
\end{equation}

\begin{equation}
\label{eqn:serie_multiplicativa}
Y(t)=T(t)\times S(t)\times C(t)\times I(t)
\end{equation}

Donde \(Y\) es la serie cronológica, \(T\) es la tendencia, \(S\) es la
parte estacional, \(C\) el componente cíclico, \(I\) la parte irregular
o aleatoria, y \(t\) es el momento en el tiempo. Cada una de sus partes
se definen a continuación.

\subsubsection{La tendencia}

A partir del texto de
\protect\hyperlink{ref-calderon2012estadistica}{Calderón}
(\protect\hyperlink{ref-calderon2012estadistica}{2012}), la tendencia
general de una serie cronológica se refiere al crecimiento,
decrecimiento o lateralización de sus movimientos a lo largo del periodo
de estudio. Una tendencia bastante marcada es la del comportamiento
poblacional, que con el tiempo su crecimiento suele comportarse de una
forma muy similar a una exponencial.

\subsubsection{Componentes estacionales}

\protect\hyperlink{ref-calderon2012estadistica}{Calderón}
(\protect\hyperlink{ref-calderon2012estadistica}{2012}) también se
refiere a los cambios estacionales que se presentan en una serie de
tiempo, los cuales se relacionan con las fluctuaciones naturales del
fenómeno dentro de una temporada de observaciones. Ejemplos comunes de
esto son las condiciones climáticas, consumo de alimentos en fechas
festivas, entre otros.

\subsubsection{Componente cíclico}

Del informe elaborado también por
\protect\hyperlink{ref-calderon2012estadistica}{Calderón}
(\protect\hyperlink{ref-calderon2012estadistica}{2012}) se desprende que
los periodos cíclicos, por su parte, se refieren a los cambios que se
dan en una serie cronológica en el mediano plazo, que son causados por
determinados eventos que suelen repetirse. Estos ciclos suelen tener una
duración determinada, como es el caso de los índice bursátil S\&P 500.
Este indicador resume el estado de las 500 empresas más importantes de
Estados Unidos, y sus ciclos suelen presentar un auge, seguido por un
descenso que, posteriormente, se vuelve una depresión, y que finalmente
se convierte en una recuperación a su estado inicial.

\subsubsection{Componente irregular}

Finalmente, la irregularidad de una serie cronológica, siguiendo a
\protect\hyperlink{ref-calderon2012estadistica}{Calderón}
(\protect\hyperlink{ref-calderon2012estadistica}{2012}), se refiere a
las fluctuaciones propias de un fenómeno que no pueden ser predichas.
Estos cambios no se dan de manera regular, es decir, no siguen un patrón
determinado.

\subsection{Supuestos en el análisis de series cronológicas}

El análisis de series temporales, según
\protect\hyperlink{ref-Hipel}{Hipel \& McLeod}
(\protect\hyperlink{ref-Hipel}{1994}), representa un método para
comprender la naturaleza de la serie en cuestión y poder utilizarla para
generar pronósticos. Es en este sentido que entran en escena las
observaciones recolectadas de la serie, pues ellas son analizadas y
sujetas a modelados matemáticos que logren capturar el proceso que
gobierna a toda la serie cronológica
(\protect\hyperlink{ref-Zhang}{Zhang, 2003}). Los pronósticos se generan
a partir de este modelo, es decir, pronosticar el futuro, se utilizan
las correlaciones con las observaciones pasadas.

En un proceso determinístico, es posible predecir con certeza lo que
ocurrirá en el futuro; las series cronológicas, sin embargo, carecen de
esta condición. El análisis de series cronológicas asume que las
observaciones pueden ajustarse a un determinado modelo estadístico, esto
se conoce como un proceso estocástico. Es de esta manera que
\protect\hyperlink{ref-Hipel}{Hipel \& McLeod}
(\protect\hyperlink{ref-Hipel}{1994}) sugieren que una serie cronológica
puede considerarse como una muestra aleatoria de una serie mucho más
grande.

Como una serie de tiempo puede considerarse como un proceso estocástico,
éstas se encuentran sujetas a múltiples supuestos. El más fundamental de
ellos es que todas las observaciones son independientes e idénticamente
distribuidas (i.i.d.) siguiendo una distribución aproximadamente Normal,
con una media y variancia dadas. Lo anterior es contrario al uso de las
observaciones pasadas para pronosticar el futuro, por lo que este
supuesto, según \protect\hyperlink{ref-Cochrane}{Cochrane}
(\protect\hyperlink{ref-Cochrane}{1997}), no es exacto pues una una
serie de tiempo no es exactamente, i.i.d., sino que siguen un patrón
medianamente regular en el largo plazo.

Otro concepto de interés en las series cronológicas es el de
estacionaridad. De acuerdo con
\protect\hyperlink{ref-stationary_def}{Agrawal \& Adhikari}
(\protect\hyperlink{ref-stationary_def}{2013}), una serie se considera
estacionaria cuando su nivel medio y su variancia son aproximadamente
las mismas durante todo el periodo, es decir, el tiempo no afecta a
estos estadísticos de variabilidad. Este supuesto busca simplificar la
identificación del proceso estocástico con el objetivo de obtener un
modelo adecuado para generar los pronósticos. Sin embargo, y de una
manera similar al supuesto de i.i.d., si una serie cronológica posee
tendencias o patrones estacionales hace que esta sea no estacionaria. En
la práctica, una serie puede volverse estacionaria al aplicarle
transformaciones o diferenciaciones de distinto orden.

El último supuesto, y quizá el que más debate genera, es el criterio de
parsimonia. Como mencionan \protect\hyperlink{ref-Zhang}{Zhang}
(\protect\hyperlink{ref-Zhang}{2003}) y
\protect\hyperlink{ref-Hipel}{Hipel \& McLeod}
(\protect\hyperlink{ref-Hipel}{1994}), este principio sugiere que se
prioricen modelos sencillos, con pocos parámetros, para representar una
serie de datos. Mientras más grande y complicado sea el modelo, mayor
será el riesgo de sobre ajuste, lo que implica que el ajuste sea muy
bueno en el conjunto de datos con que se generó el modelo, pero que los
pronósticos generados sean pobres ante nuevos conjuntos de datos. Este
problema, sin embargo, se presenta al considerar un único modelo con
muchos parámetros; pero si se consideran varios modelos y estos son
sometidos a distintos criterios, puede obtenerse un modelo
sobreparametrizado que ofrezca buenos pronósticos.

\subsection{Modelos Autorregresivos Integrados de Medias Móviles}

Hay dos grandes grupos de modelos lineales de series cronológicas: Los
modelos Autorregresivos (AR) (\protect\hyperlink{ref-Lee}{Lee, s.~f.}) y
los modelos de Medias Móviles (MA)
(\protect\hyperlink{ref-box-jenkins}{Box, Jenkins, \& Reinsel, 1994}).
La combinación de estos dos grandes grupos forman los Modelos
Autorregresivos de Medias Móviles (ARMA)
(\protect\hyperlink{ref-Hipel}{Hipel \& McLeod, 1994}) y los modelos
Autorregresivos Integrados de Medias Móviles (ARIMA), siendo este último
de particular interés en esta investigación.

Los modelos ARIMA son los de uso más extendido en el análisis de series
cronológicas. Se fundamentan en las autocorrelaciones pasadas, y
contempla un proceso iterativo para identificar un posible proceso
óptimo a partir de una clase general de modelos. El teorema de Wold
(\protect\hyperlink{ref-Wold}{Surhone, Timpledon, \& Marseken, 2010})
sugiere que todo proceso estacionario puede ser determinado de una forma
específica y cuya ecuación posee, en realidad, infinitos coeficientes,
pero que debe ser reducido a una cantidad finita para luego evaluar su
ajuste sometiéndolo a diferentes pruebas y medidas de rendimiento.

\subsubsection{Ecuación de Wold}

Según \protect\hyperlink{ref-sargent_macro}{Sargent}
(\protect\hyperlink{ref-sargent_macro}{1979}), cualquier proceso
estacionario puede ser representado mediante la ecuación
\ref{eqn:proceso_estacionario}:

\begin{equation}
\label{eqn:proceso_estacionario}
x_t=\sum_{j=0}^{\infty} \psi_j\varepsilon_{t-j}+\kappa_t
\end{equation}

donde
\(\forall \psi_j \in \mathbb{R}, \psi_0=1, \sum_{j=0}^{\infty} \psi_j^2<\infty\),
y \(\varepsilon_t\) representa un ruido blanco i.i.d., es decir,
\(\varepsilon_t \sim N(0, \sigma^2)\); además, \(\kappa_t\) es el
componente lineal determinístico de forma tal que
\(cov(\kappa_t,\varepsilon_{t-j}=0)\), lo cual implica que este
componente determinístico es independiente de la suma infinita de los
choques pasados.

De lo anterior, si se omite la parte determinística \(\kappa_t\) de
\ref{eqn:proceso_estacionario}, el remanente es la suma ponderada
infinita, lo cual implica que si se conocen los ponderadores \(\psi_j\),
y si además se conoce \(\sigma_\varepsilon^2\), es posible obtener una
representación para cualquier proceso estacionario; este concepto es
conocido como \emph{media móvil infinita}.

Sabiendo que \(\varepsilon_t \sim N(0, \sigma^2)\), se tiene que
\(\varepsilon_t\) tiene media 0, es decir, está centrado en este valor.
De esta manera el ruido blanco es por definición un proceso centrado, lo
cual implica que la suma ponderada infinita está centrada en sí misma.
De esta manera, la representación de Wold de un proceso \(x_t\) supone
que se suman los choques pasados más un componente determinístico que no
es otro que el valor esperado del proceso: \(\kappa_t=m\), donde \(m\)
es una constante cualquiera. Así, la ecuación
\ref{eqn:proceso_estacionario} puede sustuirse por:

\begin{equation}
\label{eqn:proceso_estacionario2}
x_t=\sum_{j=0}^{\infty} \psi_j\varepsilon_{t-j}+m
\end{equation}

y de \ref{eqn:proceso_estacionario2} puede verificarse que,

\begin{equation}
\label{eqn:dem_proceso_estacionario2}
E(x_t)=E\left(\sum_{j=0}^{\infty} \psi_j\varepsilon_{t-j}+m\right)=\sum_{j=0}^{\infty} \psi_jE\left(\varepsilon_{t-j}\right) + m = m
\end{equation}

La principal consecuencia del teorema de Wold es que, si se conocen los
ponderadores \(\psi_j\), y además \(\sigma_\varepsilon^2\) es ruido
blanco es posible conocer el proceso por medio del cual se rige la serie
cronológica. Esto permite realizar cualquier previsión, denotada por
\(\hat X_{T+h}\) para el proceso de interés \(x_T\) en el momento
\(T+h\) para una muestra cualquiera de \(T\) observaciones de \(x_t\).
De acuerdo con \protect\hyperlink{ref-sargent_macro}{Sargent}
(\protect\hyperlink{ref-sargent_macro}{1979}), basado en el teorema de
Wold, la mejor previsión posible para un proceso \(x_t\) para el momento
\(T+h\), denotado por \(\hat x_{T+h}\), la predicción está dada por:

\begin{equation}
\label{eqn:prevision}
\hat x_{T+h}=\sum_{j=1}^{\infty} \psi_j \varepsilon_{T-j+1}
\end{equation}

De la ecuación \ref{eqn:prevision} se desprende que el error de
previsión asociado está dado por:

\begin{equation}
\label{eqn:error_prevision}
x_{T+h}- \hat x_{T+h}=\sum_{j=1}^{\infty} \psi_j \varepsilon_{T-h+1}
\end{equation}

\subsubsection{Modelos Autorregresivos}

Un modelo autorregresivo de orden \emph{p}, denotado como \(AR(p)\),
considera los valores futuros de una serie cronológica como una
combinación lineal las \(p\) observaciones predecesoras, un componente
aleatorio y un término constante. \protect\hyperlink{ref-Hipel}{Hipel \&
McLeod} (\protect\hyperlink{ref-Hipel}{1994}) y
\protect\hyperlink{ref-Lee}{Lee} (\protect\hyperlink{ref-Lee}{s.~f.})
emplean la notación de la ecuación \ref{eqn:modelo_AR}.

\begin{equation}
\label{eqn:modelo_AR}
y_t=c+\sum_{i=1}^p \varphi_iy_{t-i}+\varepsilon_t
\end{equation}

Donde \(y_t\) y \(\varepsilon_t\) corresponden al valor de la serie y al
componente aleatorio en el momento actual \(t\), mientras que
\(\varphi_i\), con \(i=1,2,\cdots,p\) son los parámetros del modelo, y
\(c\) es su término constante, que en ciertas ocasiones se suele omitir
para simplificar la notación. Los parámetros de esta clase de modelos
suelen estimarse mediante la ecuación de Yule-Walker
(\protect\hyperlink{ref-yule.walker}{Brockwell \& Davis, 2009}).

\subsubsection{Modelos de Medias Móviles}

De manera similar a como un \(AR(p)\) utiliza los valores pasados para
pronosticar los futuros, los modelos de medias móviles de orden q,
denotados como \(MA(q)\), utilizan los errores pasados de las variables
independientes. Estos modelos se describen mediante la ecuación
\ref{eqn:modelo_MA}.

\begin{equation}
\label{eqn:modelo_MA}
y_t=\mu+\sum_{j=1}^q \theta_j \varepsilon_{t-j}+\varepsilon_t
\end{equation}

Donde \(\mu\) representa el valor medio de la serie cronológica y cada
valor de \(\theta_j(j=1,2,\cdots,q)\) son los parámetros del modelo.
Como los \(MA(q)\) utilizan los errores pasados de la serie cronológica,
se asume que estos son i.i.d. centrados en cero y con una variancia
constante, siguiendo una distribución aproximadamente Normal, con lo
cual este tipo de modelos pueden considerarse como una regresión lineal
entre una observación determinada y los términos de error que le
preceden (\protect\hyperlink{ref-stationary_def}{Agrawal \& Adhikari,
2013}).

\subsubsection{Metodología Box-Jenkins}

La combinación de un \(AR(p)\) y un \(MA(q)\), descritos en las
ecuaciones \ref{eqn:modelo_AR} y \ref{eqn:modelo_MA} respectivamente,
como se mencionó al inicio de esta sección, generan los modelos
autorregresivos de medias móviles, \(ARMA(p,q)\), representados mediante
la ecuación \ref{eqn:modelo_ARMA}.

\begin{equation}
\label{eqn:modelo_ARMA}
y_t=c+\varepsilon_t+\sum_{i=1}^p \varphi_iy_{t-i}+\sum_{j=1}^q \theta_j \varepsilon_{t-j}
\end{equation}

\protect\hyperlink{ref-Cochrane}{Cochrane}
(\protect\hyperlink{ref-Cochrane}{1997}) menciona que los modelos
\(ARMA(p,q)\) suelen manipularse mediante lo que se conoce como operador
de rezagos, denotado como \(Ly_t=y_{t-1}\). Esto significa que en un
\(AR(p)\) se tiene que \(\varepsilon_t=\varphi(L)y_t\), mientras que en
\(MA(q)\) se tiene que \(y_t=\theta(L)\varepsilon_t\), y por
consiguiente en un \(ARMA(p,q)\) se tiene
\(\varphi(L)y_t=\theta(L)\varepsilon_t\). Por lo tanto, de lo anterior
se desprende que \(\varphi(L)=1-\sum_{i=1}^p \varphi_iL^i\), y que
\(\theta(L)=1+\sum_{j=1}^q\theta_jL^j\).

Los modelos \(ARMA\), sin embargo, solamente pueden ser utilizados en
series cronológicas suyo proceso es estacionario. Esto, en la práctica,
es poco común, pues una serie de tiempo a menudo posee tendencias y
ciertos patrones estacionales y, además, como menciona
\protect\hyperlink{ref-Hamzacebi}{Hamzaçebi}
(\protect\hyperlink{ref-Hamzacebi}{2008}), presentan procesos no
estacionarios por naturaleza. Esta condición hace necesaria la
introducción de una generalización de los modelos \(ARMA\), la cual se
conoce como los modelos \(ARIMA\)
(\protect\hyperlink{ref-box-jenkins}{Box, Jenkins, \& Reinsel, 1994}).

\subsubsection{Modelos ARIMA}

Partiendo de una serie con un proceso no estacionario, es posible
aplicar transformaciones o diferenciaciones (\emph{d})a los datos con el
objetivo de convertirlos en un proceso estacionario. Utilizar la
notación de rezagos descrita anteriormente, según
\protect\hyperlink{ref-Lombardo}{Flaherty \& Lombardo}
(\protect\hyperlink{ref-Lombardo}{2000}), permite plantear un modelo
\(ARIMA(p,d,q)\) como se describe en la ecuación
\ref{eqn:modelo_ARIMApdq}.

\begin{equation}
\label{eqn:modelo_ARIMApdq}
\varphi(L)(1-L)^dy_t=\theta(L)\varepsilon_t\\
\left(1-\sum_{i=1}^p \varphi_iL^i \right)(1-L)^d y_t=\left(1+\sum_{j=1}^q \theta_jL^j \right) \varepsilon_t
\end{equation}

Donde los términos \(p, d\) y \(q\) son positivos y mayores a cero y
corresponden al modelo autorregresivo, a la diferenciación y al modelo
de medias móviles, respectivamente. El componente \(d\) es el número de
diferenciaciones, si \(d=0\) se tiene un modelo ARMA, y \(d\geq1\)
representa el número de diferenciaciones; en la mayoría de casos \(d=1\)
suele ser suficiente. Así, un \(ARIMA(p,0,0)=AR(p)\),
\(ARIMA(0,0,q)=MA(q)\), y un \(ARIMA(0,1,0)=y_t=y_{t-1}+\varepsilon_t\),
es decir, un modelo de caminata aleatoria.

Como sugieren \protect\hyperlink{ref-box-jenkins}{Box, Jenkins, \&
Reinsel} (\protect\hyperlink{ref-box-jenkins}{1994}), lo anterior puede
generalizarse aún más al considerar los efectos estacionales de la serie
cronológica. Si se considera una serie cronológica con observaciones
mensuales, una diferenciación de primer orden es igual a la diferencia
entre una observación y la observación correpondiente al mismo mes pero
del año anterior; es decir, si el periodo estacional es de \(s=12\)
meses, entonces esta diferencia estacional aplicada a un
\(ARIMA(p,d,q)(P,D,Q)_S\) es calculada mediante \(z_t=y_t-y_{t-s}\).

De esta manera, el método de \protect\hyperlink{ref-box-jenkins}{Box,
Jenkins, \& Reinsel} (\protect\hyperlink{ref-box-jenkins}{1994}) inicia
con el análisis exploratorio de la serie cronológica, teniendo un
interés particular en identificar si hay presencia de factores no
estacionarios en la misma. Si en efecto se cuenta con una serie no
estacionaria, ésta debe volverse estacionaria mediante algún tipo de
transformación, típicamente el logaritmo natural. Con la serie ya
transformada, se busca identificar el proceso que gobierna la serie. La
forma clásica de hacer esto es mediante los gráficos de autocorrelación
y autocorrelación parcial. Cuando se logra identificar un proceso que se
adecue más a la serie cronológica, se deben realizar los diagnósticos
para evaluar la calidad del ajuste del modelo, así como las medidas de
rendimiento referentes a los pronósticos que genera el modelo estimado
hasta un horizonte determinado.

\subsection{Identificación del modelo}

Los métodos más clásicos para la identificación del proceso que gobierna
a una serie cronológica son las funciones de autocorrelación y
autocorrelación parcial, las cuales sirven de indicador acerca de qué
tan relacionadas están las observaciones unas de otras. Estas funciones
ofrecen indicios sobre el orden de los términos para los modelos
\(AR(p)\), \(MA(q)\) y para la diferenciación y, por ende, para la
identificación de un modelo \(ARIMA\)
(\protect\hyperlink{ref-hyndman_box-jenkins}{R. J. Hyndman \&
Athanasopoulos, 2018b}).

Para medir la relación lineal entre dos variables cuantitativas es común
utilizar el coeficiente de correlación \(r\) de Pearson
(\protect\hyperlink{ref-pearson}{Benesty \& Chen, 2009}), el cual se
define para dos variables \(X\) e \(Y\) como se muestra en la ecuación
\ref{eqn:pearson}.

\begin{equation}
\label{eqn:pearson}
r_{X,Y}=\frac{E(XY)}{\sigma_X \sigma_Y} = \frac{\sum_{i=1}^n \left(X_i- \bar X\right) \left(Y_i- \bar Y\right)}{\sqrt{\sum_{i=1}^n \left(X_i- \bar X\right)^2 \sum_{i=1}^n \left(Y_i- \bar Y\right)^2}}
\end{equation}

Este mismo concepto puede aplicarse a las series cronológicas para
comparar el valor de la misma en el tiempo \(t\), con su valor en el
tiempo \(t-1\), es decir, se comparan las observaciones consecutivas
\(Y_t\) con \(Y_{t-1}\). Esto también es aplicable a no solo una
observación rezagada \((Y_{t-1})\), sino también con múltiples rezagos
\((Y_{t-2}), (Y_{t-3}), \cdots,(Y_{t-n})\). Para esto se hace uso del
coeficiente de autocorrelación.

El coeficiente de autocorrelación (\emph{ACF} por sus siglas en inglés)
recibe su nombre debido a que se utiliza el coeficiente de correlación
para pares de observaciones \(r_{Y_t, Y_{t-1}}\) de la serie
cronológica. Al conjunto de todas las autocorrelaciones se le llama
función de autocorrelación.

La función de autocorrelación parcial\footnote{\emph{PACF} por sus
  siglas en inglés}, como menciona
\protect\hyperlink{ref-oscarh-4}{Hernández}
(\protect\hyperlink{ref-oscarh-4}{2011b}), busca medir la asociación
lineal entre las observaciones \(Y_t\) y \(Y_{t-k}\), descartando los
efectos de los rezagos \(1,2, \cdots ,k-1\).

Cuando se tiene el modelo ARIMA debidamente identificado, es importante
realizar los pronósticos. Sin embargo, estos pronósticos no son
imperativos, sino que se debe evaluar su calidad con las llamadas
medidas de rendimiento. Estas mediciones son hechas comparando el
pronóstico y su diferencia con el valor real. Existen múltiples medidas
de rendimiento, \protect\hyperlink{ref-medidas}{Adhikari, K, \& Agrawal}
(\protect\hyperlink{ref-medidas}{2013}) menciona entre ellas el
\emph{MAE}, \emph{MAPE}, \emph{RMSE}, \emph{MASE}, \emph{AIC},
\emph{AICc} y el \emph{BIC}.

\subsection{Los autocorrelogramas}

El uso del \emph{ACF} y el \emph{PACF} se suele aplicar de manera
visual. Sin embargo, hacer usos de estos elementos implica considerar
múltiples condiciones. En el caso de la identificación del orden de la
diferenciación:

\begin{itemize}
\tightlist
\item
  Si la serie posee autocorrelaciones positivas en un amplio número de
  rezagos, entonces es posible que se requiera un orden más alto en el
  valor de \(d\).
\item
  Si la autocorrelación en \(t-1\) es menor o igual a cero, o si las
  autocorrelaciones resultan ser muy bajas y sin seguir algún patrón en
  particular, entonces no se requiere un alto orden para la
  diferenciación.
\item
  Una desviación estándar baja suele ser indicador de un orden adecuado
  de integración.
\item
  Si no se utiliza ninguna diferenciación, se asume que la serie
  cronológica es estacionaria. Aplicar una diferenciación asume que la
  serie cronológica posee una media constante, mientras que dos
  diferenciaciones sugiere que la tendencia varía en el tiempo.
\end{itemize}

Para la identificación de los términos \(p\) y \(q\):

\begin{itemize}
\tightlist
\item
  Si la \emph{PACF} de la serie cronológica diferenciada muestra una
  diferencia marcada y si, además, la autocorrelación en \(t-1\) es
  positiva, entonces debe considerarse aumentar el valor de \(p\).
\item
  Si la \emph{PACF} de la serie cronológica diferenciada muestra una
  diferencia marcada y si, además, y la autocorrelación en \(t-1\) es
  negativa, entonces debe considerarse aumentar el valor de \(q\).
\item
  Los términos \(p\) y \(q\) pueden cancelar sus efectos entre sí, por
  lo que si se cuenta con un modelo \(ARMA\) más mixto que parece
  adaptarse bien a los datos, puede deberse también a que \(p\) o \(q\)
  deben ser menores.
\item
  Si la suma de los coeficientes del modelo \(AR\) es muy cercana a la
  unidad, es necesario reducir la cantidad de términos en uno y aumentar
  el orden de la diferenciación en uno.
\item
  Si la suma de los coeficientes del modelo \(MA\) es muy cercana a la
  unidad, es necesario reducir la cantidad de términos en uno y
  disminuir el orden de la diferenciación en uno.
\end{itemize}

Tener en consideración estos y otros posibles criterios para la
identificación del proceso que gobierna la serie cronológica puede
fácilmente volverse algo subjetivo, pues dos personas diferentes pueden
llegar a dar distintas interpretaciones a las visualizaciones de los
autocorrelogramas. Estas interpretaciones pueden sesgar la
identificación de los modelos y, además, no considerar otros escenarios
para los términos de un modelo \(ARIMA\); para solventar esto es
necesario considerar un abanico más amplio de opciones que a su vez
elimine el criterio subjetivo del observador, lo cual se puede lograr al
considerar múltiples permutaciones de términos, es decir, empleando la
sobreparametrización.

\subsection{La sobreparametrización y el análisis combinatorio}

La identificación visual mediante los autocorrelogramas puede llevar a
decisiones erradas acerca del proceso que gobierna la serie cronológica.
Una alternativa es considerar estimaciones procesos de ordenes bajos,
como un \(ARMA(1,1)\) y poco a poco ir incorporando términos, este
proceso de revisión permite encontrar los puntos en que agregar un
coeficiente más al modelo no aporta ninguna mejora en los resultados del
pronóstico, y así considerar únicamente aquellos modelos que tengan
coeficientes con un aporte estadísticamente significativo. Este
procedimiento es conocido como sobreparametrización. Dependiendo de la
cantidad de observaciones y del rango con que se trabajen los
coeficientes, la comparación de los modelos puede volverse muy extensa y
complicada, razón por la cual resulta imperativo generar un
procedimiento sistemático que logre seleccionar el mejor modelo con base
en sus medidas de ajuste y rendimiento del modelo.

\newpage

\section{METODOLOGÍA}

La aplicación de las series cronológicas tiene tres objetivos: 1) el
análisis exploratorio de la serie en cuestión, 2) estimar modelos de
proyección, y 3) generar pronósticos para los posibles valores futuros
que tomará la serie cronológica. Asimismo, existen múltiples formas de
proceder mediante la etapa de estimación, como lo son los métodos de
suavizamiento exponencial (\protect\hyperlink{ref-brown}{Brown, 1956}),
modelos de regresión para series temporales
(\protect\hyperlink{ref-kedem}{Kedem \& Fokianos, 2005}), redes
neuronales secuenciales aplicadas a datos longitudinales
(\protect\hyperlink{ref-redes}{Tadayon \& Iwashita, 2020}), estimaciones
bayesianas (\protect\hyperlink{ref-bayes}{Jammalamadaka, Qiu, \& Ning,
2018}), y finalmente, los procesos Autorregresivos Integrados de Medias
Móviles o ARIMA por sus siglas en inglés
(\protect\hyperlink{ref-box-jenkins}{Box, Jenkins, \& Reinsel, 1994}),
siendo estos últimos el foco de interés en este estudio.

Esta sección aborda la metodología propuesta como método de estimación y
pronóstico de series cronológicas. En la búsqueda de un modelo adecuado
de entre varios candidatos, se cubren en un primer apartado las medidas
de bondad de ajuste y de precisión a utilizar.

El segundo apartado describe en detalle el uso de la
sobreparametrización como herramienta para la generación de pronósticos
de series cronológicas con temporalidades mensuales, bimensuales,
trimestrales, cuatrimestrales o anuales mediante un proceso de selección
fundamentada en las permutaciones de todos los parámetros de un modelo
ARIMA hasta un rango determinado. Las medidas de precisión y de bondad
de ajuste sirven de insumo para utilizar un método de consenso entre
ellas y seleccionar el modelo más adecuado mediante la
sobreparametrización: se comparan todos los posibles en un intervalo
específico de términos definiendo una diferenciación adecuada para la
serie y permutando hasta un máximo definido para los términos
autorregresivos y de medias móviles especificados para así seleccionar
la especificación que ofrezca mejores resultados al momento de
pronosticar valores futuros de la serie cronológica.

\subsection{Medidas de bondad de ajuste y rendimiento}

El objetivo último al estimar un modelo ARIMA es obtener los pronósticos
de dicho modelo Sin embargo, estos pronósticos no son pueden asumirse
como correctos, sino que se debe evaluar su calidad con las llamadas
medidas de bondad de ajuste y de rendimiento. Existen múltiples medidas,
\protect\hyperlink{ref-medidas}{Adhikari, K, \& Agrawal}
(\protect\hyperlink{ref-medidas}{2013}) menciona, entre otras, las
siguientes:

\subsubsection{AIC}

Se calcula de la siguiente manera:

\begin{equation}
\label{eqn:AIC}
AIC=-2logL\left(\hat\theta\right)+2k
\end{equation}

Donde \(k\) es el número de parámetros y \(n\) el número de datos.

\subsubsection{AICc}

Su forma de cálculo se muestra en la ecuación \ref{eqn:AICc}
\begin{equation}
\label{eqn:AICc}
AICc=-2logL\left(\hat\theta\right)+2k+\frac{2k+1}{n-k-1}
\end{equation}

Donde \(k\) es el número de parámetros y \(n\) el número de datos.

\subsubsection{BIC}

El último estadístico de bondad de ajuste se calcula como se muestran en
la ecuación \ref{eqn:BIC}.

\begin{equation}
\label{eqn:BIC}
BIC=-2logL\left(\hat\theta\right)+k\cdot log(n)
\end{equation}

Donde \(k\) es el número de parámetros y \(n\) el número de datos.

\subsubsection{MAE}

El error absoluto medio se define mediante la ecuación \ref{eqn:MAE}

\begin{equation}
\label{eqn:MAE}
\frac{1}{n}\sum_{t=1}^n |e_t|
\end{equation}

\subsubsection{MASE}

Esta medida de rendimiento tiene dos casos, uno para series cronológicas
no estacionales y otro para series cronológicas estacionales, como se
muestra en las ecuaciones \ref{eqn:MASE_no} y \ref{eqn:MASE_si}.

\begin{equation}
\label{eqn:MASE_no}
\frac{\frac{1}{J}\sum_j|e_j|}{\frac{1}{T-1}\sum_{t=2}^T|Y_t-Y_{t-1}|}
\end{equation}

\begin{equation}
\label{eqn:MASE_si}
\frac{\frac{1}{J}\sum_j|e_j|}{\frac{1}{T-m}\sum_{t=m+1}^T|Y_t-Y_{t-m}|}
\end{equation}

Donde \(m\) es la temporalidad de la serie.

\subsubsection{RMSE}

Es la raíz del error cuadrático medio, como se define en la ecuación
\ref{eqn:RMSE}.

\begin{equation}
\label{eqn:RMSE}
\sqrt{\frac{1}{n}\sum_{t=1}^n |e_t^2|}
\end{equation}

\subsection{La sobreparametrización}

La estimación de los modelos y posterior selección de los mismos vía
sobreparametrización es un proceso que requiere de distintas etapas. El
procedimiento completo fue programado utilizando el lenguaje
R\footnote{\url{https://cran.r-project.org/}} y su código se muestra en
el \ref{funcion_op_arima}, la cuál fue construida haciendo uso de los
paquetes de R \texttt{tidyr} (\protect\hyperlink{ref-tidyr}{Wickham \&
Henry, 2019}), \texttt{dplyr}(\protect\hyperlink{ref-dplyr}{Wickham,
François, Henry, \& Müller, 2019}) y
\texttt{parallel}(\protect\hyperlink{ref-parallel}{R Core Team, 2019}),
los procesos internos de esta función son descritos a continuación.

A partir de una serie cronológica \(y_t\), se realiza una partición de
los datos para tener dos conjuntos distintos. Uno de ellos servirá para
entrenar y estimar los distintos modelos; mientras que el segundo
servirá para validar los pronósticos y posteriormente seleccionar el
modelo más adecuado. De manera predeterminada, se utiliza una partición
del 80\% de los datos para el conjunto de entrenamiento y un 20\% para
los datos de validación, sin embargo, esto puede cambiar de acuerdo al
interés propio del investigador(a).

Una vez que se define la partición que tendrá la serie cronológica, se
prosigue con la selección de los escenarios para estimar los modelos de
ARIMA. Es en esta instancia en donde se decide en valor máximo de los
parámetros \(p,d,q,P,D,Q\) del modelo \(ARIMA(p,d,q)(P,D,Q)_s\) que
serán sujetos al análisis. Si se cuenta con una serie sin patrones
estacionales y cuyo modelo con mayor cantidad de parámetros es un
\(ARIMA(3,1,4)\), la matriz de valores paramétricos es la que se muestra
en \ref{eqn:matriz_arima_1}:

\begin{equation}
\label{eqn:matriz_arima_1}
\begin{tikzpicture}[mymatrixenv]
    \matrix[mymatrix] (m)  {
        0 & 0 & 1 & 0 & 0 & 0 \\
        0 & 0 & 2 & 0 & 0 & 0 \\
        0 & 0 & 3 & 0 & 0 & 0 \\
        0 & 0 & 4 & 0 & 0 & 0 \\
        0 & 1 & 0 & 0 & 0 & 0 \\
        0 & 1 & 1 & 0 & 0 & 0 \\
        0 & 1 & 2 & 0 & 0 & 0 \\
        \vdots & \vdots & \vdots & \vdots & \vdots & \vdots \\
        3 & 1 & 4 & 0 & 0 & 0 \\
    };
    \mymatrixbracetop{1}{3}{$p, d, q$};
    \mymatrixbracetop{4}{6}{$P,D,Q$}
\end{tikzpicture}
\end{equation}

De manera análoga, al trabajar con un modelo con algún efecto estacional
en una determinada periodicidad, como por ejemplo mensual, la matriz de
valores paramétricos al definir el modelo con mayor número de parámetros
como un \(ARIMA(6,1,6)(6,1,6)_{12}\) es la mostrada en
\ref{eqn:matriz_arima_2}:

\begin{equation}
\label{eqn:matriz_arima_2}
\begin{tikzpicture}[mymatrixenv]
    \matrix[mymatrix] (m)  {
        0 & 0 & 1 & 0 & 0 & 1 \\
        0 & 0 & 1 & 0 & 0 & 2 \\
        0 & 0 & 1 & 0 & 0 & 3 \\
        0 & 0 & 1 & 0 & 0 & 4 \\
        0 & 1 & 1 & 0 & 0 & 5 \\
        0 & 1 & 1 & 0 & 0 & 6 \\
        0 & 1 & 1 & 0 & 1 & 0 \\
        \vdots & \vdots & \vdots & \vdots & \vdots & \vdots \\
        6 & 1 & 6 & 6 & 1 & 6 \\
    };
    \mymatrixbracetop{1}{3}{$p, d, q$};
    \mymatrixbracetop{4}{6}{$P,D,Q$}
\end{tikzpicture}
\end{equation}

Con la matriz de valores paramétricos, como las mostradas en
\ref{eqn:matriz_arima_1} y \ref{eqn:matriz_arima_2}, se estiman los
modelos en orden descendente, del modelo con menos parámetros al que
tiene más parámetros. Al estimar un nuevo modelo, se evalúa mediante una
prueba t (\protect\hyperlink{ref-astsa}{Stoffer, 2020}) para verificar
el nuevo término incorporado al modelo es estadísticamente distinto de
cero, es decir, el nuevo parámetro está generando un impacto en el
modelo. AL tratarse de un proceso iterativo, los cálculos son pueden
volverse computacionalmente pesado, es por esta razón que la
programación del proceso fue habilitada para realizar procesamiento
paralelo y de esta manera reducir el consumo de tiempo en la obtención
de resultados.

Cuando se han realizado las pruebas de significancia estadística a los
modelos, son calculadas las medidas de bondad de ajuste y de rendimiento
mencionadas con anterioridad. Tras esto, se aplica un método de concenso
para seleccionar el modelo más adecuado. Este criterio consiste en darle
mayor o menor ponderación a los resultados obtenidos con el conjunto de
datos de entrenamiento y el de validación; de forma predeterminada se le
da una ponderación de 0.8 a los resultados de validación y un 0.2 a los
de entrenamiento, esto porque en la práctica, los datos de validación
son considerados como datos más recientes y que, mientras más cercanos
sean los pronósticos a estos datos, mejores resultados ofrece el modelo.
El método de concenso es utilizado para obtener un puntaje de cada
modelo ARIMA, su cálculo se obtiene de la ecuación \ref{eqn:concenso}:

\begin{equation}
\label{eqn:concenso}
min\left( \sum_i {m_i}\cdot w_j \right)
\end{equation}

Donde \(m_i\) representa cada una de las medidas de rendimiento y
\(w_j\) es el valor de ponderación de los conjunto de entrenamiento y
validación mencionados anteriormente. El valor más bajo de todos los
modelos es el que se define como el modelo más adecuado.

Como parte de esta investigación, es necesario validar la estimación de
modelos ARIMA mediante sobreparametrización no solo con datos reales,
sino también mediante datos simulados. Para ello es necesario obtener
series cronológicas que son gobernadas por un proceso determinado.

Con este fin, se programó un función haciendo uso del lenguaje R que
toma una serie de valores, los cuales pueden ser reales o simulados.
Además, se especifican los valores \(p,d,q,P,D,Q\) del modelo ARIMA a
partir del cual se desea obtener los datos, así como los valores de los
coeficientes presentes en el modelo. Es partiendo de este modelo que se
simulan los datos de las series cronológicas que serán insumo para la
prueba la selección mediante sobreparametrización, este procedimiento se
encuentra en el \ref{simula_series}.

\newpage

\section{RESULTADOS}

\subsection{Introducción}

El método propuesto se probará comparándose con los resultados de seis
series con distintas temporalidades: mortalidad infantil, mortalidad por
causa externa, nacimientos, demanda eléctrica, intereses y comisiones
del sector público e incentivos salariales del sector público.

\subsection{Datos simulados}

\subsubsection{Comparación en datos simulados - Sobreparametrización vs auto.arima}

\subsection{Estimaciones en datos costarricenses}

En el campo demográfico, por ejemplo, las estadísticas vitales son
sistematizadas y divulgadas año tras año, por tanto, revelan los cambios
acontecidos durante este periodo. Esta información junto con la
proveniente de los censos de población constituye la base para construir
los diferentes índices, tasas y otros indicadores que revelan la
situación demográfica del país, información de gran relevancia para la
planificación nacional, regional y local en diversos campos. Uno de
estos principales campos de acción es la salud pública, para la cual la
tasa de mortalidad infantil se considera uno de los indicadores
prioritarios dado que refleja no solo las condiciones de salud de la
población infante, sino también los niveles de desarrollo del país, pues
depende de la calidad de la atención de la salud, principalmente de la
prenatal y perinatal, así como de las condiciones de saneamiento. Por
tanto, su continuo monitoreo es fundamental para diseñar, implementar y
evaluar políticas de salud pública orientadas a disminuir y erradicar
aquellas que son prevenibles
(\protect\hyperlink{ref-calidad_vitales}{INEC, 2017}).

\subsubsection{Tasa de mortalidad infantil interanual}

\subsubsection{Tasa global de fecundidad}

\subsubsection{Mortalidad por causa externa}

\subsubsection{Incentivos salariales del sector público}

\subsubsection{Intereses y comisiones del sector público}

\subsubsection{Demanda eléctrica}

\subsubsection{Comparación en datos reales - Sobreparametrización vs auto.arima}

\subsection{Discusión de los resultados}

\newpage

\section{CONCLUSIONES Y RECOMENDACIONES}

\subsection{Introducción}

\subsection{Conclusiones}

\subsection{Recomendaciones}

\newpage

\section{ANEXOS}

\subsection{Función de sobreparametrización}

\captionof{chunk}{Función op.arima}\label{funcion_op_arima}

\begin{Shaded}
\begin{Highlighting}[]
\NormalTok{op.arima }\OtherTok{\textless{}{-}} \ControlFlowTok{function}\NormalTok{(}\AttributeTok{arima\_process =} \FunctionTok{c}\NormalTok{(}\AttributeTok{p =} \DecValTok{1}\NormalTok{, }\AttributeTok{d =} \DecValTok{1}\NormalTok{, }\AttributeTok{q =} \DecValTok{1}\NormalTok{, }
                                       \AttributeTok{P =} \DecValTok{1}\NormalTok{, }\AttributeTok{D =} \DecValTok{1}\NormalTok{, }\AttributeTok{Q =} \DecValTok{1}\NormalTok{), }
\NormalTok{                     seasonal\_periodicity,}
\NormalTok{                     time\_serie, }\AttributeTok{reg =} \ConstantTok{NULL}\NormalTok{, }\AttributeTok{horiz =} \DecValTok{12}\NormalTok{,}
                     \AttributeTok{prop=}\NormalTok{.}\DecValTok{8}\NormalTok{, }\AttributeTok{training\_weight=}\NormalTok{.}\DecValTok{2}\NormalTok{, }\AttributeTok{testing\_weight=}\NormalTok{.}\DecValTok{8}\NormalTok{,}
                     \AttributeTok{parallelize=}\ConstantTok{FALSE}\NormalTok{, }\AttributeTok{clusters=}\FunctionTok{detectCores}\NormalTok{(}\AttributeTok{logical =} \ConstantTok{FALSE}\NormalTok{))\{}
    
\NormalTok{    data\_partition }\OtherTok{\textless{}{-}} \FunctionTok{round}\NormalTok{(}\FunctionTok{length}\NormalTok{(time\_serie)}\SpecialCharTok{*}\NormalTok{prop, }\DecValTok{0}\NormalTok{)}
\NormalTok{    train }\OtherTok{\textless{}\textless{}{-}} \FunctionTok{subset}\NormalTok{(time\_serie, }\AttributeTok{end=}\NormalTok{data\_partition)}
\NormalTok{    test }\OtherTok{\textless{}\textless{}{-}} \FunctionTok{subset}\NormalTok{(time\_serie, }\AttributeTok{start=}\NormalTok{data\_partition}\SpecialCharTok{+}\DecValTok{1}\NormalTok{)}
    
\NormalTok{    arima\_model }\OtherTok{\textless{}{-}} \ControlFlowTok{function}\NormalTok{(time\_serie, non\_seasonal, seasonal, }
\NormalTok{                            periodic, }\AttributeTok{regr =} \ConstantTok{NULL}\NormalTok{)\{}
        
\NormalTok{        seasonal\_part }\OtherTok{\textless{}{-}} \FunctionTok{list}\NormalTok{(}\AttributeTok{order=}\NormalTok{seasonal, }\AttributeTok{period=}\NormalTok{periodic)}
        \ControlFlowTok{if}\NormalTok{(}\FunctionTok{is.null}\NormalTok{(regr))\{}
\NormalTok{            arima\_model }\OtherTok{\textless{}{-}} \FunctionTok{tryCatch}\NormalTok{(\{}
                \FunctionTok{Arima}\NormalTok{(time\_serie, }
                      \AttributeTok{order =}\NormalTok{ non\_seasonal, }
                      \AttributeTok{seasonal =}\NormalTok{ seasonal\_part)}
\NormalTok{            \}, }
            \AttributeTok{error =} \ControlFlowTok{function}\NormalTok{(e) }\ConstantTok{NULL}\NormalTok{)  }
\NormalTok{        \}}
        
        \ControlFlowTok{if}\NormalTok{(}\SpecialCharTok{!}\FunctionTok{is.null}\NormalTok{(regr))\{}
\NormalTok{            arima\_model }\OtherTok{\textless{}{-}}\FunctionTok{tryCatch}\NormalTok{(\{}
                \FunctionTok{Arima}\NormalTok{(time\_serie, }
                      \AttributeTok{order =}\NormalTok{ non\_seasonal, }
                      \AttributeTok{seasonal =}\NormalTok{ seasonal\_part, }
                      \AttributeTok{xreg =}\NormalTok{ regr)}
\NormalTok{            \}, }
            \AttributeTok{error =} \ControlFlowTok{function}\NormalTok{(e) }\ConstantTok{NULL}\NormalTok{)   }
\NormalTok{        \}}
        
        \ControlFlowTok{if}\NormalTok{(}\SpecialCharTok{!}\FunctionTok{is.null}\NormalTok{(arima\_model))\{}
\NormalTok{            degrees\_of\_freedom }\OtherTok{\textless{}{-}}\NormalTok{ arima\_model}\SpecialCharTok{$}\NormalTok{nobs }\SpecialCharTok{{-}} \FunctionTok{length}\NormalTok{(arima\_model}\SpecialCharTok{$}\NormalTok{coef)}
\NormalTok{            t\_value }\OtherTok{\textless{}{-}}\NormalTok{ arima\_model}\SpecialCharTok{$}\NormalTok{coef}\SpecialCharTok{/}\FunctionTok{sqrt}\NormalTok{(}\FunctionTok{diag}\NormalTok{(arima\_model}\SpecialCharTok{$}\NormalTok{var.coef))}
\NormalTok{            prob }\OtherTok{\textless{}{-}}\NormalTok{ stats}\SpecialCharTok{::}\FunctionTok{pf}\NormalTok{(t\_value}\SpecialCharTok{\^{}}\DecValTok{2}\NormalTok{, }\AttributeTok{df1 =} \DecValTok{1}\NormalTok{, }\AttributeTok{df2 =}\NormalTok{ degrees\_of\_freedom, }
                              \AttributeTok{lower.tail =} \ConstantTok{FALSE}\NormalTok{)}
            \FunctionTok{ifelse}\NormalTok{(}\FunctionTok{sum}\NormalTok{(}\DecValTok{1}\SpecialCharTok{*}\NormalTok{prob}\SpecialCharTok{\textgreater{}}\FloatTok{0.05}\NormalTok{)}\SpecialCharTok{\textless{}}\DecValTok{1}\NormalTok{, }\FunctionTok{return}\NormalTok{(arima\_model), }\DecValTok{1}\NormalTok{)}
\NormalTok{        \}}
        
\NormalTok{    \}}

\NormalTok{    arima\_measures }\OtherTok{\textless{}{-}} \ControlFlowTok{function}\NormalTok{(arima\_model, testing, horizon, }\AttributeTok{regr =} \ConstantTok{NULL}\NormalTok{)\{}
        
\NormalTok{        model\_spec }\OtherTok{\textless{}{-}} \FunctionTok{capture.output}\NormalTok{(arima\_model)}
\NormalTok{        model\_spec }\OtherTok{\textless{}{-}} \FunctionTok{substr}\NormalTok{(model\_spec[}\DecValTok{2}\NormalTok{],}\DecValTok{1}\NormalTok{, }\DecValTok{23}\NormalTok{)}
        
\NormalTok{        data }\OtherTok{\textless{}{-}} \FunctionTok{capture.output}\NormalTok{(}\FunctionTok{summary}\NormalTok{(arima\_model))}
\NormalTok{        data }\OtherTok{\textless{}{-}}\NormalTok{ data[}\FunctionTok{grepl}\NormalTok{(}\StringTok{"AIC"}\NormalTok{, data) }\SpecialCharTok{==}\NormalTok{ T]}
\NormalTok{        model\_info }\OtherTok{\textless{}{-}} \FunctionTok{strsplit}\NormalTok{(data, }\StringTok{" "}\NormalTok{)}
\NormalTok{        pos }\OtherTok{\textless{}{-}} \FunctionTok{which}\NormalTok{(}\FunctionTok{sapply}\NormalTok{(model\_info, nchar)}\SpecialCharTok{\textgreater{}}\DecValTok{0}\NormalTok{)}
\NormalTok{        model\_info }\OtherTok{\textless{}{-}}\NormalTok{ model\_info[[}\DecValTok{1}\NormalTok{]][pos]}
\NormalTok{        model\_info }\OtherTok{\textless{}{-}} \FunctionTok{do.call}\NormalTok{(}\StringTok{"rbind"}\NormalTok{, }\FunctionTok{strsplit}\NormalTok{(model\_info, }\StringTok{"="}\NormalTok{)) }\SpecialCharTok{\%\textgreater{}\%} 
            \FunctionTok{data.frame}\NormalTok{() }
        \FunctionTok{colnames}\NormalTok{(model\_info) }\OtherTok{\textless{}{-}} \FunctionTok{c}\NormalTok{(}\StringTok{"Medida"}\NormalTok{, }\StringTok{"Valor"}\NormalTok{)}
\NormalTok{        model\_info }\OtherTok{\textless{}{-}}\NormalTok{ model\_info }\SpecialCharTok{\%\textgreater{}\%} 
            \FunctionTok{mutate}\NormalTok{(}\AttributeTok{Valor =} \FunctionTok{as.numeric}\NormalTok{(}\FunctionTok{as.character}\NormalTok{(Valor))) }\SpecialCharTok{\%\textgreater{}\%} 
            \FunctionTok{spread}\NormalTok{(Medida, Valor) }\SpecialCharTok{\%\textgreater{}\%} 
            \FunctionTok{data.frame}\NormalTok{(}\AttributeTok{arima\_model =}\NormalTok{ model\_spec)}
        
\NormalTok{        model\_performance }\OtherTok{\textless{}{-}} \FunctionTok{data.frame}\NormalTok{(}\AttributeTok{arima\_model =} \FunctionTok{c}\NormalTok{(model\_spec, }
                                                        \FunctionTok{paste}\NormalTok{(model\_spec, }
                                                              \StringTok{"Validacion"}\NormalTok{)), }
                                        \FunctionTok{accuracy}\NormalTok{(}\FunctionTok{forecast}\NormalTok{(arima\_model, }
\NormalTok{                                                          horizon, }
                                                          \AttributeTok{xreg =}\NormalTok{ regr), }
\NormalTok{                                                 testing))}
        \FunctionTok{merge}\NormalTok{(model\_info, model\_performance, }\AttributeTok{by=}\StringTok{"arima\_model"}\NormalTok{, }\AttributeTok{all =} \ConstantTok{TRUE}\NormalTok{) }\SpecialCharTok{\%\textgreater{}\%} 
            \FunctionTok{select}\NormalTok{(arima\_model, AIC, AICc, BIC, MAE, RMSE, MASE)}
\NormalTok{    \}}
    
\NormalTok{    arima\_selected }\OtherTok{\textless{}{-}} \ControlFlowTok{function}\NormalTok{(model\_table, }
                               \AttributeTok{Wtrain=}\NormalTok{training\_weight, }
                               \AttributeTok{Wtest=}\NormalTok{testing\_weight)\{}
        

\NormalTok{        model\_table }\OtherTok{\textless{}{-}}\NormalTok{ model\_table }\SpecialCharTok{\%\textgreater{}\%}
            \FunctionTok{distinct}\NormalTok{(arima\_model, }\AttributeTok{.keep\_all =} \ConstantTok{TRUE}\NormalTok{)}
        

\NormalTok{        model\_table }\OtherTok{\textless{}{-}}\NormalTok{ model\_table }\SpecialCharTok{\%\textgreater{}\%} 
            \FunctionTok{mutate}\NormalTok{(}\AttributeTok{mod =} \FunctionTok{as.character}\NormalTok{(}\FunctionTok{c}\NormalTok{(}\DecValTok{0}\NormalTok{, }\FunctionTok{rep}\NormalTok{(}\DecValTok{1}\SpecialCharTok{:}\NormalTok{(}\FunctionTok{nrow}\NormalTok{(model\_table)}\SpecialCharTok{{-}}\DecValTok{1}\NormalTok{)}\SpecialCharTok{\%/\%}\DecValTok{2}\NormalTok{))))}
        

\NormalTok{        tabla2 }\OtherTok{\textless{}{-}}\NormalTok{ model\_table }\SpecialCharTok{\%\textgreater{}\%} 
            \FunctionTok{mutate\_at}\NormalTok{(}\FunctionTok{vars}\NormalTok{(}\FunctionTok{contains}\NormalTok{(}\StringTok{"C"}\NormalTok{)), }\ControlFlowTok{function}\NormalTok{(x)\{x}\SpecialCharTok{{-}}\FunctionTok{min}\NormalTok{(x, }\AttributeTok{na.rm=}\ConstantTok{TRUE}\NormalTok{)\}) }\SpecialCharTok{\%\textgreater{}\%} 
            \FunctionTok{mutate\_if}\NormalTok{(is.numeric, }\ControlFlowTok{function}\NormalTok{(x) }\FunctionTok{ifelse}\NormalTok{(}\FunctionTok{is.na}\NormalTok{(x),}\DecValTok{0}\NormalTok{,x)) }\SpecialCharTok{\%\textgreater{}\%} 
            \FunctionTok{mutate}\NormalTok{(}\AttributeTok{puntaje =}\NormalTok{ AIC}\SpecialCharTok{+}\NormalTok{AICc}\SpecialCharTok{+}\NormalTok{BIC}\SpecialCharTok{+}\NormalTok{MAE}\SpecialCharTok{+}\NormalTok{RMSE}\SpecialCharTok{+}\NormalTok{MASE,}
                   \AttributeTok{ponde =} \FunctionTok{ifelse}\NormalTok{(}\FunctionTok{grepl}\NormalTok{(}\StringTok{"Validacion"}\NormalTok{, }
\NormalTok{                                        arima\_model)}\SpecialCharTok{==}\ConstantTok{TRUE}\NormalTok{,}
\NormalTok{                                  Wtest, }
\NormalTok{                                  Wtrain),}
                   \AttributeTok{puntaje =}\NormalTok{ puntaje}\SpecialCharTok{*}\NormalTok{ponde)}
        

        \FunctionTok{suppressMessages}\NormalTok{(\{}
\NormalTok{            minimal\_score }\OtherTok{\textless{}{-}}\NormalTok{ tabla2 }\SpecialCharTok{\%\textgreater{}\%} 
                \FunctionTok{group\_by}\NormalTok{(mod) }\SpecialCharTok{\%\textgreater{}\%} 
                \FunctionTok{summarise}\NormalTok{(}\AttributeTok{puntaje=}\FunctionTok{sum}\NormalTok{(puntaje)) }\SpecialCharTok{\%\textgreater{}\%} 
\NormalTok{                ungroup}
\NormalTok{        \})}

\NormalTok{        pos }\OtherTok{\textless{}{-}}\NormalTok{ minimal\_score}\SpecialCharTok{$}\NormalTok{mod[}\FunctionTok{which}\NormalTok{(minimal\_score}\SpecialCharTok{$}\NormalTok{puntaje}\SpecialCharTok{==}\FunctionTok{min}\NormalTok{(}
\NormalTok{            minimal\_score}\SpecialCharTok{$}\NormalTok{puntaje))]}

\NormalTok{        model\_table }\SpecialCharTok{\%\textgreater{}\%} 
            \FunctionTok{filter}\NormalTok{(mod }\SpecialCharTok{\%in\%}\NormalTok{ pos) }\SpecialCharTok{\%\textgreater{}\%} 
\NormalTok{            dplyr}\SpecialCharTok{::}\FunctionTok{select}\NormalTok{(arima\_model}\SpecialCharTok{:}\NormalTok{MASE)}
\NormalTok{    \}}
    
    \FunctionTok{suppressWarnings}\NormalTok{(\{     }

\NormalTok{        valores }\OtherTok{\textless{}{-}} \FunctionTok{expand.grid}\NormalTok{(}\AttributeTok{p =} \DecValTok{0}\SpecialCharTok{:}\NormalTok{arima\_process[}\DecValTok{1}\NormalTok{], }
                               \AttributeTok{d =} \DecValTok{0}\SpecialCharTok{:}\NormalTok{arima\_process[}\DecValTok{2}\NormalTok{], }
                               \AttributeTok{q =} \DecValTok{0}\SpecialCharTok{:}\NormalTok{arima\_process[}\DecValTok{3}\NormalTok{], }
                               \AttributeTok{P =} \DecValTok{0}\SpecialCharTok{:}\NormalTok{arima\_process[}\DecValTok{4}\NormalTok{], }
                               \AttributeTok{D =} \DecValTok{0}\SpecialCharTok{:}\NormalTok{arima\_process[}\DecValTok{5}\NormalTok{], }
                               \AttributeTok{Q =} \DecValTok{0}\SpecialCharTok{:}\NormalTok{arima\_process[}\DecValTok{6}\NormalTok{])}
        
\NormalTok{        non\_seasonal\_values }\OtherTok{\textless{}{-}} \FunctionTok{split}\NormalTok{(}\FunctionTok{as.matrix}\NormalTok{(valores[, }\DecValTok{1}\SpecialCharTok{:}\DecValTok{3}\NormalTok{]), }
                                     \FunctionTok{row}\NormalTok{(valores[, }\DecValTok{1}\SpecialCharTok{:}\DecValTok{3}\NormalTok{]))}
\NormalTok{        seasonal\_values }\OtherTok{\textless{}{-}} \FunctionTok{split}\NormalTok{(}\FunctionTok{as.matrix}\NormalTok{(valores[, }\DecValTok{4}\SpecialCharTok{:}\DecValTok{6}\NormalTok{]), }
                                 \FunctionTok{row}\NormalTok{(valores[, }\DecValTok{4}\SpecialCharTok{:}\DecValTok{6}\NormalTok{]))}
        

        \ControlFlowTok{if}\NormalTok{(parallelize}\SpecialCharTok{==}\ConstantTok{FALSE}\NormalTok{)\{}
\NormalTok{            arima\_models }\OtherTok{\textless{}{-}} \FunctionTok{mapply}\NormalTok{(arima\_model, }
                                   \AttributeTok{non\_seasonal=}\NormalTok{non\_seasonal\_values, }
                                   \AttributeTok{seasonal=}\NormalTok{seasonal\_values, }
                                   \AttributeTok{MoreArgs =} \FunctionTok{list}\NormalTok{(}\AttributeTok{time\_serie=}\NormalTok{train, }\AttributeTok{regr=}\NormalTok{reg, }
                                                   \AttributeTok{periodic=}\NormalTok{seasonal\_periodicity), }
                                   \AttributeTok{SIMPLIFY =} \ConstantTok{FALSE}\NormalTok{)}
\NormalTok{        \}}
        
        \ControlFlowTok{if}\NormalTok{(parallelize}\SpecialCharTok{==}\ConstantTok{TRUE}\NormalTok{)\{}
            

            
\NormalTok{            obj }\OtherTok{\textless{}{-}} \FunctionTok{sapply}\NormalTok{(}\FunctionTok{as.list}\NormalTok{(}\FunctionTok{match.call}\NormalTok{()), paste)}
\NormalTok{            pos }\OtherTok{\textless{}{-}} \FunctionTok{which}\NormalTok{(}\FunctionTok{names}\NormalTok{(obj) }\SpecialCharTok{\%in\%} \FunctionTok{c}\NormalTok{(}\StringTok{"time\_serie"}\NormalTok{, }\StringTok{"validacion"}\NormalTok{))}
            

\NormalTok{            clp }\OtherTok{\textless{}{-}} \FunctionTok{makeCluster}\NormalTok{(clusters, }\AttributeTok{type =} \StringTok{"SOCK"}\NormalTok{, }\AttributeTok{useXDR=}\ConstantTok{FALSE}\NormalTok{)}
            \FunctionTok{clusterExport}\NormalTok{(clp, }\AttributeTok{varlist =} \FunctionTok{c}\NormalTok{(obj[pos], }
                                           \StringTok{"arima\_measures"}\NormalTok{, }
                                           \StringTok{"arima\_model"}\NormalTok{))}
            

\NormalTok{            arima\_models }\OtherTok{\textless{}{-}} \FunctionTok{clusterMap}\NormalTok{(}\AttributeTok{cl=}\NormalTok{clp, }\AttributeTok{fun =}\NormalTok{ arima\_model, }
                                       \AttributeTok{non\_seasonal=}\NormalTok{non\_seasonal\_values, }
                                       \AttributeTok{seasonal=}\NormalTok{seasonal\_values, }
                                       \AttributeTok{MoreArgs =} \FunctionTok{list}\NormalTok{(}\AttributeTok{time\_serie=}\NormalTok{train, }
                                                       \AttributeTok{regr=}\NormalTok{reg, }
                                                       \AttributeTok{periodic=}\NormalTok{seasonal\_periodicity), }
                                       \AttributeTok{SIMPLIFY =} \ConstantTok{FALSE}\NormalTok{, }\AttributeTok{.scheduling =} \StringTok{"dynamic"}\NormalTok{)}
            \FunctionTok{stopCluster}\NormalTok{(clp)}
            
\NormalTok{        \}}
        

\NormalTok{        pos }\OtherTok{\textless{}{-}} \FunctionTok{which}\NormalTok{(}\FunctionTok{sapply}\NormalTok{(}\FunctionTok{lapply}\NormalTok{(arima\_models, class), length)}\SpecialCharTok{\textgreater{}}\DecValTok{1}\NormalTok{)}
        

\NormalTok{        final\_measures }\OtherTok{\textless{}{-}} \FunctionTok{do.call}\NormalTok{(}\StringTok{"rbind"}\NormalTok{, }\FunctionTok{lapply}\NormalTok{(arima\_models[pos], }
\NormalTok{                                                  arima\_measures, }
                                                  \AttributeTok{testing =}\NormalTok{ test, }
                                                  \AttributeTok{horizon=}\NormalTok{ horiz, }
                                                  \AttributeTok{regr =}\NormalTok{ reg)) }\SpecialCharTok{\%\textgreater{}\%} 
            \FunctionTok{mutate\_if}\NormalTok{(is.numeric, round, }\DecValTok{2}\NormalTok{)}

        \FunctionTok{list}\NormalTok{(}\AttributeTok{arima\_models=}\NormalTok{arima\_models[pos], }
             \AttributeTok{final\_measures=}\NormalTok{final\_measures, }
             \AttributeTok{bests=}\FunctionTok{arima\_selected}\NormalTok{(final\_measures, }
                                  \AttributeTok{Wtrain =}\NormalTok{ training\_weight, }
                                  \AttributeTok{Wtest =}\NormalTok{ testing\_weight))\})}
\NormalTok{\}}
\end{Highlighting}
\end{Shaded}

\subsection{Función de simulación de series cronológicas}

\captionof{chunk}{Función ts.sim}\label{simula_series}

\begin{Shaded}
\begin{Highlighting}[]
\NormalTok{ts.sim }\OtherTok{\textless{}{-}} \ControlFlowTok{function}\NormalTok{(data, n, temporalidad, }
\NormalTok{                            no.estacional, estacional, }
                            \AttributeTok{p=}\ConstantTok{NULL}\NormalTok{, }\AttributeTok{q=}\ConstantTok{NULL}\NormalTok{, }\AttributeTok{P=}\ConstantTok{NULL}\NormalTok{, }\AttributeTok{Q=}\ConstantTok{NULL}\NormalTok{)\{}
  
  \FunctionTok{require}\NormalTok{(forecast)}
  \FunctionTok{tryCatch}\NormalTok{(\{}
\NormalTok{    coeficientes }\OtherTok{\textless{}{-}} \FunctionTok{list}\NormalTok{(p, q, P, Q)}
\NormalTok{    coeficientes.simulados }\OtherTok{\textless{}{-}} \FunctionTok{lapply}\NormalTok{(}\FunctionTok{c}\NormalTok{(no.estacional[}\FunctionTok{c}\NormalTok{(}\DecValTok{1}\NormalTok{,}\DecValTok{3}\NormalTok{)], }
\NormalTok{                                       estacional[}\FunctionTok{c}\NormalTok{(}\DecValTok{1}\NormalTok{,}\DecValTok{3}\NormalTok{)]), }
                                     \ControlFlowTok{function}\NormalTok{(x) }\FunctionTok{sample}\NormalTok{(}\FunctionTok{seq}\NormalTok{(}\SpecialCharTok{{-}}\DecValTok{1}\NormalTok{,}\DecValTok{1}\NormalTok{,.}\DecValTok{1}\NormalTok{), x))}
\NormalTok{    pos }\OtherTok{\textless{}{-}} \FunctionTok{which}\NormalTok{(}\FunctionTok{sapply}\NormalTok{(coeficientes, is.null)}\SpecialCharTok{==}\ConstantTok{TRUE}\NormalTok{)}
    
\NormalTok{    coeficientes[pos] }\OtherTok{\textless{}{-}}\NormalTok{ coeficientes.simulados[pos]}
    \FunctionTok{names}\NormalTok{(coeficientes) }\OtherTok{\textless{}{-}} \FunctionTok{c}\NormalTok{(}\StringTok{"p"}\NormalTok{, }\StringTok{"q"}\NormalTok{, }\StringTok{"P"}\NormalTok{, }\StringTok{"Q"}\NormalTok{)}
    
\NormalTok{    modelo }\OtherTok{\textless{}{-}} \FunctionTok{Arima}\NormalTok{(}\FunctionTok{ts}\NormalTok{(}\AttributeTok{data=}\NormalTok{data, }\AttributeTok{freq=}\NormalTok{temporalidad), }
                    \AttributeTok{order =}\NormalTok{ no.estacional, }
                    \AttributeTok{seasonal =}\NormalTok{ estacional,}
                    \AttributeTok{fixed=}\FunctionTok{c}\NormalTok{(}\AttributeTok{phi=}\NormalTok{coeficientes}\SpecialCharTok{$}\NormalTok{p, }
                            \AttributeTok{theta=}\NormalTok{coeficientes}\SpecialCharTok{$}\NormalTok{q, }
                            \AttributeTok{Phi=}\NormalTok{coeficientes}\SpecialCharTok{$}\NormalTok{P, }
                            \AttributeTok{Theta=}\NormalTok{coeficientes}\SpecialCharTok{$}\NormalTok{Q))}
\NormalTok{    datos }\OtherTok{\textless{}{-}} \FunctionTok{simulate}\NormalTok{(modelo, }\AttributeTok{nsim=}\NormalTok{(n}\SpecialCharTok{+}\FunctionTok{length}\NormalTok{(data)))}
    
\NormalTok{    datos }\OtherTok{\textless{}{-}} \FunctionTok{subset}\NormalTok{(datos, }\AttributeTok{start=}\FunctionTok{length}\NormalTok{(data)}\SpecialCharTok{+}\DecValTok{1}\NormalTok{)}
    \FunctionTok{list}\NormalTok{(}\AttributeTok{modelo=}\NormalTok{modelo, }\AttributeTok{datos=}\NormalTok{datos)\}, }
    \AttributeTok{error =} \ControlFlowTok{function}\NormalTok{(e) }\FunctionTok{simular.proceso}\NormalTok{(data, n, temporalidad,}
\NormalTok{                                        no.estacional, estacional, }
\NormalTok{                                        p, q, P, Q))}
\NormalTok{\}}
\end{Highlighting}
\end{Shaded}

\newpage

\section{REFERENCIAS}

\hypertarget{refs}{}
\begin{CSLReferences}{1}{0}
\leavevmode\hypertarget{ref-medidas}{}%
Adhikari, R., K, A. R., \& Agrawal, R. K. (2013). \emph{An Introductory
Study on Time Series Modeling and Forecasting} (pp. 42-45). Recuperado
de \url{https://arxiv.org/ftp/arxiv/papers/1302/1302.6613.pdf}

\leavevmode\hypertarget{ref-stationary_def}{}%
Agrawal, R., \& Adhikari, R. (2013). An introductory study on time
series modeling and forecasting. \emph{Nova York: CoRR}.

\leavevmode\hypertarget{ref-pearson}{}%
Benesty, J., \& Chen, Y. and C., J.and Huang. (2009). Pearson
Correlation Coefficient. En \emph{Noise Reduction in Speech Processing}
(pp. 37-38). \url{https://doi.org/10.1007/978-3-642-00296-0_5}

\leavevmode\hypertarget{ref-box-jenkins}{}%
Box, G. E. P., Jenkins, G. M., \& Reinsel, G. C. (1994). \emph{Time
Series Analysis: Forecasting and Control}. Recuperado de
\url{https://books.google.co.cr/books?id=sRzvAAAAMAAJ}

\leavevmode\hypertarget{ref-yule.walker}{}%
Brockwell, P. J., \& Davis, R. A. (2009). \emph{Time Series: Theory and
Methods}. En \emph{Springer Series en Statistics} (p. 239). Recuperado
de \url{https://books.google.co.cr/books?id=_DcYu_EhVzUC}

\leavevmode\hypertarget{ref-brown}{}%
Brown, R. (1956). \emph{Exponential Smoothing for Predicting Demand}.
Recuperado de \url{https://www.industrydocuments.ucsf.edu/docs/jzlc0130}

\leavevmode\hypertarget{ref-burnham2007model}{}%
Burnham, K. P., \& Anderson, D. R. (2007). \emph{Model Selection and
Multimodel Inference: A Practical Information-Theoretic Approach}.
Recuperado de \url{https://books.google.co.cr/books?id=IWUKBwAAQBAJ}

\leavevmode\hypertarget{ref-calderon2012estadistica}{}%
Calderón, C. E. (2012). Estadística para Estudiantes de Administración
de Empresas de la Universidad Nacional del Callao. \emph{Editorial San
Marcos, 2da Edición, Lima Perú}. Recuperado de
\url{https://unac.edu.pe/documentos/organizacion/vri/cdcitra/Informes_Finales_Investigacion/IF_JUNIO_2012/IF_CALDERON\%20OTOYA_FCA/capitulo\%208.pdf}

\leavevmode\hypertarget{ref-10.2307ux2f1392184}{}%
Canova, F., \& Hansen, B. E. (1995). Are Seasonal Patterns Constant over
Time? A Test for Seasonal Stability. \emph{Journal of Business \&
Economic Statistics}, \emph{13}(3), 237-252. Recuperado de
\url{http://www.jstor.org/stable/1392184}

\leavevmode\hypertarget{ref-Cochrane}{}%
Cochrane, J. H. (1997). \emph{Time Series for Macroeconomics and
Finance}. Recuperado de
\url{http://econ.lse.ac.uk/staff/wdenhaan/teach/cochrane.pdf}

\leavevmode\hypertarget{ref-tsa_decades}{}%
De Gooijer, J. G., \& Hyndman, R. J. (2006). 25 years of time series
forecasting. \emph{International Journal of Forecasting}, \emph{22}(3),
443-473.
https://doi.org/\url{https://doi.org/10.1016/j.ijforecast.2006.01.001}

\leavevmode\hypertarget{ref-Lombardo}{}%
Flaherty, J., \& Lombardo, R. (2000, enero). \emph{Modelling Private New
Housing Starts in Australia}. Recuperado de
\url{http://www.prres.net/papers/Flaherty_Modelling_Private_New_Housing_Starts_In_Australia.pdf}

\leavevmode\hypertarget{ref-fuller1995introduction}{}%
Fuller, W. A. (1995). \emph{Introduction to Statistical Time Series}.
Recuperado de \url{https://books.google.co.cr/books?id=wyRhjmAPQIYC}

\leavevmode\hypertarget{ref-Hamzacebi}{}%
Hamzaçebi, C. (2008). Improving Artificial Neural Networks' Performance
in Seasonal Time Series Forecasting. \emph{Inf. Sci.}, \emph{178}(23),
4550-4559. \url{https://doi.org/10.1016/j.ins.2008.07.024}

\leavevmode\hypertarget{ref-oscarh-1}{}%
Hernández, O. (2011a). \emph{Introducción a las Series Cronológicas}
(1.ª ed.). Recuperado de
\url{http://www.editorial.ucr.ac.cr/ciencias-naturales-y-exactas/item/1985-introduccion-a-las-series-cronologicas.html}

\leavevmode\hypertarget{ref-oscarh-4}{}%
Hernández, O. (2011b). \emph{Introducción a las Series Cronológicas}.
Recuperado de
\url{http://www.editorial.ucr.ac.cr/ciencias-naturales-y-exactas/item/1985-introduccion-a-las-series-cronologicas.html}

\leavevmode\hypertarget{ref-Hipel}{}%
Hipel, K. W., \& McLeod, A. I. (1994). \emph{Time Series Modelling of
Water Resources and Environmental Systems}. Recuperado de
\url{https://books.google.co.cr/books?id=t1zG8OUbgdgC}

\leavevmode\hypertarget{ref-hyndman2018forecasting}{}%
Hyndman, R. J., \& Athanasopoulos, G. (2018a). \emph{Forecasting:
principles and practice}. Recuperado de
\url{https://books.google.co.cr/books?id=_bBhDwAAQBAJ}

\leavevmode\hypertarget{ref-hyndman_box-jenkins}{}%
Hyndman, R. J., \& Athanasopoulos, G. (2018b). \emph{Forecasting:
principles and practice}. Recuperado de
\url{https://books.google.co.cr/books?id=_bBhDwAAQBAJ}

\leavevmode\hypertarget{ref-auto.arima}{}%
Hyndman, R., \& Khandakar, Y. (2008). Automatic Time Series Forecasting:
The forecast Package for R. \emph{Journal of Statistical Software,
Articles}, \emph{27}(3), 1-22.
\url{https://doi.org/10.18637/jss.v027.i03}

\leavevmode\hypertarget{ref-calidad_vitales}{}%
INEC. (2017). \emph{Población, nacimientos, defunciones y matrimonios}.
Recuperado de
\url{http://inec.cr/sites/default/files/documetos-biblioteca-virtual/repoblacev2017_0.pdf}

\leavevmode\hypertarget{ref-bayes}{}%
Jammalamadaka, S. R., Qiu, J., \& Ning, N. (2018). \emph{Multivariate
Bayesian Structural Time Series Model}. Recuperado de
\url{https://arxiv.org/pdf/1801.03222.pdf}

\leavevmode\hypertarget{ref-kedem}{}%
Kedem, B., \& Fokianos, K. (2005). \emph{Regression Models for Time
Series Analysis}. Recuperado de
\url{https://books.google.co.cr/books?id=8r0qE35wt44C}

\leavevmode\hypertarget{ref-Lee}{}%
Lee, J. (s.~f.). Univariate time series modeling and forecasting
(Box-Jenkins Method). \emph{Econ 413, lecture 4}.

\leavevmode\hypertarget{ref-Osborn2009SEASONALITYAT}{}%
Osborn, D. R., Chui, A. P. L., Smith, J., \& Birchenhall, C. (2009).
\emph{Seasonality and the order of integration for consumption}.
Recuperado de
\url{http://www.est.uc3m.es/esp/nueva_docencia/comp_col_get/lade/tecnicas_prediccion/OCSB_OxBull1988.pdf}

\leavevmode\hypertarget{ref-parallel}{}%
R Core Team. (2019). \emph{R: A Language and Environment for Statistical
Computing}. Recuperado de \url{https://www.R-project.org/}

\leavevmode\hypertarget{ref-tsa_decision_making}{}%
Rezaee, Z., Aliabadi, S., Dorestani, A., \& Rezaee, N. J. (2020).
Application of Time Series Models in Business Research: Correlation,
Association, Causation. \emph{Sustainability}, \emph{12}(12), 4833.

\leavevmode\hypertarget{ref-supenprodc}{}%
Rosero-Bixby, L. (2018). \emph{Producto C para SUPEN. Proyección de la
mortalidad de Costa Rica 2015-2150}. Recuperado de CCP-UCR website:
\url{http://srv-website.cloudapp.net/documents/10179/999061/Nota+t}

\leavevmode\hypertarget{ref-sargent_macro}{}%
Sargent, T. J. (1979). \emph{Macroeconomic Theory}. Recuperado de
\url{https://books.google.co.cr/books?id=X6u7AAAAIAAJ}

\leavevmode\hypertarget{ref-astsa}{}%
Stoffer, D. (2020). \emph{astsa: Applied Statistical Time Series
Analysis}. Recuperado de \url{https://CRAN.R-project.org/package=astsa}

\leavevmode\hypertarget{ref-Wold}{}%
Surhone, L. M., Timpledon, M. T., \& Marseken, S. F. (2010). \emph{Wold
Decomposition}. Recuperado de
\url{https://books.google.co.cr/books?id=7cSqcQAACAAJ}

\leavevmode\hypertarget{ref-redes}{}%
Tadayon, M., \& Iwashita, Y. (2020). \emph{Comprehensive Analysis of
Time Series Forecasting Using Neural Networks}. Recuperado de
\url{https://arxiv.org/pdf/2001.09547.pdf}

\leavevmode\hypertarget{ref-dplyr}{}%
Wickham, H., François, R., Henry, L., \& Müller, K. (2019). \emph{dplyr:
A Grammar of Data Manipulation}. Recuperado de
\url{https://CRAN.R-project.org/package=dplyr}

\leavevmode\hypertarget{ref-tidyr}{}%
Wickham, H., \& Henry, L. (2019). \emph{tidyr: Tidy Messy Data}.
Recuperado de \url{https://CRAN.R-project.org/package=tidyr}

\leavevmode\hypertarget{ref-doi:10.1111ux2f1467-9892.00213}{}%
Xiao, Z. (2001). Testing the Null Hypothesis of Stationarity Against an
Autoregressive Unit Root Alternative. \emph{Journal of Time Series
Analysis}, \emph{22}(1), 87-105.
\url{https://doi.org/10.1111/1467-9892.00213}

\leavevmode\hypertarget{ref-Zhang}{}%
Zhang, G. (2003). Time series forecasting using a hybrid ARIMA and
neural network model. \emph{Neurocomputing}, \emph{50}, 159-175.

\end{CSLReferences}

\end{document}
