\documentclass[]{article}
\usepackage{lmodern}
\usepackage{amssymb,amsmath}
\usepackage{ifxetex,ifluatex}
\usepackage{fixltx2e} % provides \textsubscript
\ifnum 0\ifxetex 1\fi\ifluatex 1\fi=0 % if pdftex
  \usepackage[T1]{fontenc}
  \usepackage[utf8]{inputenc}
\else % if luatex or xelatex
  \ifxetex
    \usepackage{mathspec}
  \else
    \usepackage{fontspec}
  \fi
  \defaultfontfeatures{Ligatures=TeX,Scale=MatchLowercase}
\fi
% use upquote if available, for straight quotes in verbatim environments
\IfFileExists{upquote.sty}{\usepackage{upquote}}{}
% use microtype if available
\IfFileExists{microtype.sty}{%
\usepackage{microtype}
\UseMicrotypeSet[protrusion]{basicmath} % disable protrusion for tt fonts
}{}
\usepackage[margin=1in]{geometry}
\usepackage{hyperref}
\PassOptionsToPackage{usenames,dvipsnames}{color} % color is loaded by hyperref
\hypersetup{unicode=true,
            pdftitle={Determinación de modelos ARIMA vía sobre parametrización según la temporalidad de la serie cronológica con aplicaciones en datos costarricenses},
            colorlinks=true,
            linkcolor=blue,
            citecolor=Blue,
            urlcolor=blue,
            breaklinks=true}
\urlstyle{same}  % don't use monospace font for urls
\usepackage{graphicx,grffile}
\makeatletter
\def\maxwidth{\ifdim\Gin@nat@width>\linewidth\linewidth\else\Gin@nat@width\fi}
\def\maxheight{\ifdim\Gin@nat@height>\textheight\textheight\else\Gin@nat@height\fi}
\makeatother
% Scale images if necessary, so that they will not overflow the page
% margins by default, and it is still possible to overwrite the defaults
% using explicit options in \includegraphics[width, height, ...]{}
\setkeys{Gin}{width=\maxwidth,height=\maxheight,keepaspectratio}
\IfFileExists{parskip.sty}{%
\usepackage{parskip}
}{% else
\setlength{\parindent}{0pt}
\setlength{\parskip}{6pt plus 2pt minus 1pt}
}
\setlength{\emergencystretch}{3em}  % prevent overfull lines
\providecommand{\tightlist}{%
  \setlength{\itemsep}{0pt}\setlength{\parskip}{0pt}}
\setcounter{secnumdepth}{0}
% Redefines (sub)paragraphs to behave more like sections
\ifx\paragraph\undefined\else
\let\oldparagraph\paragraph
\renewcommand{\paragraph}[1]{\oldparagraph{#1}\mbox{}}
\fi
\ifx\subparagraph\undefined\else
\let\oldsubparagraph\subparagraph
\renewcommand{\subparagraph}[1]{\oldsubparagraph{#1}\mbox{}}
\fi

%%% Use protect on footnotes to avoid problems with footnotes in titles
\let\rmarkdownfootnote\footnote%
\def\footnote{\protect\rmarkdownfootnote}

%%% Change title format to be more compact
\usepackage{titling}

% Create subtitle command for use in maketitle
\providecommand{\subtitle}[1]{
  \posttitle{
    \begin{center}\large#1\end{center}
    }
}

\setlength{\droptitle}{-2em}

  \title{Determinación de modelos ARIMA vía sobre parametrización según la
temporalidad de la serie cronológica con aplicaciones en datos
costarricenses}
    \pretitle{\vspace{\droptitle}\centering\huge}
  \posttitle{\par}
    \author{Universidad de Costa Rica\\
César Gamboa Sanabria\\
www.cesargamboasanabria.com\\
\href{mailto:info@cesargamboasanabria.com}{\nolinkurl{info@cesargamboasanabria.com}}}
    \preauthor{\centering\large\emph}
  \postauthor{\par}
      \predate{\centering\large\emph}
  \postdate{\par}
    \date{03 octubre, 2019}

\usepackage{geometry}
\geometry{a4paper, left=20mm, right=30mm}
\usepackage{setspace}
\doublespacing
\usepackage[spanish]{babel}
\usepackage{color}
\usepackage{xcolor}
\usepackage{framed}
\colorlet{shadecolor}{gray!20}
\setcounter{secnumdepth}{0}
\usepackage{sectsty}
\chapternumberfont{\Large}
\chaptertitlefont{\Large}
\setcounter{tocdepth}{5}
\setcounter{secnumdepth}{5}
\usepackage{graphics}
\usepackage{setspace} %paquete para el doble espaciado
\doublespacing %inicia el doble espaciado
%Esto quita el punto final en la numeracion de cada seccion
\usepackage{tocloft}
\usepackage{titlesec}
\titleformat{\section}
{\Large\bfseries}{\thesection}{0.5em}{}
\titleformat{\subsection}
{\large\bfseries}{\thesubsection}{0.5em}{}
\titleformat{\subsubsection}
{\normalsize\bfseries}{\thesubsubsection}{0.5em}{}
\titleformat{\paragraph}
{\normalsize\bfseries}{\theparagraph}{0.5em}{}
\renewcommand\cftsecaftersnum{}
\renewcommand\thesection{\arabic{section}}
\renewcommand\thesubsection{\thesection.\arabic{subsection}}
\usepackage{caption}
\usepackage{fancyhdr}
\pagestyle{fancy}
\fancyhf{}
\fancyhead[R]{\rightmark}
\fancyfoot[C]{\thepage}
\setlength{\headheight}{21.9pt}
\renewcommand\sectionmark[1]{
\markright{\thesection\ #1}}

\begin{document}
\maketitle

\newpage

\section*{RESUMEN}

La metodología de Box-Jenkins busca encontrar el mejor proceso
autorregresivo integrado de medias móviles (ARIMA) que explique una
serie temporal \(y_t\) de \(T\) periodos, para pronosticar hasta
\(T+h\). El paquete \texttt{forecast} de R permite hacer uso de la
función \texttt{auto.arima()} para estimar un modelo ARIMA basado en
pruebas de raíz unitaria, minimización del AICc y de la MLE. De esta
forma se obtiene un modelo temporal definiendo las diferenciaciones
requeridas en la parte estacional d mediante las pruebas KPSS o ADF, y
la no estacional D utilizando las pruebas OCSB o la Canova-Hansen,
seleccionado el orden óptimo para los términos \(ARIMA(p,d,q)(P,D,Q)\)
para una serie \(y_t\). Se propone un método de selección fundamentada
en las permutaciones de los parámetros de un modelo ARIMA, seleccionando
la mejor especificación con base en medidas de rendimiento MAE, RMSE,
MAPE y MASE: se comparan todos los posibles términos definiendo una
diferenciación adecuada para la serie y permutando hasta un máximo
determinado para los términos de especificación de un
\(ARIMA(p,d,q)(P,D,Q)\). El método propuesto se probó en 6 series
cronológicas de distinta temporalidad: mortalidad infantil, mortalidad
por causa externa, nacimientos, demanda eléctrica, intereses y
comisiones del sector público, e incentivos salariales del sector
público.

\textbf{\emph{Palabras clave}}: ARIMA, R, automatización, selección,
estadística

\newpage

\tableofcontents

\newpage

\section{INTRODUCCIÓN}

El manejo de información obtenida de manera secuencial a lo largo del
tiempo hace referencia al uso de series cronológicas. Este tipo de datos
se encuentras en diferentes áreas de investigación. En el campo
financiero, por ejemplo, es común hablar de la devaluación del colón con
respecto al dólar, cantidad de exportaciones mensuales de un determinado
producto o las ventas de este (Hernández
\protect\hyperlink{ref-oscarh-1}{2011}\protect\hyperlink{ref-oscarh-1}{a}).

En demografía, por ejemplo, el tema de las proyecciones de población
tiene un alto impacto a nivel social, pues conocer con anticipación en
posible comportamiento de la población en el futuro es clave para una
adecuada planificación en diversos proyectos sobre los cuales se debe
distribuir un presupuesto que es finito. Durante una emergencia, que
difícilmente se sabe cuándo ocurrirá, conocer la posible población que
se tiene en una zona es clave para la rápida reacción de las autoridades
para el envío de ayuda o para ejecutar planes de evacuación.

El campo actuarial también se ve beneficiado al mejorar sus métodos de
pronóstico, pues uno de sus campos de estudio es la mortalidad pues
representan un insumo de vital importancia para la planificación y
sostenibilidad de los sistemas de pensiones, servicios de salud tanto
pública como privada, seguros de vida y asuntos hipotecarios
(Rosero-Bixby \protect\hyperlink{ref-supenprodc}{2018}).

Sin embargo, las series cronológicas por sí solas representan solo un
insumo para abordar, como mínimo, tres objetivos básicos: 1) realizar
análisis exploratorios de estos datos mediante métodos de visualización
y medidas de posición y variabilidad, como ver su crecimiento o
decrecimiento a lo largo del tiempo, detectar valores atípicos o cambios
drásticos en el nivel o valor medio de la serie, 2) generar modelos
estadísticos que sirvan como un simplificación de la realidad, y 3)
generar pronósticos para los posibles valores futuros que tomará el
problema en cuestión (Hernández
\protect\hyperlink{ref-oscarh-2}{2011}\protect\hyperlink{ref-oscarh-2}{b}).

Los tres objetivos anteriores se trabajan de manera secuencial, pues es
necesario realizar primero el análisis exploratorio de los datos para
tener una noción global del panorama y así conocer la serie cronológica
con la que se está trabajando. Una vez hecho esto, existen múltiples
formas de generar modelos para estos datos, como por ejemplo los métodos
de suavizamiento exponencial desarrollados en la década de 1950 (Brown
\protect\hyperlink{ref-brown}{1956}), modelos de regresión para series
temporales (Kedem and Fokianos \protect\hyperlink{ref-kedem}{2005}) o
los procesos autorregresivos integrados de medias móviles (ARIMA) (Box,
Jenkins, and Reinsel \protect\hyperlink{ref-box-jenkins}{1994}). Cuando
se ha establecido el modelo, los pronósticos son utilizados en
instituciones públicas, gobiernos municipales, instituciones del sector
privado, centros académicos, población civil, centros nacionales o
regionales de investigación y ONG dedicadas al desarrollo social. Si las
entidades previamente mencionadas cuentan con proyecciones de calidad,
la puesta en marcha de sus respectivos planes tendrá un impacto mayor y
más efectivo.

De lo anterior, generar un modelo adecuado es fundamental para obtener
un pronóstico de calidad, y es aquí donde resulta importante mencionar
una diferencia clave entre los dos modelos clásicos más comúnmente
utilizados: los modelos de suavizamiento y los modelos ARIMA. Ambos
representan enfoques complementarios a un problema, pues según Hyndman
(Hyndman and Athanasopoulos
\protect\hyperlink{ref-hyndman2018forecasting}{2018}), los modelos de
suavizamiento exponencial se fundamentan en un enfoque más descriptivo
de los componentes de la serie cronológica en estudio; mientras que los
modelos ARIMA tienen como objetivo explicar las relaciones pasadas de
ésta. La importancia de la metodología de Box-Jenkins radica en que no
supone ningún patrón en particular en la serie histórica que se busca
pronosticar, sino que contempla un proceso iterativo para identificar un
posible modelo a partir de una clase general de modelos y luego someter
dicho modelo a diferentes pruebas y medidas de rendimiento para evaluar
su ajuste. Al trabajar la metodología de Box-Jenkins, uno de los pasos
es identificar el los parámetros del proceso ARIMA(p,d,q)(P,D,Q) que
gobiernan la serie, siendo la manera clásica de trabajar este paso, el
análisis visual de las funciones de autocorrelación parcial y total.

El gran obstáculo que presenta esta identificación visual es que en la
actualidad contar con una gran cantidad de series cronológicas para
analizar es algo muy común. Incluso con cantidades moderadas de series
cronológicas a analizar, es difícil contar con personal capacitado para
realizar este análisis visual y poder identificar los modelos, por lo
que la generación de algoritmos que ayuden a dicha identificación se
vuelven cada vez más necesarios (Hyndman and Khandakar
\protect\hyperlink{ref-auto.arima}{2008}).

Han sido varias las aproximaciones a un método que genere de manera
automática un modelo ARIMA, como por ejemplo los propuestos por Hannan y
Rissanen (Hannan and Rissanen \protect\hyperlink{ref-hannan}{1982}), la
extensión de dicha propuesta realizada por Gómez (Gómez
\protect\hyperlink{ref-gomez}{1998}) y posteriormente aplicada (Gómez
and Maraval \protect\hyperlink{ref-tramo}{1998}) en los software
\textbf{TRAMO} y \textbf{SEATS}; de manera similar se planteó una
aplicación en los software \textbf{SCA-Expert} (Liu
\protect\hyperlink{ref-liu}{1989}) y \textbf{TSE-AX} (Mélard and
Pasteels \protect\hyperlink{ref-melard}{2000}). Otros algoritmos
implementados en programas de cómputo de paga son \textbf{Forecast Pro}
(Goodrich \protect\hyperlink{ref-forecastpro}{2000}) y \textbf{Autobox}
(Reilly \protect\hyperlink{ref-autobox}{2000}). Uno de los métodos
automatizados de estimación es el que ofrece el paquete
\texttt{forecast} (Hyndman and Khandakar
\protect\hyperlink{ref-auto.arima}{2008}) del lenguaje de programación
R\footnote{\url{https://cran.r-project.org/}} permite hacer uso de la
función \texttt{auto.arima()} para estimar un modelo ARIMA basado en
pruebas de raíz unitaria, minimización del AICc y de la MLE. De esta
forma se obtiene un modelo temporal definiendo las diferenciaciones
requeridas en la parte estacional d mediante las pruebas KPSS o ADF, y
la no estacional D utilizando las pruebas OCSB o la Canova-Hansen,
seleccionado el orden óptimo para los términos \(ARIMA(p,d,q)(P,D,Q)_s\)
para una serie cronológica determinada.

Es a partir de esta necesidad que se propone una metodología para la
estimación del mejor modelo ARIMA para una serie cronológica determinada
cuya temporalidad sea mensual, bimensual, trimestral o cuatrimestral
mediante un proceso de selección fundamentada en las permutaciones de
todos los parámetros de un modelo ARIMA hasta un cierto límite,
considerando además la inclusión semi-automática de intervenciones en
periodos específicos y la validación cruzada para evaluar la calidad de
las particiones de la base de datos en conjuntos para entrenar y probar
el rendimiento del modelo; dichas pruebas involucran, entre otras
medidas de rendimiento, el MAE, RMSE, MAPE y MASE, las cuales sirven de
insumo para utilizar un método de consenso entre ellas para seleccionar
el modelo más adecuado: se comparan todos los posibles términos
definiendo una diferenciación adecuada para la serie y permutando hasta
un máximo definido para los términos de especificación de un
\(ARIMA(p,d,q)(P,D,Q)_s\) para así seleccionar la especificación que
ofrezca mejores resultados al momento de pronosticar valores futuros de
la serie cronológica. El método propuesto se probará comparándose con
los resultados de 6 series con distintas temporalidades: mortalidad
infantil, mortalidad por causa externa, nacimientos, demanda eléctrica,
intereses y comisiones del sector público, e incentivos salariales del
sector público.

\subsection{Contribución de la tesis a la Estadística como disciplina}

El principal aporte de este estudio es, por medio de un estudio de
simulación, aportar evidencia sobre cómo la sobreparametrización puede
representar una herramienta para definir la especificación de un modelo
ARIMA que genere pronósticos adecuados, contrastando la calidad de estos
con respecto a otros métodos similares, como lo es la función
auto.arima().

\subsection{Objetivos}

\subsubsection{Objetivos general}

\begin{itemize}
\tightlist
\item
  Evaluar la calidad de los pronósticos realizados con modelos ARIMA
  especificados vía sobre parametrización.
\end{itemize}

\subsubsection{Objetivos específicos}

\begin{itemize}
\tightlist
\item
  Diseñar un algoritmo para la selección del mejor modelo ARIMA según la
  temporalidad de la serie.
\item
  Aplicar validación cruzada en distintos horizontes de pronóstico para
  identificar la mejor especificación de un modelo ARIMA.
\item
  Comparar la precisión de los pronósticos con el método propuesto por
  Rob Hyndman.
\item
  Integrar la metodología de análisis de series temporales en una
  librería del lenguaje estadístico R.
\end{itemize}

\section{REFERENCIAS}

\hypertarget{refs}{}
\leavevmode\hypertarget{ref-box-jenkins}{}%
Box, G.E.P., G.M. Jenkins, and G.C. Reinsel. 1994. \emph{Time Series
Analysis: Forecasting and Control}. Forecasting and Control Series.
Prentice Hall. \url{https://books.google.co.cr/books?id=sRzvAAAAMAAJ}.

\leavevmode\hypertarget{ref-brown}{}%
Brown, Robert G. 1956. \emph{Exponential Smoothing for Predicting
Demand}. A.D.Little.
\url{https://www.industrydocuments.ucsf.edu/docs/jzlc0130}.

\leavevmode\hypertarget{ref-forecastpro}{}%
Goodrich, RL. 2000. ``The Forecast Pro Methodology.''
\emph{International Journal of Forecasting} 16 (4): 533--35.
\url{http://www.forecasting-competition.com/downloads/NN3/methods/Goodrich\%20(2000)\%20The\%20Forecast\%20Pro\%20methodology\%20science.pdf}.

\leavevmode\hypertarget{ref-gomez}{}%
Gómez, V. 1998. ``Automatic Model Identification in the Presence of
Missing Observations and Outliers.'' Edited by Dirección General de
Análisis y Programación Presupuestaria Ministerio de Economía y
Hacienda. Working paper D-98009.

\leavevmode\hypertarget{ref-tramo}{}%
Gómez, V., and A. Maraval. 1998. ``Programs Tramo and Seats,
Instructions for the Users.'' Edited by Dirección General de Análisis y
Programación Presupuestaria Ministerio de Economía y Hacienda. Working
paper 97001.

\leavevmode\hypertarget{ref-hannan}{}%
Hannan, E. J., and J. Rissanen. 1982. ``Recursive Estimation of Mixed
Autoregressive-Moving Average Order.'' \emph{Biometrika} 69 (1): 81--94.
\url{http://www.jstor.org/stable/2335856}.

\leavevmode\hypertarget{ref-oscarh-1}{}%
Hernández, O. 2011a. ``Introducción a Las Series Cronológicas.'' In, 1st
ed., 1. Editorial Universidad de Costa Rica.
\url{http://www.editorial.ucr.ac.cr/ciencias-naturales-y-exactas/item/1985-introduccion-a-las-series-cronologicas.html}.

\leavevmode\hypertarget{ref-oscarh-2}{}%
---------. 2011b. ``Introducción a Las Series Cronológicas.'' In, 1st
ed., 2. Editorial Universidad de Costa Rica.
\url{http://www.editorial.ucr.ac.cr/ciencias-naturales-y-exactas/item/1985-introduccion-a-las-series-cronologicas.html}.

\leavevmode\hypertarget{ref-hyndman2018forecasting}{}%
Hyndman, R.J., and G. Athanasopoulos. 2018. \emph{Forecasting:
Principles and Practice}. OTexts.
\url{https://books.google.co.cr/books?id=/_bBhDwAAQBAJ}.

\leavevmode\hypertarget{ref-auto.arima}{}%
Hyndman, Rob, and Yeasmin Khandakar. 2008. ``Automatic Time Series
Forecasting: The Forecast Package for R.'' \emph{Journal of Statistical
Software, Articles} 27 (3): 1--22.
\url{https://doi.org/10.18637/jss.v027.i03}.

\leavevmode\hypertarget{ref-kedem}{}%
Kedem, B., and K. Fokianos. 2005. \emph{Regression Models for Time
Series Analysis}. Wiley Series in Probability and Statistics. Wiley.
\url{https://books.google.co.cr/books?id=8r0qE35wt44C}.

\leavevmode\hypertarget{ref-liu}{}%
Liu, Lon-Mu. 1989. ``Identification of Seasonal Arima Models Using a
Filtering Method.'' \emph{Communications in Statistics - Theory and
Methods} 18 (6): 2279--88.
\url{https://doi.org/10.1080/03610928908830035}.

\leavevmode\hypertarget{ref-melard}{}%
Mélard, G., and J.-M. Pasteels. 2000. ``Automatic Arima Modeling
Including Interventions, Using Time Series Expert Software.''
\emph{International Journal of Forecasting} 16 (4): 497--508.
\url{https://doi.org/https://doi.org/10.1016/S0169-2070(00)00067-4}.

\leavevmode\hypertarget{ref-autobox}{}%
Reilly, D. 2000. ``The Autobox System.'' \emph{International Journal of
Forecasting} 16 (4): 531--33.
\url{https://ideas.repec.org/a/eee/intfor/v16y2000i4p531-533.html}.

\leavevmode\hypertarget{ref-supenprodc}{}%
Rosero-Bixby, L. 2018. ``Producto c Para Supen. Proyección de La
Mortalidad de Costa Rica 2015-2150.'' CCP-UCR.
\url{http://srv-website.cloudapp.net/documents/10179/999061/Nota+t\%C3\%A9cnica+tablas+de+vida+segunda+parte}.


\end{document}
