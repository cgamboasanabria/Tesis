% Options for packages loaded elsewhere
\PassOptionsToPackage{unicode}{hyperref}
\PassOptionsToPackage{hyphens}{url}
\PassOptionsToPackage{dvipsnames,svgnames*,x11names*}{xcolor}
%
\documentclass[
]{article}
\usepackage{lmodern}
\usepackage{amssymb,amsmath}
\usepackage{ifxetex,ifluatex}
\ifnum 0\ifxetex 1\fi\ifluatex 1\fi=0 % if pdftex
  \usepackage[T1]{fontenc}
  \usepackage[utf8]{inputenc}
  \usepackage{textcomp} % provide euro and other symbols
\else % if luatex or xetex
  \usepackage{unicode-math}
  \defaultfontfeatures{Scale=MatchLowercase}
  \defaultfontfeatures[\rmfamily]{Ligatures=TeX,Scale=1}
\fi
% Use upquote if available, for straight quotes in verbatim environments
\IfFileExists{upquote.sty}{\usepackage{upquote}}{}
\IfFileExists{microtype.sty}{% use microtype if available
  \usepackage[]{microtype}
  \UseMicrotypeSet[protrusion]{basicmath} % disable protrusion for tt fonts
}{}
\makeatletter
\@ifundefined{KOMAClassName}{% if non-KOMA class
  \IfFileExists{parskip.sty}{%
    \usepackage{parskip}
  }{% else
    \setlength{\parindent}{0pt}
    \setlength{\parskip}{6pt plus 2pt minus 1pt}}
}{% if KOMA class
  \KOMAoptions{parskip=half}}
\makeatother
\usepackage{xcolor}
\IfFileExists{xurl.sty}{\usepackage{xurl}}{} % add URL line breaks if available
\IfFileExists{bookmark.sty}{\usepackage{bookmark}}{\usepackage{hyperref}}
\hypersetup{
  colorlinks=true,
  linkcolor=blue,
  filecolor=Maroon,
  citecolor=Blue,
  urlcolor=blue,
  pdfcreator={LaTeX via pandoc}}
\urlstyle{same} % disable monospaced font for URLs
\usepackage[margin=1in]{geometry}
\usepackage{color}
\usepackage{fancyvrb}
\newcommand{\VerbBar}{|}
\newcommand{\VERB}{\Verb[commandchars=\\\{\}]}
\DefineVerbatimEnvironment{Highlighting}{Verbatim}{commandchars=\\\{\}}
% Add ',fontsize=\small' for more characters per line
\usepackage{framed}
\definecolor{shadecolor}{RGB}{248,248,248}
\newenvironment{Shaded}{\begin{snugshade}}{\end{snugshade}}
\newcommand{\AlertTok}[1]{\textcolor[rgb]{0.94,0.16,0.16}{#1}}
\newcommand{\AnnotationTok}[1]{\textcolor[rgb]{0.56,0.35,0.01}{\textbf{\textit{#1}}}}
\newcommand{\AttributeTok}[1]{\textcolor[rgb]{0.77,0.63,0.00}{#1}}
\newcommand{\BaseNTok}[1]{\textcolor[rgb]{0.00,0.00,0.81}{#1}}
\newcommand{\BuiltInTok}[1]{#1}
\newcommand{\CharTok}[1]{\textcolor[rgb]{0.31,0.60,0.02}{#1}}
\newcommand{\CommentTok}[1]{\textcolor[rgb]{0.56,0.35,0.01}{\textit{#1}}}
\newcommand{\CommentVarTok}[1]{\textcolor[rgb]{0.56,0.35,0.01}{\textbf{\textit{#1}}}}
\newcommand{\ConstantTok}[1]{\textcolor[rgb]{0.00,0.00,0.00}{#1}}
\newcommand{\ControlFlowTok}[1]{\textcolor[rgb]{0.13,0.29,0.53}{\textbf{#1}}}
\newcommand{\DataTypeTok}[1]{\textcolor[rgb]{0.13,0.29,0.53}{#1}}
\newcommand{\DecValTok}[1]{\textcolor[rgb]{0.00,0.00,0.81}{#1}}
\newcommand{\DocumentationTok}[1]{\textcolor[rgb]{0.56,0.35,0.01}{\textbf{\textit{#1}}}}
\newcommand{\ErrorTok}[1]{\textcolor[rgb]{0.64,0.00,0.00}{\textbf{#1}}}
\newcommand{\ExtensionTok}[1]{#1}
\newcommand{\FloatTok}[1]{\textcolor[rgb]{0.00,0.00,0.81}{#1}}
\newcommand{\FunctionTok}[1]{\textcolor[rgb]{0.00,0.00,0.00}{#1}}
\newcommand{\ImportTok}[1]{#1}
\newcommand{\InformationTok}[1]{\textcolor[rgb]{0.56,0.35,0.01}{\textbf{\textit{#1}}}}
\newcommand{\KeywordTok}[1]{\textcolor[rgb]{0.13,0.29,0.53}{\textbf{#1}}}
\newcommand{\NormalTok}[1]{#1}
\newcommand{\OperatorTok}[1]{\textcolor[rgb]{0.81,0.36,0.00}{\textbf{#1}}}
\newcommand{\OtherTok}[1]{\textcolor[rgb]{0.56,0.35,0.01}{#1}}
\newcommand{\PreprocessorTok}[1]{\textcolor[rgb]{0.56,0.35,0.01}{\textit{#1}}}
\newcommand{\RegionMarkerTok}[1]{#1}
\newcommand{\SpecialCharTok}[1]{\textcolor[rgb]{0.00,0.00,0.00}{#1}}
\newcommand{\SpecialStringTok}[1]{\textcolor[rgb]{0.31,0.60,0.02}{#1}}
\newcommand{\StringTok}[1]{\textcolor[rgb]{0.31,0.60,0.02}{#1}}
\newcommand{\VariableTok}[1]{\textcolor[rgb]{0.00,0.00,0.00}{#1}}
\newcommand{\VerbatimStringTok}[1]{\textcolor[rgb]{0.31,0.60,0.02}{#1}}
\newcommand{\WarningTok}[1]{\textcolor[rgb]{0.56,0.35,0.01}{\textbf{\textit{#1}}}}
\usepackage{graphicx}
\makeatletter
\def\maxwidth{\ifdim\Gin@nat@width>\linewidth\linewidth\else\Gin@nat@width\fi}
\def\maxheight{\ifdim\Gin@nat@height>\textheight\textheight\else\Gin@nat@height\fi}
\makeatother
% Scale images if necessary, so that they will not overflow the page
% margins by default, and it is still possible to overwrite the defaults
% using explicit options in \includegraphics[width, height, ...]{}
\setkeys{Gin}{width=\maxwidth,height=\maxheight,keepaspectratio}
% Set default figure placement to htbp
\makeatletter
\def\fps@figure{htbp}
\makeatother
\setlength{\emergencystretch}{3em} % prevent overfull lines
\providecommand{\tightlist}{%
  \setlength{\itemsep}{0pt}\setlength{\parskip}{0pt}}
\setcounter{secnumdepth}{-\maxdimen} % remove section numbering


%\usepackage{fancyhdr}
%\pagestyle{fancy}
%\rhead{\includegraphics[width = 1\textwidth]{marca.jpg}}


\usepackage{geometry}
\geometry{a4paper, left=35mm, right=25mm, bottom=15mm}
\usepackage{setspace}
\doublespacing
\usepackage[spanish]{babel}
\usepackage{color}
\usepackage{xcolor}
\usepackage{framed}
\colorlet{shadecolor}{gray!20}
\setcounter{secnumdepth}{0}
\usepackage{sectsty}


\chapternumberfont{\Large}
\chaptertitlefont{\Large}
\setcounter{tocdepth}{5}
\setcounter{secnumdepth}{5}
\setlength{\footskip}{20pt}%Esto sube el número de página
\usepackage{graphics}
\usepackage{setspace} %paquete para el doble espaciado
\doublespacing %inicia el doble espaciado
 %Esto quita el punto final en la numeracion de cada seccion
\usepackage{tocloft}

\usepackage{titlesec}
\titleformat{\section}
{\Large\bfseries}{\thesection}{0.5em}{}
\titleformat{\subsection}
{\large\bfseries}{\thesubsection}{0.5em}{}
\titleformat{\subsubsection}
{\normalsize\bfseries}{\thesubsubsection}{0.5em}{}
\titleformat{\paragraph}
{\normalsize\bfseries}{\theparagraph}{0.5em}{}
\renewcommand\cftsecaftersnum{}
\renewcommand\thesection{\arabic{section}}
\renewcommand\thesubsection{\thesection.\arabic{subsection}}
\usepackage{caption}
\usepackage{fancyhdr}
\pagestyle{fancy}
\fancyhf{}
\fancyhead[R]{\thepage}
%\fancyfoot[R]{\rightmark}
%\fancyfoot[C]{Teléfono  2511-1400    /    posgrado@sep.ucr.ac.cr  /   www.sep.ucr.ac.cr}
\setlength{\headheight}{21.9pt}
\renewcommand\sectionmark[1]{%
\markright{\thesection\ #1}}
%\renewcommand{\footrulewidth}{0.4pt}


%\renewcommand{\footnoterule}{%
%  \kern -1pt
%  \hrule width \textwidth height 1pt
%  \kern 4pt
%}


%MARCA DE AGUA
%\usepackage{graphicx}
% \usepackage{fancyhdr}
%  \pagestyle{fancy}
%  \setlength\headheight{28pt}
%   \fancyhead[L]{\includegraphics[width=16cm]{marca.jpg}}
%   \fancyfoot[LE,RO]{}

\usepackage{booktabs}
\usepackage{longtable}
\usepackage{array}
\usepackage{multirow}
\usepackage{wrapfig}
%\usepackage{float}
\usepackage{colortbl}
\usepackage{pdflscape}
\usepackage{tabu}
\usepackage{threeparttable}
\usepackage{threeparttablex}
\usepackage[normalem]{ulem}
\usepackage{makecell}
\usepackage{xcolor}

\usepackage{tocloft}
\renewcommand{\cftsecleader}{\cftdotfill{\cftdotsep}}

%\renewcommand{\familydefault}{\sfdefault} %Para cambiar la fuente


%Para referenciar chunks
\usepackage{caption}
\usepackage{floatrow}
\floatsetup[figure]{capposition=top}
\floatsetup[table]{capposition=top}
\floatplacement{figure}{H}
\floatplacement{table}{H}

\DeclareNewFloatType{chunk}{placement=H, fileext=chk, name=}
\captionsetup{options=chunk}
\renewcommand{\thechunk}{Código~\arabic{chunk}}
\makeatletter
\@addtoreset{chunk}{section}
\makeatother
\usepackage{booktabs}
\usepackage{longtable}
\usepackage{array}
\usepackage{multirow}
\usepackage{wrapfig}
\usepackage{float}
\usepackage{colortbl}
\usepackage{pdflscape}
\usepackage{tabu}
\usepackage{threeparttable}
\usepackage{threeparttablex}
\usepackage[normalem]{ulem}
\usepackage{makecell}
\usepackage{xcolor}
\newlength{\cslhangindent}
\setlength{\cslhangindent}{1.5em}
\newenvironment{cslreferences}%
  {\setlength{\parindent}{0pt}%
  \everypar{\setlength{\hangindent}{\cslhangindent}}\ignorespaces}%
  {\par}

\title{UNIVERSIDAD DE COSTA RICA\\
SISTEMA DE ESTUDIOS DE POSGRADO\\
~\\
~\\
~\\}
\usepackage{etoolbox}
\makeatletter
\providecommand{\subtitle}[1]{% add subtitle to \maketitle
  \apptocmd{\@title}{\par {\large #1 \par}}{}{}
}
\makeatother
\subtitle{LA SOBREPARAMETRIZACIÓN EN EL ARIMA: UNA APLICACIÓN A DATOS
COSTARRICENCES\\
~\\
~\\
~\\
~\\
Tesis sometida a la consideración de la Comisión del Programa de
Estudios de Posgrado en Estadística para optar por el grado y título de
Maestría Académica en Estadística}
\author{\hfill\break
\hfill\break
\hfill\break
\hfill\break
\hfill\break
CÉSAR ANDRÉS GAMBOA SANABRIA B12672\\
~\\
~\\
~\\
~\\
~\\
Ciudad Universitaria Rodrigo Facio, Costa Rica\\
~\\
~\\}
\date{2020}

\begin{document}
\maketitle

\pagenumbering{gobble}
\cleardoublepage

\newpage

\addcontentsline{toc}{section}{DEDICATORIA}
\section*{DEDICATORIA}

\pagenumbering{roman}

Pendiente

\cleardoublepage

\addcontentsline{toc}{section}{AGRADECIMIENTOS}
\section*{AGRADECIMIENTOS}

También pendiente

\cleardoublepage

\begin{center}

``Esta tesis fue aceptada por la Comisión del Programa de Estudios de Posgrado en Estadística de la Universidad de Costa Rica, como requisito parcial para optar al grado y título de Maestría Académica en Estadística''

\text{}

\noindent\rule{7cm}{0.4pt}\\
Ph.D. Álvaro Morales Ramírez\\
\textbf{Decano Sistema de Estudios de Posgrado}

\text{}

\noindent\rule{7cm}{0.4pt}\\
MSc. Óscar Centeno Mora\\
\textbf{Director de Tesis}

\text{}

\noindent\rule{7cm}{0.4pt}\\
Ph.D. Gilbert Brenes Camacho\\
\textbf{Lector}

\text{}

\noindent\rule{7cm}{0.4pt}\\
Ph.D. ShuWei Chou.\\
\textbf{Lector}

\text{}

\noindent\rule{7cm}{0.4pt}\\
MSc. Johnny Madrigal Pana\\
\textbf{Director Programa de Posgrado en Estadística}

\text{}

\noindent\rule{7cm}{0.4pt}\\
César Andrés Gamboa Sanabria\\
\textbf{Candidato}

\end{center}

\cleardoublepage

\tableofcontents
\listoftables
\listoffigures

\cleardoublepage
\pagenumbering{arabic}

\newpage

\addcontentsline{toc}{section}{RESUMEN}
\section*{RESUMEN}

\cleardoublepage

\addcontentsline{toc}{section}{ABSTRACT}
\section*{ABSTRACT}

\cleardoublepage

\section{INTRODUCCIÓN}

\subsection{Antecedentes}

Estimar los valores futuros en un determinado contexto ha producido un
aumento en el análisis de los datos referidos en el tiempo, conocido
también como series cronológicas. Este tipo de datos se encuentra en
diferentes áreas, tanto en investigación académica como en el análisis
de datos para la toma de decisiones. En el campo financiero es común
hablar de la devaluación del colón con respecto al dólar, cantidad de
exportaciones mensuales de un determinado producto o las ventas
(Hernández, \protect\hyperlink{ref-oscarh-1}{2011}). Las series
cronológicas son particularmente importantes en la investigación de
mercados o en las proyecciones demográficas, que de manera conjunta
apoyan la toma de decisiones para la aprobación presupuestaria en
distintas áreas.

En la actualidad, la información temporal es muy relevante en la
actualidad: El Banco Mundial\footnote{\url{https://databank.worldbank.org/home.aspx}}
cuenta en su sitio web con datos para el análisis de series cronológicas
de indicadores de desarrollo, capacidad estadística, indicadores
educativos, estadísticas de género, nutrición y población. De manera
similar, uno de los sitios más populares relacionados con el análisis de
información es Kaggle\footnote{Se trata de una subsidiaria de la
  compañía Google que sirve de centro de reunión para todos aquellos
  interesados en la ciencia de datos.}. Kaggle, uno de los sitios más
populares relacionados con el análisis de información, ofrece una gran
cantidad de datos temporales para realizar competencias relacionadas con
las series temporales y determinar los modelos ganadores para una
determinada temática\footnote{Muchas de ellas incluyen recompensas
  económicas que van desde los \$500 hasta los \$100,000 para aquellos
  que logren obtener los mejor pronósticos.}.

Asimismo, los pronósticos (estimación futura de una partícula en una
serie temporal) son utilizados por instituciones públicas o del sector
privado, centros nacionales o regionales de investigación y
organizaciones no gubernamentales dedicadas al desarrollo social. Si las
entidades previamente mencionadas cuentan con proyecciones de calidad,
la puesta en marcha de sus respectivos planes tendrá un impacto más
efectivo.

Los métodos existentes para llevar a cabo un análisis de series
cronológicas son diversos y responden al propio contexto y tipo de
datos. Obtener buenos pronósticos o explicar el comportamiento de un
fenómeno en el tiempo siempre será un tema recurrente de investigación.
Generar una adecuada estimación es fundamental para obtener un
pronóstico de confianza, además resulta importante mencionar que los
modelos ARIMA tienen como objetivo explicar las relaciones pasadas de la
serie cronológica, para de esta manera conocer el posible comportamiento
futuro de la misma (Hyndman \& Athanasopoulos,
\protect\hyperlink{ref-hyndman2018forecasting}{2018}).

Al trabajar con la metodología de Box-Jenkins, uno de las etapas a
concretar es identificar los parámetros de estimación que gobiernan la
serie temporal. Para indagar los términos en el proceso de investigación
se ha utilizado la identificación de parámetros mediante
autocorrelogramas parciales y totales, sin embargo, los
autocorrelogramas formados no analizan de forma exhaustiva y óptima los
posibles coeficientes que podrían contemplarse la ecuación de Wold.
Según su definición matemática, esta posee infinitos coeficientes, por
tanto, se debe buscar una alternativa distinta, que opte por aproximar
de una mejor manera la identificación de los parámetros estimados,
cubriendo un mayor número de posibilidades. Esto se podría obtener
mediante un método analítico de sobreparametrización.

\subsection{La problemática}

La dificultad visual a la hora de identificar un modelo ARIMA radica en
que los autocorrelogramas solo aportan una aproximación al proceso que
gobierna la serie. De forma complementaria, es común caer en el problema
de la subjetividad, pues a pesar de que alguien proponga un patrón que
gobierne la serie, otro analista podría tener una interpretación visual
diferente del mismo proceso, proponiendo así distintas identificaciones
para un mismo proceso. Además, se posee el inconveniente de que algunos
métodos de identificación automática del proceso que gobierna la serie
subestiman el número de parámetros que se debería de contemplar.

Alternativas como la función \texttt{auto.arima()}, que ofrece el
paquete \texttt{forecast} del lenguaje de programación R\footnote{Descarga
  gratuita en \url{https://cran.r-project.org/}} (Hyndman \& Khandakar,
\protect\hyperlink{ref-auto.arima}{2008}), que permite estimar un modelo
ARIMA basado en pruebas de raíz unitaria y minimización del AICc
(Burnham \& Anderson, \protect\hyperlink{ref-burnham2007model}{2007}).
Así se obtiene un modelo temporal definiendo las diferenciaciones
requeridas en la parte estacional \(d\) mediante las pruebas KPSS (Xiao,
\protect\hyperlink{ref-doi:10.1111ux2f1467-9892.00213}{2001}) o ADF
(Fuller, \protect\hyperlink{ref-fuller1995introduction}{1995}), y la no
estacional \(D\) utilizando las pruebas OCSB (Osborn, Chui, Smith, \&
Birchenhall, \protect\hyperlink{ref-Osborn2009SEASONALITYAT}{2009}) o la
Canova-Hansen (Canova \& Hansen,
\protect\hyperlink{ref-10.2307ux2f1392184}{1995}), seleccionado el orden
óptimo para los términos \(ARIMA(p, d, q)(P, D, Q)_s\) para una serie
cronológica determinada.

Estas pruebas, sin embargo, suelen ignorar diversos términos que bien
podrían ofrecer mejores pronósticos; no someten a prueba las posibles
especificaciones de un modelo en un rango determinado, sino que realizan
aproximaciones analíticas para definir el proceso que gobierna la serie
cronológica, dejando así un vacío en el cual se corre el riesgo de no
seleccionar un modelo que ofrezca mejores pronósticos. De esta manera,
poner a prueba un mayor número de posibilidades para la especificación
de los modelos tiene la ventaja descartar unos modelos y mantener otros
con un criterio más científico y una evidencia numérica que despalde esa
decisión.

\subsection{Objetivos del estudio}

El objetivo general del presente trabajo de investigación es proponer un
nuevo algoritmo para la selección de modelos ARIMA mediante la
sobreparametrización de los términos de la ecuación del ARIMA.

Para lograr esto, se pretende:

\textbf{1.} Generar los escenarios de estimación de los distintos
modelos ARIMA mediante permutaciones de los términos \((p,d,q)\) y
\((P,D,Q)\) para la estimación de los posibles procesos que gobiernan
una determinada serie temporal.

\textbf{2.} Aplicar diversos métodos de validación en la estimación de
procesos que gobiernan la serie cronológica.

\textbf{3.} Contrastar la precisión de la estimación así como la
generación de pronósticos con otros métodos similares, aplicados en
datos costarricenses.

\textbf{4.} Integrar el desarrollo de la metodología de análisis de
series temporales en una librería del lenguaje estadístico R.

\subsection{Metodología de la investigación}

La aplicación de las series cronológicas tiene tres objetivos: 1) el
análisis exploratorio de la serie en cuestión, 2) estimar modelos de
proyección, y 3) generar pronósticos para los posibles valores futuros
que tomará el problema en cuestión. Asimismo, existen múltiples formas
de proceder mediante la etapa de estimación, como lo son los métodos de
suavizamiento exponencial (Brown, \protect\hyperlink{ref-brown}{1956}),
modelos de regresión para series temporales (Kedem \& Fokianos,
\protect\hyperlink{ref-kedem}{2005}), redes neuronales secuenciales
aplicadas a datos longitudinales (Tadayon \& Iwashita,
\protect\hyperlink{ref-redes}{2020}), estimaciones bayesianas
(Jammalamadaka, Qiu, \& Ning, \protect\hyperlink{ref-bayes}{2018}), y
finalmente, los procesos autorregresivos integrados de medias móviles o
ARIMA por sus siglas en inglés (Box, Jenkins, \& Reinsel,
\protect\hyperlink{ref-box-jenkins}{1994}), siendo estos últimos el foco
de interés en este estudio.

Un proceso ARIMA es caracterizado por dos funciones: la autocorrelación
y la autocorrelación parcial; mediante la comparación de dichas
funciones se busca la identificación del proceso que describa de manera
adecuada el comportamiento de una serie cronológica.

\subsection{Justificación del estudio}

El accionar de políticas gubernamentales, así como de otro tipo de
sectores, se apoya cada vez más en un acertado análisis de la
información temporal. En demografía, uno de los principales temas de
investigación son las proyecciones de población; durante una emergencia
conocer la posible cantidad de población que habita una zona es clave
para la rápida reacción de las autoridades en el envío de ayuda o en la
ejecución de planes de evacuación.

Asimismo, los análisis actuariales se ven beneficiados al mejorar sus
métodos de pronóstico. Una de sus principales áreas de estudio es la
mortalidad, ya que representa un insumo de vital importancia para la
planificación y sostenibilidad de los sistemas de pensiones, servicios
de salud tanto pública como privada, seguros de vida y asuntos
hipotecarios (Rosero-Bixby, \protect\hyperlink{ref-supenprodc}{2018}).

La estimación de series de tiempo es una labor común es distintos campos
de investigación con el objetivo de poder pronosticar con una buena
precisión lo que sucederá dentro de los próximos periodos. Métodos
actuales como el \texttt{auto.arima()} solamente realizan aproximaciones
analíticas no óptimas, por lo que suelen omitir procesos que
describirían de una mejor manera el comportamiento futuro de una serie
cronológica.

Estimar modelos ARIMA considerando diversas permutaciones, permite
mitigar las falencias de otras aproximaciones analíticas que no analizan
muchos de esos escenarios. El desarrollo y evaluación del método
propuesto de sobreparametrización mostrará el potencial de esta
metodología en la calidad de los pronósticos. El principal aporte de
este estudio es, brindar evidencia sobre cómo la sobreparametrización
puede contribuir a definir la especificación de un modelo ARIMA que
genere pronósticos de mayor calidad.

\subsection{Organización del estudio}

El presente trabajo de investigación consta de cinco capítulos. El
primer ofreció una contextualización del uso de las series de tiempo,
así como la importancia de poder contar con pronósticos de calidad. Se
presentó el objetivo del estudio, así como una breve descripción de la
metodología empleada en la aplicación de series temporales, y cómo se
planea modificar el método de estimación en los modelos ARIMA. Se
concluye esta sección con hechos que justifican la importancia de esta
investigación.

El siguiente capítulo consiste en el marco teórico, abarcando aspectos
fundamentales de la ecuación de Wold, la metodología Box-Jenkins, la
selección de los procesos que gobiernan la serie, la descripción del
proceso iterativo, el análisis combinatorio, entre otros.

En el tercer capítulo se describe la metodología relacionada al estudio,
iniciando con una descripción global de los conceptos más fundamentales
del análisis de series cronológicas, pasando por los componentes
fundamentales de las mismas. Se discuten también los supuestos clásicos
del análisis de series cronológicas, los distintos tipos de modelos, el
análisis de intervención, los métodos de validación y las medidas de
rendimiento; aspectos cruciales para obtener un modelo ARIMA vía
sobreparametrización. La sección metodológica culmina con la descripción
del proceso de simulación que se utilizará, así como la discusión del
método propuesto.

El capítulo cuatro consiste en la presentación de los resultados, tanto
en los datos simulados como en la aplicación a datos costarricenses y se
contrastarán contra los obtenidos por otros métodos como el de la
función \texttt{auto.arima()}, entre otros.

El último capítulo busca discutir los principales resultados, así como
señalar las conclusiones más importantes y ofrecer algunas
recomendaciones que orienten futuros estudios relacionados.

\newpage

\section{MARCO TEÓRICO}

\subsection{Introducción}

Los modelos ARIMA, son los de uso más extendido en el análisis de series
cronológicas. El nombre ARIMA es la abreviatura inglesa para
AutoRegresive Integrated Moving Average, y son aplicados mediante la
metodología de Box-Jenkins.

De esta manera, el método de Box-Jenkins inicia con el análisis
exploratorio de la serie cronológica de interés, teniendo un interés
particular en identificar si hay presencia de factores no estacionarios
en la misma. Si en efecto se cuenta con una serie no estacionaria, ésta
debe volverse estacionaria mediante algún tipo de transformación,
típicamente el logaritmo natural. Con la serie ya transformada, se busca
identificar el proceso que gobierna la serie, la forma clásica de hacer
esto es mediante los gráficos de autocorrelación y autocorrelación
parcial. Cuando se logra identificar un proceso que se adecúe más a la
serie cronológica, se deben realizar los diagnósticos para evaluar la
calidad del ajuste del modelo, así como las medidas de rendimiento
referentes a los pronósticos que genera el modelo estimado hasta un
horizonte determinado.

El método ARIMA se fundamenta en las autocorrelaciones pasadas, y
contempla un proceso iterativo para identificar un posible proceso
óptimo a partir de una clase general de modelos. El teorema de Wold
sugiere que todo proceso estacionario puede ser determinado de una forma
específica y cuya ecuación posee, en realidad, infinitos coeficientes,
pero que debe ser reducido a una cantidad finita para luego evaluar su
ajuste sometiéndolo a diferentes pruebas y medidas de rendimiento.

\subsection{Investigaciones relacionadas}

\subsection{Observaciones finales sobre la revisión bibliográfica}

\newpage

\section{METODOLOGÍA}

\subsection{Introducción}

En la búsqueda de un modelo adecuado entre varios candidatos, se llevan
a cabo comparaciones de medidas de bondad de ajuste y de precisión. Se
consideran temporalidades mensuales, bimensuales, trimestrales o
cuatrimestrales, mediante un proceso de selección fundamentada en las
permutaciones de todos los parámetros de un modelo ARIMA hasta un rango
determinado, considerando la inclusión semiautomática de intervenciones
en periodos específicos y la validación cruzada para evaluar la calidad
de las particiones de la base de datos en conjuntos para entrenar y
probar el rendimiento del modelo. Dichas pruebas sirven de insumo para
utilizar un método de consenso entre ellas y seleccionar el modelo más
adecuado mediante la sobreparametrización: se comparan todos los
posibles en in intervalo específico de términos definiendo una
diferenciación adecuada para la serie y permutando hasta un máximo
definido para los términos autorregresivos y de medias móviles
especificados para así seleccionar la especificación que ofrezca mejores
resultados al momento de pronosticar valores futuros de la serie
cronológica.

\subsection{Conceptos y definiciones en el análisis de series cronológicas}

\subsubsection{Definición de una serie cronológica}

\subsubsection{Procedimiento al analizar series cronológicas}

\subsubsection{Estacionaridad}

\subsubsection{La parsimonia}

\subsection{Componentes de una serie cronológica}

\subsubsection{La tendencia}

\subsubsection{Componentes estacionales}

\subsubsection{Componente cíclico}

\subsubsection{Componente irregular}

\subsection{Supuestos en el análisis de series cronológicas}

\subsection{Modelos de series cronológicas}

\subsection{Modelos Autorregresivos Integrados de Medias Móviles}

\subsubsection{Modelos Autorregresivos}

\subsubsection{Modelos de Medias Móviles}

\subsubsection{Metodología Box-Jenkins}

\subsubsection{Etapa 1 - Identificación}

\subsubsection{Etapa 2 - Estimación y diagnóstico}

\subsubsection{Etapa 3 - Pronóstico}

\subsubsection{Notación de los modelos ARIMA}

\subsubsection{Diferenciación}

\subsection{Análisis de intervención}

\subsection{Validación cruzada}

\subsection{Medidas de rendimiento}

\subsubsection{MFE}

\subsubsection{MAE}

\subsubsection{MAPE}

\subsubsection{MPE}

\subsubsection{MSE}

\subsubsection{SSE}

\subsubsection{SMSE}

\subsubsection{RMSE}

\subsubsection{NMSE}

\subsubsection{AIC}

\subsubsection{AICc}

\subsubsection{BIC}

\subsection{La sobreparametrización}

\subsection{Simulación de series cronológicas}

\subsection{El método propuesto}

\newpage

\section{RESULTADOS}

\subsection{Introducción}

El método propuesto se probará comparándose con los resultados de seis
series con distintas temporalidades: mortalidad infantil, mortalidad por
causa externa, nacimientos, demanda eléctrica, intereses y comisiones
del sector público e incentivos salariales del sector público.

\subsection{Datos simulados}

\subsubsection{Comparación en datos simulados - Sobreparametrización vs auto.arima}

\subsection{Estimaciones en datos costarricenses}

En el campo demográfico, por ejemplo, las estadísticas vitales son
sistematizadas y divulgadas año tras año, por tanto, revelan los cambios
acontecidos durante este periodo. Esta información junto con la
proveniente de los censos de población constituye la base para construir
los diferentes índices, tasas y otros indicadores que revelan la
situación demográfica del país, información de gran relevancia para la
planificación nacional, regional y local en diversos campos. Uno de
estos principales campos de acción es la salud pública, para la cual la
tasa de mortalidad infantil se considera uno de los indicadores
prioritarios dado que refleja no solo las condiciones de salud de la
población infante, sino también los niveles de desarrollo del país, pues
depende de la calidad de la atención de la salud, principalmente de la
prenatal y perinatal, así como de las condiciones de saneamiento. Por
tanto, su continuo monitoreo es fundamental para diseñar, implementar y
evaluar políticas de salud pública orientadas a disminuir y erradicar
aquellas que son prevenibles (INEC,
\protect\hyperlink{ref-calidad_vitales}{2017}).

\subsubsection{Tasa de mortalidad infantil interanual}

\subsubsection{Tasa global de fecundidad}

\subsubsection{Mortalidad por causa externa}

\subsubsection{Incentivos salariales del sector público}

\subsubsection{Intereses y comisiones del sector público}

\subsubsection{Demanda eléctrica}

\subsubsection{Comparación en datos reales - Sobreparametrización vs auto.arima}

\subsection{Discusión de los resultados}

\newpage

\section{CONCLUSIONES Y RECOMENDACIONES}

\subsection{Introducción}

\subsection{Conclusiones}

\subsection{Recomendaciones}

\newpage

\section{ANEXOS}

\subsection{La función funcion\_1}

\captionof{chunk}{Una función}\label{funcion1}

\begin{Shaded}
\begin{Highlighting}[]
\NormalTok{funcion\_}\DecValTok{1}\NormalTok{ \textless{}{-}}\StringTok{ }\ControlFlowTok{function}\NormalTok{(x,y)\{}
\NormalTok{    x}\OperatorTok{+}\NormalTok{y}
\NormalTok{\}}
\end{Highlighting}
\end{Shaded}

\newpage

\section{REFERENCIAS}

\hypertarget{refs}{}
\begin{cslreferences}
\leavevmode\hypertarget{ref-box-jenkins}{}%
Box, G. E. P., Jenkins, G. M., \& Reinsel, G. C. (1994). \emph{Time
Series Analysis: Forecasting and Control}. Recuperado de
\url{https://books.google.co.cr/books?id=sRzvAAAAMAAJ}

\leavevmode\hypertarget{ref-brown}{}%
Brown, R. (1956). \emph{Exponential Smoothing for Predicting Demand}.
Recuperado de \url{https://www.industrydocuments.ucsf.edu/docs/jzlc0130}

\leavevmode\hypertarget{ref-burnham2007model}{}%
Burnham, K. P., \& Anderson, D. R. (2007). \emph{Model Selection and
Multimodel Inference: A Practical Information-Theoretic Approach}.
Recuperado de \url{https://books.google.co.cr/books?id=IWUKBwAAQBAJ}

\leavevmode\hypertarget{ref-10.2307ux2f1392184}{}%
Canova, F., \& Hansen, B. E. (1995). Are Seasonal Patterns Constant over
Time? A Test for Seasonal Stability. \emph{Journal of Business \&
Economic Statistics}, \emph{13}(3), 237-252. Recuperado de
\url{http://www.jstor.org/stable/1392184}

\leavevmode\hypertarget{ref-fuller1995introduction}{}%
Fuller, W. A. (1995). \emph{Introduction to Statistical Time Series}.
Recuperado de \url{https://books.google.co.cr/books?id=wyRhjmAPQIYC}

\leavevmode\hypertarget{ref-oscarh-1}{}%
Hernández, O. (2011). \emph{Introducción a las Series Cronológicas} (1.ª
ed., p. 1). Recuperado de
\url{http://www.editorial.ucr.ac.cr/ciencias-naturales-y-exactas/item/1985-introduccion-a-las-series-cronologicas.html}

\leavevmode\hypertarget{ref-hyndman2018forecasting}{}%
Hyndman, R. J., \& Athanasopoulos, G. (2018). \emph{Forecasting:
principles and practice}. Recuperado de
\url{https://books.google.co.cr/books?id=/_bBhDwAAQBAJ}

\leavevmode\hypertarget{ref-auto.arima}{}%
Hyndman, R., \& Khandakar, Y. (2008). Automatic Time Series Forecasting:
The forecast Package for R. \emph{Journal of Statistical Software,
Articles}, \emph{27}(3), 1-22.
\url{https://doi.org/10.18637/jss.v027.i03}

\leavevmode\hypertarget{ref-calidad_vitales}{}%
INEC. (2017). \emph{Población, nacimientos, defunciones y matrimonios}.
Recuperado de
\url{http://inec.cr/sites/default/files/documetos-biblioteca-virtual/repoblacev2017_0.pdf}

\leavevmode\hypertarget{ref-bayes}{}%
Jammalamadaka, S. R., Qiu, J., \& Ning, N. (2018). \emph{Multivariate
Bayesian Structural Time Series Model}. Recuperado de
\url{https://arxiv.org/pdf/1801.03222.pdf}

\leavevmode\hypertarget{ref-kedem}{}%
Kedem, B., \& Fokianos, K. (2005). \emph{Regression Models for Time
Series Analysis}. Recuperado de
\url{https://books.google.co.cr/books?id=8r0qE35wt44C}

\leavevmode\hypertarget{ref-Osborn2009SEASONALITYAT}{}%
Osborn, D. R., Chui, A. P. L., Smith, J., \& Birchenhall, C. (2009).
\emph{Seasonality and the order of integration for consumption}.
Recuperado de
\url{http://www.est.uc3m.es/esp/nueva_docencia/comp_col_get/lade/tecnicas_prediccion/OCSB_OxBull1988.pdf}

\leavevmode\hypertarget{ref-supenprodc}{}%
Rosero-Bixby, L. (2018). \emph{Producto C para SUPEN. Proyección de la
mortalidad de Costa Rica 2015-2150}. Recuperado de CCP-UCR website:
\url{http://srv-website.cloudapp.net/documents/10179/999061/Nota+t\%C3\%A9cnica+tablas+de+vida+segunda+parte}

\leavevmode\hypertarget{ref-redes}{}%
Tadayon, M., \& Iwashita, Y. (2020). \emph{Comprehensive Analysis of
Time Series Forecasting Using Neural Networks}. Recuperado de
\url{https://arxiv.org/pdf/2001.09547.pdf}

\leavevmode\hypertarget{ref-doi:10.1111ux2f1467-9892.00213}{}%
Xiao, Z. (2001). Testing the Null Hypothesis of Stationarity Against an
Autoregressive Unit Root Alternative. \emph{Journal of Time Series
Analysis}, \emph{22}(1), 87-105.
\url{https://doi.org/10.1111/1467-9892.00213}
\end{cslreferences}

\end{document}
