% Options for packages loaded elsewhere
\PassOptionsToPackage{unicode}{hyperref}
\PassOptionsToPackage{hyphens}{url}
\PassOptionsToPackage{dvipsnames,svgnames*,x11names*}{xcolor}
%
\documentclass[
]{article}
\usepackage{lmodern}
\usepackage{amssymb,amsmath}
\usepackage{ifxetex,ifluatex}
\ifnum 0\ifxetex 1\fi\ifluatex 1\fi=0 % if pdftex
  \usepackage[T1]{fontenc}
  \usepackage[utf8]{inputenc}
  \usepackage{textcomp} % provide euro and other symbols
\else % if luatex or xetex
  \usepackage{unicode-math}
  \defaultfontfeatures{Scale=MatchLowercase}
  \defaultfontfeatures[\rmfamily]{Ligatures=TeX,Scale=1}
\fi
% Use upquote if available, for straight quotes in verbatim environments
\IfFileExists{upquote.sty}{\usepackage{upquote}}{}
\IfFileExists{microtype.sty}{% use microtype if available
  \usepackage[]{microtype}
  \UseMicrotypeSet[protrusion]{basicmath} % disable protrusion for tt fonts
}{}
\makeatletter
\@ifundefined{KOMAClassName}{% if non-KOMA class
  \IfFileExists{parskip.sty}{%
    \usepackage{parskip}
  }{% else
    \setlength{\parindent}{0pt}
    \setlength{\parskip}{6pt plus 2pt minus 1pt}}
}{% if KOMA class
  \KOMAoptions{parskip=half}}
\makeatother
\usepackage{xcolor}
\IfFileExists{xurl.sty}{\usepackage{xurl}}{} % add URL line breaks if available
\IfFileExists{bookmark.sty}{\usepackage{bookmark}}{\usepackage{hyperref}}
\hypersetup{
  colorlinks=true,
  linkcolor=blue,
  filecolor=Maroon,
  citecolor=Blue,
  urlcolor=blue,
  pdfcreator={LaTeX via pandoc}}
\urlstyle{same} % disable monospaced font for URLs
\usepackage[margin=1in]{geometry}
\usepackage{color}
\usepackage{fancyvrb}
\newcommand{\VerbBar}{|}
\newcommand{\VERB}{\Verb[commandchars=\\\{\}]}
\DefineVerbatimEnvironment{Highlighting}{Verbatim}{commandchars=\\\{\}}
% Add ',fontsize=\small' for more characters per line
\usepackage{framed}
\definecolor{shadecolor}{RGB}{248,248,248}
\newenvironment{Shaded}{\begin{snugshade}}{\end{snugshade}}
\newcommand{\AlertTok}[1]{\textcolor[rgb]{0.94,0.16,0.16}{#1}}
\newcommand{\AnnotationTok}[1]{\textcolor[rgb]{0.56,0.35,0.01}{\textbf{\textit{#1}}}}
\newcommand{\AttributeTok}[1]{\textcolor[rgb]{0.77,0.63,0.00}{#1}}
\newcommand{\BaseNTok}[1]{\textcolor[rgb]{0.00,0.00,0.81}{#1}}
\newcommand{\BuiltInTok}[1]{#1}
\newcommand{\CharTok}[1]{\textcolor[rgb]{0.31,0.60,0.02}{#1}}
\newcommand{\CommentTok}[1]{\textcolor[rgb]{0.56,0.35,0.01}{\textit{#1}}}
\newcommand{\CommentVarTok}[1]{\textcolor[rgb]{0.56,0.35,0.01}{\textbf{\textit{#1}}}}
\newcommand{\ConstantTok}[1]{\textcolor[rgb]{0.00,0.00,0.00}{#1}}
\newcommand{\ControlFlowTok}[1]{\textcolor[rgb]{0.13,0.29,0.53}{\textbf{#1}}}
\newcommand{\DataTypeTok}[1]{\textcolor[rgb]{0.13,0.29,0.53}{#1}}
\newcommand{\DecValTok}[1]{\textcolor[rgb]{0.00,0.00,0.81}{#1}}
\newcommand{\DocumentationTok}[1]{\textcolor[rgb]{0.56,0.35,0.01}{\textbf{\textit{#1}}}}
\newcommand{\ErrorTok}[1]{\textcolor[rgb]{0.64,0.00,0.00}{\textbf{#1}}}
\newcommand{\ExtensionTok}[1]{#1}
\newcommand{\FloatTok}[1]{\textcolor[rgb]{0.00,0.00,0.81}{#1}}
\newcommand{\FunctionTok}[1]{\textcolor[rgb]{0.00,0.00,0.00}{#1}}
\newcommand{\ImportTok}[1]{#1}
\newcommand{\InformationTok}[1]{\textcolor[rgb]{0.56,0.35,0.01}{\textbf{\textit{#1}}}}
\newcommand{\KeywordTok}[1]{\textcolor[rgb]{0.13,0.29,0.53}{\textbf{#1}}}
\newcommand{\NormalTok}[1]{#1}
\newcommand{\OperatorTok}[1]{\textcolor[rgb]{0.81,0.36,0.00}{\textbf{#1}}}
\newcommand{\OtherTok}[1]{\textcolor[rgb]{0.56,0.35,0.01}{#1}}
\newcommand{\PreprocessorTok}[1]{\textcolor[rgb]{0.56,0.35,0.01}{\textit{#1}}}
\newcommand{\RegionMarkerTok}[1]{#1}
\newcommand{\SpecialCharTok}[1]{\textcolor[rgb]{0.00,0.00,0.00}{#1}}
\newcommand{\SpecialStringTok}[1]{\textcolor[rgb]{0.31,0.60,0.02}{#1}}
\newcommand{\StringTok}[1]{\textcolor[rgb]{0.31,0.60,0.02}{#1}}
\newcommand{\VariableTok}[1]{\textcolor[rgb]{0.00,0.00,0.00}{#1}}
\newcommand{\VerbatimStringTok}[1]{\textcolor[rgb]{0.31,0.60,0.02}{#1}}
\newcommand{\WarningTok}[1]{\textcolor[rgb]{0.56,0.35,0.01}{\textbf{\textit{#1}}}}
\usepackage{graphicx,grffile}
\makeatletter
\def\maxwidth{\ifdim\Gin@nat@width>\linewidth\linewidth\else\Gin@nat@width\fi}
\def\maxheight{\ifdim\Gin@nat@height>\textheight\textheight\else\Gin@nat@height\fi}
\makeatother
% Scale images if necessary, so that they will not overflow the page
% margins by default, and it is still possible to overwrite the defaults
% using explicit options in \includegraphics[width, height, ...]{}
\setkeys{Gin}{width=\maxwidth,height=\maxheight,keepaspectratio}
% Set default figure placement to htbp
\makeatletter
\def\fps@figure{htbp}
\makeatother
\setlength{\emergencystretch}{3em} % prevent overfull lines
\providecommand{\tightlist}{%
  \setlength{\itemsep}{0pt}\setlength{\parskip}{0pt}}
\setcounter{secnumdepth}{-\maxdimen} % remove section numbering


%\usepackage{fancyhdr}
%\pagestyle{fancy}
%\rhead{\includegraphics[width = 1\textwidth]{marca.jpg}}


\usepackage{geometry}
\geometry{a4paper, left=35mm, right=25mm, bottom=15mm}
\usepackage{setspace}
\doublespacing
\usepackage[spanish]{babel}
\usepackage{color}
\usepackage{xcolor}
\usepackage{framed}
\colorlet{shadecolor}{gray!20}
\setcounter{secnumdepth}{0}
\usepackage{sectsty}


\chapternumberfont{\Large}
\chaptertitlefont{\Large}
\setcounter{tocdepth}{5}
\setcounter{secnumdepth}{5}
\setlength{\footskip}{20pt}%Esto sube el número de página
\usepackage{graphics}
\usepackage{setspace} %paquete para el doble espaciado
\doublespacing %inicia el doble espaciado
 %Esto quita el punto final en la numeracion de cada seccion
\usepackage{tocloft}

\usepackage{titlesec}
\titleformat{\section}
{\Large\bfseries}{\thesection}{0.5em}{}
\titleformat{\subsection}
{\large\bfseries}{\thesubsection}{0.5em}{}
\titleformat{\subsubsection}
{\normalsize\bfseries}{\thesubsubsection}{0.5em}{}
\titleformat{\paragraph}
{\normalsize\bfseries}{\theparagraph}{0.5em}{}
\renewcommand\cftsecaftersnum{}
\renewcommand\thesection{\arabic{section}}
\renewcommand\thesubsection{\thesection.\arabic{subsection}}
\usepackage{caption}
\usepackage{fancyhdr}
\pagestyle{fancy}
\fancyhf{}
\fancyhead[R]{\thepage}
%\fancyfoot[R]{\rightmark}
%\fancyfoot[C]{Teléfono  2511-1400    /    posgrado@sep.ucr.ac.cr  /   www.sep.ucr.ac.cr}
\setlength{\headheight}{21.9pt}
\renewcommand\sectionmark[1]{%
\markright{\thesection\ #1}}
%\renewcommand{\footrulewidth}{0.4pt}


%\renewcommand{\footnoterule}{%
%  \kern -1pt
%  \hrule width \textwidth height 1pt
%  \kern 4pt
%}


%MARCA DE AGUA
%\usepackage{graphicx}
% \usepackage{fancyhdr}
%  \pagestyle{fancy}
%  \setlength\headheight{28pt}
%   \fancyhead[L]{\includegraphics[width=16cm]{marca.jpg}}
%   \fancyfoot[LE,RO]{}

\usepackage{booktabs}
\usepackage{longtable}
\usepackage{array}
\usepackage{multirow}
\usepackage{wrapfig}
%\usepackage{float}
\usepackage{colortbl}
\usepackage{pdflscape}
\usepackage{tabu}
\usepackage{threeparttable}
\usepackage{threeparttablex}
\usepackage[normalem]{ulem}
\usepackage{makecell}
\usepackage{xcolor}

\usepackage{tocloft}
\renewcommand{\cftsecleader}{\cftdotfill{\cftdotsep}}

%\renewcommand{\familydefault}{\sfdefault} %Para cambiar la fuente


%Para referenciar chunks
\usepackage{caption}
\usepackage{floatrow}
\floatsetup[figure]{capposition=top}
\floatsetup[table]{capposition=top}
\floatplacement{figure}{H}
\floatplacement{table}{H}

\DeclareNewFloatType{chunk}{placement=H, fileext=chk, name=}
\captionsetup{options=chunk}
\renewcommand{\thechunk}{Código~\arabic{chunk}}
\makeatletter
\@addtoreset{chunk}{section}
\makeatother
\usepackage{booktabs}
\usepackage{longtable}
\usepackage{array}
\usepackage{multirow}
\usepackage{wrapfig}
\usepackage{float}
\usepackage{colortbl}
\usepackage{pdflscape}
\usepackage{tabu}
\usepackage{threeparttable}
\usepackage{threeparttablex}
\usepackage[normalem]{ulem}
\usepackage{makecell}
\usepackage{xcolor}

\title{UNIVERSIDAD DE COSTA RICA\\
SISTEMA DE ESTUDIOS DE POSGRADO\\
~\\
~\\
~\\}
\usepackage{etoolbox}
\makeatletter
\providecommand{\subtitle}[1]{% add subtitle to \maketitle
  \apptocmd{\@title}{\par {\large #1 \par}}{}{}
}
\makeatother
\subtitle{LA SOBREPARAMETRIZACIÓN EN EL ARIMA: UNA APLICACIÓN A DATOS
COSTARRICENCES\\
~\\
~\\
~\\
~\\
Tesis sometida a la consideración de la Comisión del Programa de
Estudios de Posgrado en Estadística para optar por el grado y título de
Maestría Académica en Estadística}
\author{CÉSAR ANDRÉS GAMBOA SANABRIA B12672\\
~\\
~\\
~\\
~\\
~\\
Ciudad Universitaria Rodrigo Facio, Costa Rica\\
~\\
~\\}
\date{2020}

\begin{document}
\maketitle

\pagenumbering{gobble}
\cleardoublepage

\newpage

\addcontentsline{toc}{section}{DEDICATORIA}
\section*{DEDICATORIA}

\pagenumbering{roman}

Pendiente

\cleardoublepage

\addcontentsline{toc}{section}{AGRADECIMIENTOS}
\section*{AGRADECIMIENTOS}

También pendiente

\cleardoublepage

\begin{center}

``Esta tesis fue aceptada por la Comisión del Programa de Estudios de Posgrado en Estadística de la Universidad de Costa Rica, como requisito parcial para optar al grado y título de Maestría Académica en Estadística''

\text{}

\noindent\rule{7cm}{0.4pt}\\
Ph.D. Álvaro Morales Ramírez\\
\textbf{Decano Sistema de Estudios de Posgrado}

\text{}

\noindent\rule{7cm}{0.4pt}\\
MSc. Óscar Centeno Mora\\
\textbf{Director de Tesis}

\text{}

\noindent\rule{7cm}{0.4pt}\\
Ph.D. Gilbert Brenes Camacho\\
\textbf{Lector}

\text{}

\noindent\rule{7cm}{0.4pt}\\
Ph.D. ShuWei Chou.\\
\textbf{Lector}

\text{}

\noindent\rule{7cm}{0.4pt}\\
MSc. Johnny Madrigal Pana\\
\textbf{Director Programa de Posgrado en Estadística}

\text{}

\noindent\rule{7cm}{0.4pt}\\
César Andrés Gamboa Sanabria\\
\textbf{Candidato}

\end{center}

\cleardoublepage

\tableofcontents
\listoftables
\listoffigures

\cleardoublepage
\pagenumbering{arabic}

\newpage

\addcontentsline{toc}{section}{RESUMEN}
\section*{RESUMEN}

\cleardoublepage

\addcontentsline{toc}{section}{ABSTRACT}
\section*{ABSTRACT}

\cleardoublepage

\section{INTRODUCCIÓN}

\subsection{Antecedentes}

Conocer tanto el estado pasado, actual y futuro tanto de un mundo y una
economía tan volatil ha producido un aumento primordial en el análisis
de los datos refereridos en el tiempo. El manejo de información obtenida
de manera secuencial hace referencia al uso de series cronológicas. Este
tipo de datos se encuentra en diferentes áreas, tanto en investigación
académica como en el análisis de datos para la toma de deciciones. En el
campo financiero es común hablar de la devaluación del colón con
respecto al dólar, cantidad de exportaciones mensuales de un determinado
producto o las ventas de este (Hernández,
\protect\hyperlink{ref-oscarh-1}{2011}). El estudio de las series
cronológicas posee una particular importancia en el análisis de datos en
la investigación de mercados bancarios y proyecciones demográficas que
de manera conjunta apoyan la toma de decisiones para la aprobación
presupuestaria en distintas áreas.

El \href{https://databank.worldbank.org/home.aspx}{Banco
Mundial}\footnote{\url{https://databank.worldbank.org/home.aspx}} cuenta
en su sitio web con datos para el análisis de series cronológicas de
indicadores de desarrollo, capacidad estadística, indicadores
educativos, estadísticas de género, nutrición y población. De manera
similar al Banco Mundial, uno de los sitios más populares relacionados
con el análisis de información es
\href{https://www.kaggle.com/}{Kaggle}\footnote{Se trata de una
  susidiaria de la compañía Google que sirve de centro de reunión para
  todos aquellos interesados en la ciencia de datos.}. En su sitio web,
Kaggle ofrece una gran cantidad de conjuntos de datos para poner a
prueba distintas formas de análisis o bien, para participar en
competencias. Actualmente, para el análisis de series cronológicas,
Kaggle cuenta con más de 30 competiciones\footnote{Muchas de ellas
  incluyen recompensas económicas que van desde los \$500 hasta los
  \$100,000 para aquellos que logren obtener los mejor pronósticos.}.

Cuando se ha establecido un modelo, los pronósticos son utilizados en
instituciones públicas, gobiernos municipales, instituciones del sector
privado, centros académicos, población civil, centros nacionales o
regionales de investigación y organizaciones no gubernamentales
dedicadas al desarrollo social. Si las entidades previamente mencionadas
cuentan con proyecciones de calidad, la puesta en marcha de sus
respectivos planes tendrá un impacto mayor y más efectivo.

La cantidad de métodos existentes para realizar un análisis de series
cronológicas son diversos y responden al propio contexto y tipo de
datos, razón por la cual en la actualidad obtener buenos pronósticos o
bien explicar el comportamiento de un fenómeno en el tiempo si siendo un
tema recurrente de investigación Generar un modelo adecuado es
fundamental para obtener un pronóstico de calidad y resulta importante
mencionar una diferencia clave entre los dos modelos clásicos más
utilizados: los modelos de suavizamiento exponencial y los modelos
ARIMA. Ambos representan enfoques complementarios a un problema, pues
los modelos de suavizamiento exponencial se fundamentan en un enfoque
más descriptivo de los componentes de la serie cronológica en estudio,
mientras que los modelos ARIMA tienen como objetivo explicar las
relaciones pasadas de ésta (R. J. Hyndman \& Athanasopoulos,
\protect\hyperlink{ref-hyndman2018forecasting}{2018}\protect\hyperlink{ref-hyndman2018forecasting}{a}).

Este tipo de métodos que estudian las autocorrelaciones pasadas no
supone ningún patrón en particular en la serie histórica que se busca
pronosticar, sino que contempla un proceso iterativo para identificar un
posible modelo a partir de una clase general de modelos, El teorema de
Wold sugiere que todo proceso estacionario puede ser determinado de una
forma específica y cuya ecuación posee, en realidad, infinitos
coeficientes, pero que debe ser reducido a una cantidad finita para
luego evaluar su ajuste sometiéndolo a diferentes pruebas y medidas de
rendimiento.

Al trabajar con la metodología de Box-Jenkins, uno de los pasos a
concretar es identificar los parámetros autoregresivos y de medias
móviles que gobiernan la serie. En la actualidad, para estudiar este
tipo de modelos se ha utilizado la identificación de parámetros mediante
autocorrelogramas parciales y totales, sin embargo, estos
autocorrelogramas no abarcan muchos de los coeficientes que contempla la
mencionada ecuación de Wold, pues como se mencionó, esta posee infinitos
coeficientes, razón por la cual el método propuesto busca aproximar de
una mejor manera la identificación del modelo cubriendo un mayor número
de posibilidades de dichos parámetros mediante lo que suele considerarse
un problema: la sobreparametrización.

\subsection{El problema}

La dificultad visual a la hora de identificar un modelo ARIMA es que
actualmente es común disponer de una gran cantidad de series
cronológicas para analizar, y los autocorrelogramas solo aportan una
aproximación al proceso que gobierna la serie, representando en muchos
casos una pobre alternativa dado que subestiman el número de parámetros
que se debería de contemplar. Aunado a esto siempre existirá el problema
de la subjetividad, pues a pesar de que alguien proponga un patrón que
gobierne la serie, otra persona podría tener otra interpretación del
proceso, porponiendo así una especificación diferente. Esto sucede
incluso con cantidades moderadas de series cronológicas a analizar, por
lo que la generación de algoritmos que ayuden a esta identificación se
vuelve cada vez más necesaria (Hyndman \& Khandakar,
\protect\hyperlink{ref-auto.arima}{2008}), siendo estos métodos una
valiosa alternativa ante los muchas veces pobres resultados en la
identificación correcta del modelo.

Han sido varias las aproximaciones a un método que genere de manera
automática un modelo ARIMA, siendo uno de los métodos automatizados de
estimación más populares es el que ofrece el paquete \texttt{forecast}
(Hyndman \& Khandakar, \protect\hyperlink{ref-auto.arima}{2008}) del
lenguaje de programación R\footnote{Descarga gratuita en
  \url{https://cran.r-project.org/}} y que permite hacer uso de la
función \texttt{auto.arima()} para estimar un modelo ARIMA basado en
pruebas de raíz unitaria y minimización del AICc (Burnham \& Anderson,
\protect\hyperlink{ref-burnham2007model}{2007}). Así se obtiene un
modelo temporal definiendo las diferenciaciones requeridas en la parte
estacional \(d\) mediante las pruebas KPSS (Xiao,
\protect\hyperlink{ref-doi:10.1111ux2f1467-9892.00213}{2001}) o ADF
(Fuller, \protect\hyperlink{ref-fuller1995introduction}{1995}), y la no
estacional \(D\) utilizando las pruebas OCSB (Osborn, Chui, Smith, \&
Birchenhall, \protect\hyperlink{ref-Osborn2009SEASONALITYAT}{2009}) o la
Canova-Hansen (Canova \& Hansen,
\protect\hyperlink{ref-10.2307ux2f1392184}{1995}), seleccionado el orden
óptimo para los términos \(ARIMA(p, d, q)(P, D, Q)_s\) para una serie
cronológica determinada. Sin embargo, alternativas como la función
\texttt{auto.arima()} no someten a prueba las posibles especificaciones
de un modelo en un rango determinado, concepto conocido como
sobreparametrización, dejando así un vacío en el cuál se corre el riesgo
de no seleccionar un modelo que ofrezca mejores pronósticos.

\subsection{Objetivos del estudio}

Esta investigación tiene entonces como objetivo general diseñar un
algoritmo para la selección de modelos ARIMA según la temporalidad de la
serie vía sobreparametrización. Para lograr esto, se busca de manera
específica alcanzar lo siguiente:

\textbf{1.} Aplicar validación cruzada en distintos horizontes de
pronóstico para identificar la mejor especificación de un modelo ARIMA.

\textbf{2.} Comparar la precisión de los pronósticos con métodos
similares, como el propuesto por Rob Hyndman, de la Oficina de Censos de
los Estados Unidos, entre otros, aplicados en datos costarricences.

\textbf{3.} Integrar el desarrollo de la metodología de análisis de
series temporales en una librería del lenguaje estadístico R.

\subsection{Metodología de la investigación}

Las series cronológicas representan un insumo para abordar, como mínimo,
tres objetivos básicos: 1) realizar análisis exploratorios usando
mediante métodos de visualización y medidas de posición y variabilidad,
como ver su crecimiento o decrecimiento a lo largo del tiempo, detectar
valores atípicos o cambios drásticos en el nivel o valor medio de la
serie, 2) generar modelos estadísticos que sirvan como una
simplificación de la realidad, y 3) generar pronósticos para los
posibles valores futuros que tomará el problema en cuestión. Los tres
objetivos anteriores se trabajan de manera conjunta, pues es necesario
realizar primero el análisis exploratorio de los datos para tener una
noción global del panorama y así conocer la serie cronológica con la que
se está trabajando. Una vez hecho esto, existen múltiples formas de
generar modelos para estos datos, como por ejemplo los métodos de
suavizamiento exponencial desarrollados en la década de 1950 (Brown,
\protect\hyperlink{ref-brown}{1956}), modelos de regresión para series
temporales (Kedem \& Fokianos, \protect\hyperlink{ref-kedem}{2005}),
redes neuronales para la estimación de pronósticos (Tadayon \& Iwashita,
\protect\hyperlink{ref-redes}{2020}), estimaciones bayesianas
(Jammalamadaka, Qiu, \& Ning, \protect\hyperlink{ref-bayes}{2018}) o los
procesos autorregresivos integrados de medias móviles o ARIMA por sus
siglas en inglés (Box, Jenkins, \& Reinsel,
\protect\hyperlink{ref-box-jenkins}{1994}), siendo estos últimos el foco
de interés en este estudio pues se trata de modelos que se basan en las
relaciones pasadas de la propia serie cronológica; es decir, toman como
referencia las correlaciones entre los valores actuales y pasados de la
serie para entender el comportamiento de la misma en el futuro.

Como menciona Rob. Hyndman (R. J. Hyndman \& Athanasopoulos,
\protect\hyperlink{ref-hyndman_box-jenkins}{2018}\protect\hyperlink{ref-hyndman_box-jenkins}{b}),
la metodología de Box-Jenkins difiere a los demás métodos porque no
supone un determinado patrón en la serie cronológica, sino que parte de
un proceso iterativo para identificar el modelo de un gran grupo de
estos para luego ponerlo a prueba según varias medidas de rendimiento.
Un proceso ARIMA es caracterizado por dos funciones: la autocorrelación
y la autocorrelación parcial; es mediante la comparación de dichas
funciones que la metodologóa Box-Jenkins busca la identifiación el
proceso que describa de manera adecuada el comportamiento de una serie
cronológica.

Con tal de mejorar la precisión y calidad del modelo estimado es que el
presente trabajo propone una metodología para la estimación un modelo
ARIMA de una serie cronológica determinada abarcando más posibilidades
que los enfoques tradicionales. Se condirearán temporalidades mensuales,
bimensuales, trimestrales o cuatrimestrales, mediante un proceso de
selección fundamentada en las permutaciones de todos los parámetros de
un modelo ARIMA hasta en un rango determinado, considerando la inclusión
semiautomática de intervenciones en periodos específicos y la validación
cruzada para evaluar la calidad de las particiones de la base de datos
en conjuntos para entrenar y probar el rendimiento del modelo. Dichas
pruebas involucran criterios de información como el AIC, el AICc y el
BIC, además de medidas de rendimiento como el MAE, RMSE, MAPE y MASE,
las cuales sirven de insumo para utilizar un método de consenso entre
ellas y seleccionar el modelo más adecuado mediante la
sobreparametrización: se comparan todos los posibles en in intervalo
específico de términos definiendo una diferenciación adecuada para la
serie y permutando hasta un máximo definido para los términos
autoregresivos y de medias móviles especificados para así seleccionar la
especificación que ofrezca mejores resultados al momento de pronosticar
valores futuros de la serie cronológica.

\subsection{Justificación del estudio}

El accionar de politicas tanto gubernamentales así como de otro tipo de
sectores se apoya cada vez más en un acertado análisis de la información
temporal de los hechos observados hasta una fecha determinada, y por
ende sus posibles evoluciones en el futuro. Por ejemplo, en la
demografía, uno de los principales temas de investigación son las
proyecciones de población; durante una emergencia conocer la posible
cantidad de población que habita una zona es clave para la rápida
reacción de las autoridades en el envío de ayuda o en la ejecución de
planes de evacuación; este tipo de situaciones se dan en Costa Rica y en
muchos lugares del mundo ante condiciones climáticas adversas como
huracanes, inundaciones, o más recientemente ante la pandemia del
COVID-19. También en el campo actuarial se ve beneficiado al mejorar sus
métodos de pronóstico, pues una de sus principales áreas de estudio es
la mortalidad, ya que representa un insumo de vital importancia para la
planificación y sostenibilidad de los sistemas de pensiones, servicios
de salud tanto pública como privada, seguros de vida y asuntos
hipotecarios (Rosero-Bixby, \protect\hyperlink{ref-supenprodc}{2018}).

El principal aporte de este estudio es, por medio de un proceso de
simulación, brindar evidencia sobre cómo la sobreparametrización puede
contribuir a definir la especificación de un modelo ARIMA que genere
pronósticos de amyor calidad, contrastando la calidad de estos con
respecto a otros métodos similares, como lo son las funciones
\texttt{auto.arima()} o \texttt{seas()}.

Tras desarrollar y probar el método mediante datos simulados, la
aplicación real del algoritmo mostrará el potencial de la
sobreparametrización en la estimación de modelos ARIMA. En el campo
demográfico, por ejemplo, las estadísticas vitales son sistematizadas y
divulgadas año tras año, por tanto, revelan los cambios acontecidos
durante este periodo. Esta información junto con la proveniente de los
censos de población constituye la base para construir los diferentes
índices, tasas y otros indicadores que revelan la situación demográfica
del país, información de gran relevancia para la planificación nacional,
regional y local en diversos campos. Uno de estos principales campos de
acción es la salud pública, para la cual la tasa de mortalidad infantil
se considera uno de los indicadores prioritarios dado que refleja no
solo las condiciones de salud de la población infante, sino también los
niveles de desarrollo del país, pues depende de la calidad de la
atención de la salud, principalmente de la prenatal y perinatal, así
como de las condiciones de saneamiento. Por tanto, su continuo monitoreo
es fundamental para diseñar, implementar y evaluar políticas de salud
pública orientadas a disminuir y erradicar aquellas que son prevenibles
(INEC, \protect\hyperlink{ref-calidad_vitales}{2017}).

\subsection{Organización del estudio}

EL presenta trabajo de investigación consta de cinco capítulos, de los
cuáles el primero ofrece una contextualización del uso de las series de
tiempo, así como la importancia de poder contar con pronósticos de
calidad. Se presentaron además objetivos que busca alcanzar el estudio
así como una breve descripción de la metodología que se empleará y que
será discutida más detalladamente en los capítulos posteriores. Se
concluye esta sección con hechos que justifican la importancia de esta
investigación.

El siguiente capítulo consiste en el marco teórico y la revisión de la
literatura asociada con el estudio, abarcando aspectos fundamentales
como la metodología Box-Jenkins, la descripción del proceso iterativo,
entre otros.

En el tercer capítulo se describe en detalle toda la metodología
relacionada al estudio, iniciando con una descripción global de los
conceptos más fundamentales del análisis de series cronológicas, pasando
por los componentes fundamentales de las mismas: tendencia,
estacionalidad, ciclos e irregularidades. En este capítulo se discuten
también los supuestos clásicos del análisis de series cronológicas, los
distintos tipos de modelos, el análisis de intervención, la validación
cruzada y las medidas de rendimiento; aspectos cruciales para obtener un
modelo ARIMA vía sobreparametrización. La sección metodológica culmina
con la descripción del proceso de simulación que se utilizará, así como
la discusión del método propuesto.

El scapítulo cuatro consiste trata la presentación de los resultados,
tanto en los datos simulados como en la aplicación a datos
costarricenses y se contrastarán contra los obtenidos por otros métodos
como el de la función \texttt{auto.arima()}.

El último capítulo busca discutir los principales resultados, así como
señalar las conclusiones más importantes y ofrecer algunas
recomendaciones que orienten futuros estudios relacionados.

Finalmente se ofrece en la sección de anexos los códigos en lenguaje R
más importantes utilizados en el estudio, así como una sección dedicada
a las referencias bilbiográficas obtenidas para esta investigación.

\newpage

\section{MARCO TEÓRICO}

\subsection{Introducción}

Los modelos ARIMA, son los de uso más extendido en el análisis de series
cronológicas. El nombre ARIMA es la abreviatura inglesa para
AutoRegresive Integrated Moving Average, y son aplicados mediante la
metodología de Box-Jenkins.

De esta manera, el método de Box-Jenkins inicia con el análisis
exploratorio de la serie cronológica de interés, teniendo un interés
particular en identificar si hay presencia de factores no estacionarios
en la misma. Si en efecto se cuenta con una serie no estacionaria, ésta
debe volverse estacionaria mediante algún tipo de transformación,
típicamente el logaritmo natural. Con la serie ya transformada, se busca
identificar el proceso que gobierna la serie, la forma clásica de hacer
esto es mediante los gráficos de autocorrelación y autocorrelación
parcial. Cuando se logra identificar un proceso que se adecúe más a la
serie cronológica, se deben realizar los diagnósticos para evaluar la
calidad del ajuste del modelo, así como las medidas de rendimiento
referentes a los pronósticos que genera el modelo estimado hasta un
horizonte determinado.

\subsection{Investigaciones relacionadas}

\subsection{Observaciones finales sobre la revisión bibliográfica}

\newpage

\section{METODOLOGÍA}

\subsection{Introducción}

\subsection{Conceptos y definiciones en el análisis de series cronológicas}

\subsubsection{Definición de una serie cronológica}

\subsubsection{Procedimiento al analizar series cronológicas}

\subsubsection{Estacionaridad}

\subsubsection{La parsimonia}

\subsection{Componentes de una serie cronológica}

\subsubsection{La tendencia}

\subsubsection{Componentes estacionales}

\subsubsection{Componente cíclico}

\subsubsection{Componente irregular}

\subsection{Supuestos en el análisis de series cronológicas}

\subsection{Modelos de series cronológicas}

\subsection{Modelos Autorregresivos Integrados de Medias Móviles}

\subsubsection{Modelos Autorregresivos}

\subsubsection{Modelos de Medias Móviles}

\subsubsection{Metodología Box-Jenkins}

\subsubsection{Etapa 1 - Identificación}

\subsubsection{Etapa 2 - Estimación y diagnóstico}

\subsubsection{Etapa 3 - Pronóstico}

\subsubsection{Notación de los modelos ARIMA}

\subsubsection{Diferenciación}

\subsection{Análisis de intervención}

\subsection{Validación cruzada}

\subsection{Medidas de rendimiento}

\subsubsection{MFE}

\subsubsection{MAE}

\subsubsection{MAPE}

\subsubsection{MPE}

\subsubsection{MSE}

\subsubsection{SSE}

\subsubsection{SMSE}

\subsubsection{RMSE}

\subsubsection{NMSE}

\subsubsection{AIC}

\subsubsection{AICc}

\subsubsection{BIC}

\subsection{La sobreparametrización}

\subsection{Simulación de series cronológicas}

\subsection{El método propuesto}

\newpage

\section{RESULTADOS}

\subsection{Introducción}

El método propuesto se probará comparándose con los resultados de seis
series con distintas temporalidades: mortalidad infantil, mortalidad por
causa externa, nacimientos, demanda eléctrica, intereses y comisiones
del sector público e incentivos salariales del sector público.

\subsection{Datos simulados}

\subsubsection{Comparación en datos simulados - Sobreparametrización vs auto.arima}

\subsection{Estimaciones en datos costarricenses}

\subsubsection{Tasa de mortalidad infantil interanual}

\subsubsection{Tasa global de fecundidad}

\subsubsection{Mortalidad por causa externa}

\subsubsection{Incentivos salariales del sector público}

\subsubsection{Intereses y comisiones del sector público}

\subsubsection{Demanda eléctrica}

\subsubsection{Comparación en datos reales - Sobreparametrización vs auto.arima}

\subsection{Discusión de los resultados}

\newpage

\section{CONCLUSIONES Y RECOMENDACIONES}

\subsection{Introducción}

\subsection{Conclusiones}

\subsection{Recomendaciones}

\newpage

\section{ANEXOS}

\subsection{La función funcion\_1}

\captionof{chunk}{Una función}\label{funcion1}

\begin{Shaded}
\begin{Highlighting}[]
\NormalTok{funcion_}\DecValTok{1}\NormalTok{ <-}\StringTok{ }\ControlFlowTok{function}\NormalTok{(x,y)\{}
\NormalTok{    x}\OperatorTok{+}\NormalTok{y}
\NormalTok{\}}
\end{Highlighting}
\end{Shaded}

\newpage

\section{REFERENCIAS}

\hypertarget{refs}{}
\leavevmode\hypertarget{ref-box-jenkins}{}%
Box, G. E., Jenkins, G. M., \& Reinsel, G. C. (1994). \emph{Time Series
Analysis: Forecasting and Control}. Recuperado de
\url{https://books.google.co.cr/books?id=sRzvAAAAMAAJ}

\leavevmode\hypertarget{ref-brown}{}%
Brown, R. (1956). \emph{Exponential Smoothing for Predicting Demand}.
Recuperado de \url{https://www.industrydocuments.ucsf.edu/docs/jzlc0130}

\leavevmode\hypertarget{ref-burnham2007model}{}%
Burnham, K. P., \& Anderson, D. R. (2007). \emph{Model Selection and
Multimodel Inference: A Practical Information-Theoretic Approach}.
Recuperado de \url{https://books.google.co.cr/books?id=IWUKBwAAQBAJ}

\leavevmode\hypertarget{ref-10.2307ux2f1392184}{}%
Canova, F., \& Hansen, B. E. (1995). Are Seasonal Patterns Constant over
Time? A Test for Seasonal Stability. \emph{Journal of Business \&
Economic Statistics}, \emph{13}(3), 237-252. Recuperado de
\url{http://www.jstor.org/stable/1392184}

\leavevmode\hypertarget{ref-fuller1995introduction}{}%
Fuller, W. A. (1995). \emph{Introduction to Statistical Time Series}.
Recuperado de \url{https://books.google.co.cr/books?id=wyRhjmAPQIYC}

\leavevmode\hypertarget{ref-oscarh-1}{}%
Hernández, O. (2011). \emph{Introducción a las Series Cronológicas} (1.ª
ed., p. 1). Recuperado de
\url{http://www.editorial.ucr.ac.cr/ciencias-naturales-y-exactas/item/1985-introduccion-a-las-series-cronologicas.html}

\leavevmode\hypertarget{ref-hyndman2018forecasting}{}%
Hyndman, R. J., \& Athanasopoulos, G. (2018a). \emph{Forecasting:
principles and practice}. Recuperado de
\url{https://books.google.co.cr/books?id=/_bBhDwAAQBAJ}

\leavevmode\hypertarget{ref-hyndman_box-jenkins}{}%
Hyndman, R. J., \& Athanasopoulos, G. (2018b). \emph{Forecasting:
principles and practice}. Recuperado de
\url{https://books.google.co.cr/books?id=/_bBhDwAAQBAJ}

\leavevmode\hypertarget{ref-auto.arima}{}%
Hyndman, R., \& Khandakar, Y. (2008). Automatic Time Series Forecasting:
The forecast Package for R. \emph{Journal of Statistical Software,
Articles}, \emph{27}(3), 1-22.
\url{https://doi.org/10.18637/jss.v027.i03}

\leavevmode\hypertarget{ref-calidad_vitales}{}%
INEC. (2017). \emph{Población, nacimientos, defunciones y matrimonios}.
Recuperado de
\url{http://inec.cr/sites/default/files/documetos-biblioteca-virtual/repoblacev2017_0.pdf}

\leavevmode\hypertarget{ref-bayes}{}%
Jammalamadaka, S. R., Qiu, J., \& Ning, N. (2018). \emph{Multivariate
Bayesian Structural Time Series Model}. Recuperado de
\url{https://arxiv.org/pdf/1801.03222.pdf}

\leavevmode\hypertarget{ref-kedem}{}%
Kedem, B., \& Fokianos, K. (2005). \emph{Regression Models for Time
Series Analysis}. Recuperado de
\url{https://books.google.co.cr/books?id=8r0qE35wt44C}

\leavevmode\hypertarget{ref-Osborn2009SEASONALITYAT}{}%
Osborn, D. R., Chui, A. P. L., Smith, J., \& Birchenhall, C. (2009).
\emph{Seasonality and the order of integration for consumption}.
Recuperado de
\url{http://www.est.uc3m.es/esp/nueva_docencia/comp_col_get/lade/tecnicas_prediccion/OCSB_OxBull1988.pdf}

\leavevmode\hypertarget{ref-supenprodc}{}%
Rosero-Bixby, L. (2018). \emph{Producto C para SUPEN. Proyección de la
mortalidad de Costa Rica 2015-2150}. Recuperado de CCP-UCR website:
\url{http://srv-website.cloudapp.net/documents/10179/999061/Nota+t\%C3\%A9cnica+tablas+de+vida+segunda+parte}

\leavevmode\hypertarget{ref-redes}{}%
Tadayon, M., \& Iwashita, Y. (2020). \emph{Comprehensive Analysis of
Time Series Forecasting Using Neural Networks}. Recuperado de
\url{https://arxiv.org/pdf/2001.09547.pdf}

\leavevmode\hypertarget{ref-doi:10.1111ux2f1467-9892.00213}{}%
Xiao, Z. (2001). Testing the Null Hypothesis of Stationarity Against an
Autoregressive Unit Root Alternative. \emph{Journal of Time Series
Analysis}, \emph{22}(1), 87-105.
\url{https://doi.org/10.1111/1467-9892.00213}

\end{document}
